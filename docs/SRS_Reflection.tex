\begin{enumerate}
  \item What went well while writing this deliverable? 
  \item What pain points did you experience during this deliverable, and how did
  you resolve them?
  \item How many of your requirements were inspired by speaking to your
  client(s) or their proxies (e.g. your peers, stakeholders, potential users)?
  \item Which of the courses you have taken, or are currently taking, will help
  your team to be successful with your capstone project.
  \item What knowledge and skills will the team collectively need to acquire to
  successfully complete this capstone project?  Examples of possible knowledge
  to acquire include domain specific knowledge from the domain of your
  application, or software engineering knowledge, mechatronics knowledge or
  computer science knowledge.  Skills may be related to technology, or writing,
  or presentation, or team management, etc.  You should look to identify at
  least one item for each team member.
  \item For each of the knowledge areas and skills identified in the previous
  question, what are at least two approaches to acquiring the knowledge or
  mastering the skill?  Of the identified approaches, which will each team
  member pursue, and why did they make this choice?
\end{enumerate}

\section{Reflections}

\subsection{Amanbeer Minhas Reflection}

\begin{enumerate}
  \item \textbf{What went well while writing this deliverable?} \\
  One thing that went well while writing this deliverable was that I had a 
  much clearer understanding of what was expected compared to our previous 
  submissions. The Volere template helped a lot because it gave a clear 
  structure and description of what needed to be included in each section, 
  so I was able to focus on writing instead of figuring out the format. 
  As a team, we also communicated better and stayed consistent in how we 
  wrote different sections, which made the final document more cohesive.

  \item \textbf{What pain points did you experience during this deliverable, 
  and how did you resolve them?} \\
  The biggest challenge we faced was dividing up the work evenly across the 
  team. At first, some team members started earlier and took on more tasks, 
  while others joined later, leading to an imbalance in workload. This caused 
  some sections to be rushed near the end. To resolve this, we agreed to plan 
  task assignments ahead of time for future deliverables so that everyone has 
  clear responsibilities and the work is distributed more fairly. This will 
  help us stay more organized and avoid last-minute issues.

  \item \textbf{How many of your requirements were inspired by speaking to 
  your client(s) or their proxies?} \\
  It is hard to put an exact number on how many requirements were inspired 
  by client conversations, but several key ones were shaped by those 
  discussions. Our focus on accessibility and ease of use came directly from 
  client feedback about making the system usable for people with motor or 
  visual impairments. Additionally, requirements around safety, such as 
  confirming system-level actions before execution, were influenced by 
  stakeholder input about avoiding accidental or harmful commands.

  \item \textbf{Which of the courses you have taken, or are currently taking, 
  will help your team to be successful with your capstone project?} \\
  Several courses I have taken will directly support our work on this project. 
  Software Requirements (SFWRENG 3RA3) taught me how to gather and define 
  clear requirements, which is essential for this deliverable. Software 
  Architecture and Design courses will help with structuring the system and 
  designing modular components. Human-Computer Interfaces will be useful 
  for designing an accessible and user-friendly interface. Finally, our 
  programming and systems courses like Object-Oriented Programming and 
  Concurrent Systems Design provide the technical foundation for building 
  and connecting different parts of the system.

  \item \textbf{What knowledge and skills will the team collectively need to 
  acquire to successfully complete this capstone project?} \\
  As a team, we will need to gain deeper knowledge of how to integrate the 
  MCP (Model Context Protocol) effectively to control system-level actions. 
  We also need to improve our understanding of operating system differences 
  to make sure the solution works across Windows, macOS, and Linux. On the 
  skills side, we will need to strengthen our testing practices, improve our 
  technical writing for documentation, and enhance our team coordination and 
  time management skills. Individually, I want to focus on improving my 
  testing and integration skills, as that will be important for the reliability 
  of our system.

  \item \textbf{For each of the knowledge areas and skills identified in the 
  previous question, what are at least two approaches to acquiring the knowledge 
  or mastering the skill? Which will each team member pursue, and why did they 
  make this choice?} \\
  To improve our knowledge of MCP and system-level integration, we can study 
  official documentation and build small prototype projects to experiment with 
  its capabilities. We can also seek help from online forums and developer 
  communities to learn from real-world use cases. For cross-platform knowledge, 
  we can test our software on multiple operating systems throughout development 
  and review OS-specific guidelines. To improve testing skills, we can practice 
  writing unit and integration tests for smaller modules early in the project 
  and use tools like automated testing frameworks. For writing and documentation, 
  we plan to review examples of high-quality documentation and get feedback from 
  instructors and TAs. I will personally focus on building prototypes to learn 
  MCP integration and writing more automated tests, as these are most relevant 
  to the parts of the project I am contributing to.
\end{enumerate}

\subsection{Gourob Podder Reflection}

\begin{enumerate}
  \item \textbf{What went well while writing this deliverable?} \\
  While writing this deliverable, we were able to clearly define the functional 
  and operational requirements of our AI-powered assistive technology platform. 
  Structuring requirements as black-box and verifiable statements ensured clarity 
  and made it easier to anticipate testing criteria. Breaking down the system into 
  modular components such as input handling, agent planning, and feedback mechanisms 
  allowed us to comprehensively cover potential user interactions and system behaviors. 
  Additionally, using a structured LaTeX format helped maintain consistency across sections.

  \item \textbf{What pain points did you experience during this deliverable, and how did you resolve them?} \\
  A key challenge was ensuring that all requirements were verifiable and free of design 
  decisions. Initially, some requirements were too abstract (e.g., “the system shall 
  be user-friendly”), which was not testable. We resolved this by reframing each 
  requirement to include measurable criteria, such as task completion rates, recognition 
  accuracy, or response times. Another pain point was anticipating all potential user 
  problems and environmental constraints without making assumptions. We addressed this 
  by reviewing accessibility guidelines, examining similar assistive technologies, 
  and considering diverse user scenarios, including elderly users and people with disabilities.

  \item \textbf{How many of your requirements were inspired by speaking to your client(s) or their proxies?} \\
  We don't have an exact measure of client impact, but the core requirements were either 
  constructed or revised based on talking to potential clients. Discussions with accessibility-focused 
  community members provided insight into what tasks are most important, potential user errors, 
  and expectations for feedback mechanisms. This input was crucial for shaping functional requirements 
  such as conversational memory and clarification prompt features.

  \item \textbf{Which of the courses you have taken, or are currently taking, will help your team to be successful with your capstone project?} \\
  Several courses will directly support our work on this project: 
  \begin{itemize}
      \item \textbf{Human-Computer Interaction (4HC3):} Understanding usability principles and accessibility guidelines.  
      \item \textbf{Software Engineering Requirements (3RA3):} Writing SRS, functional and non-functional requirements, and system design.  
      \item \textbf{Operating Systems (3SH3):} Managing agent execution, concurrency, and integration with host systems.  
      \item \textbf{Databases and Networking (4C03):} Handling logging, cloud-based APIs, and data persistence for the system.  
  \end{itemize}

  \item \textbf{What knowledge and skills will the team collectively need to acquire to successfully complete this capstone project?} \\
  To successfully complete this project, the team will need:
  \begin{itemize}
      \item \textbf{Natural Language Processing (NLP) knowledge:} Understanding speech-to-text, intent recognition, and dialogue management.  
      \item \textbf{Agent-based system design:} Planning, execution, and tool integration for autonomous agents.  
      \item \textbf{Accessibility and usability skills:} Designing for diverse users, including those with disabilities or low technical proficiency.   
  \end{itemize}

  \item \textbf{For each of the knowledge areas and skills identified in the previous question, what are at least two approaches to acquiring the knowledge or mastering the skill? Which will each team member pursue, and why did they make this choice?} \\
  To acquire these skills:
  \begin{itemize}
      \item \textbf{NLP knowledge:}  
      \begin{enumerate}
          \item Complete online tutorials and courses on speech recognition and intent detection.  
          \item Implement small prototypes of text and speech processing modules for hands-on experience.  
      \end{enumerate}
      \item \textbf{Agent-based system design:}  
      \begin{enumerate}
          \item Study academic papers and open-source projects on multi-agent systems.  
          \item Develop proof-of-concept agents in a controlled environment to understand task planning and execution.  
      \end{enumerate}
      \item \textbf{Accessibility and usability skills:}  
      \begin{enumerate}
          \item Review accessibility guidelines such as WCAG 2.1 and analyze existing assistive software.  
          \item Conduct user testing sessions with peers or volunteers to identify usability issues.  
      \end{enumerate}
  \end{itemize}
  Each team member will focus on specific areas while gaining a breadth-first understanding of other areas. 
  In my case, I have prior knowledge with NLP and basic agentic systems, so I will focus on reviewing accessibility studies 
  related to software design.
\end{enumerate}

\subsection{Ajay Grewal Reflection}

\begin{enumerate}
  \item \textbf{What went well while writing this deliverable?} \\
  We agreed early on what Proxi should do, which is to turn prompts from a user into safe actions. 
  That made the writing easier and overall requirements easier to understand and write down. This 
  resulted in the sections coming together quickly. We also had good team communication, with everyone contributing ideas.
  \item \textbf{What pain points did you experience during this deliverable, and how did you resolve them?} \\
  We kept slipping into how this would be built instead of what Proxi would do. We fixed this by asking ourselves if this 
  can be ideated properly without going too deep into the technology side. We also had many LaTeX issues when writing this 
  regarding formatting. This was resolved through heavy trial and error.
  \item \textbf{How many of your requirements were inspired by speaking to your client(s) or their proxies?} \\
  It is difficult to put an exact number on this, but many important ones were from those discussions. Making Proxi accessible 
  and simple came from obvious client needs. Safety requirements, such as approving system level actions, were also heavily 
  influenced by stakeholders.
  \item \textbf{Which of the courses you have taken, or are currently taking, will help your team to be successful with your 
  capstone project?} \\
  Data structures and algorithms will help us in having efficient code and data processing. Software requirements will help in 
  defining structured requirements. Human computer interactions will help in making an effective and simple to use UI design for 
  Proxi.
  \item \textbf{What knowledge and skills will the team collectively need to acquire to successfully complete this capstone 
  project?} \\
  Better understanding of MCP tools, how to do OS automation, and how to build reliable tests/logs for our software. 
  Personally, I really want to get better at MCP and OS automation as this will define how well Proxi works. We will also need to 
  get better at writing tests and logs to ensure Proxi is reliable and safe. We also need to better work as a group to ensure the 
  code work is separated well and everyone is contributing equally.
  \item \textbf{For each of the knowledge areas and skills identified in the 
  previous question, what are at least two approaches to acquiring the knowledge 
  or mastering the skill? Which will each team member pursue, and why did they 
  make this choice?} \\
  Firstly, for this to work, a lot of trial and error will need to happen. Multiple prototypes will have to be built 
  to see what works best, furthering our understanding of how MCP, and OS automation works, as well as writing effective test 
  cases. I and each team member will also watch and read tutorials on MCP and OS automation to further understanding of these 
  topics. This way we can all equally contribute to the technical design of Proxi. 
\end{enumerate}

\subsection{Savinay Chhabra Reflection}
\begin{enumerate}
  \item \textbf{What went well while writing this deliverable?} \\
  The templates helped a lot as these explained what was expected of us. The 
  weekly team meetings also helped us with coordinate with other team members 
  for inspiration and ideas.

  \item \textbf{What pain points did you experience during this deliverable, 
  and how did you resolve them?} \\
  Dividing the work proved challenging once again. The previous deliverable
  also suffered from an imbalanced workload. The problem is that we don't know
  the time commitment each section requires until we actually do that section. 
  We split out the workload based on what we thougth were our individual 
  strengths however sections are not necessarily even in their workload. Another
  challenge was staying solution neutral while coming up with requirements.
  

  \item \textbf{How many of your requirements were inspired by speaking to 
  your client(s) or their proxies?} \\
  I don't recall excatly how many requirements were inspired by the clients, 
  but the core requirements were made after understanding the needs of older
  adults. Other requirements were inferred from general best practices and what
  we thought would make sense.

  \item \textbf{Which of the courses you have taken, or are currently taking, 
  will help your team to be successful with your capstone project?} \\
  SFWRENG 2AA4 and SFWRENG 3RA3 are two course in particular that will help our
  team to be successful. These courses include requirements and software 
  delivery which are particularly important for this course.
  

  \item \textbf{What knowledge and skills will the team collectively need to 
  acquire to successfully complete this capstone project?} \\
  As a team, we will need to learn more about MCP and how to use it to control
  the user's system. We will also need to learn about LLM integration and 
  voice to text features. We will also need to upskill in the testing domain 
  as none of us have exhaustive knowledge on end-to-end system testing.
 

  \item \textbf{For each of the knowledge areas and skills identified in the 
  previous question, what are at least two approaches to acquiring the knowledge 
  or mastering the skill? Which will each team member pursue, and why did they 
  make this choice?} \\
  For MCP, documentation and smaller projects are good approaches to familiarize
  ourselves with the technology. For LLM and Voice to speech integrations, 
  online courses and exmaple projects on github are good real world examples we 
  can learn from. Projects are always a good learning method as they provide 
  hands on learning.

\end{enumerate}