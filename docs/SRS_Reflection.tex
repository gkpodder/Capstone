\begin{enumerate}
  \item What went well while writing this deliverable? 
  \item What pain points did you experience during this deliverable, and how did
  you resolve them?
  \item How many of your requirements were inspired by speaking to your
  client(s) or their proxies (e.g. your peers, stakeholders, potential users)?
  \item Which of the courses you have taken, or are currently taking, will help
  your team to be successful with your capstone project.
  \item What knowledge and skills will the team collectively need to acquire to
  successfully complete this capstone project?  Examples of possible knowledge
  to acquire include domain specific knowledge from the domain of your
  application, or software engineering knowledge, mechatronics knowledge or
  computer science knowledge.  Skills may be related to technology, or writing,
  or presentation, or team management, etc.  You should look to identify at
  least one item for each team member.
  \item For each of the knowledge areas and skills identified in the previous
  question, what are at least two approaches to acquiring the knowledge or
  mastering the skill?  Of the identified approaches, which will each team
  member pursue, and why did they make this choice?
\end{enumerate}

\section{Amanbeer Minhas Reflection}

\begin{enumerate}
  \item \textbf{What went well while writing this deliverable?} \\
  One thing that went well while writing this deliverable was that I had a 
  much clearer understanding of what was expected compared to our previous 
  submissions. The Volere template helped a lot because it gave a clear 
  structure and description of what needed to be included in each section, 
  so I was able to focus on writing instead of figuring out the format. 
  As a team, we also communicated better and stayed consistent in how we 
  wrote different sections, which made the final document more cohesive.

  \item \textbf{What pain points did you experience during this deliverable, 
  and how did you resolve them?} \\
  The biggest challenge we faced was dividing up the work evenly across the 
  team. At first, some team members started earlier and took on more tasks, 
  while others joined later, leading to an imbalance in workload. This caused 
  some sections to be rushed near the end. To resolve this, we agreed to plan 
  task assignments ahead of time for future deliverables so that everyone has 
  clear responsibilities and the work is distributed more fairly. This will 
  help us stay more organized and avoid last-minute issues.

  \item \textbf{How many of your requirements were inspired by speaking to 
  your client(s) or their proxies?} \\
  It is hard to put an exact number on how many requirements were inspired 
  by client conversations, but several key ones were shaped by those 
  discussions. Our focus on accessibility and ease of use came directly from 
  client feedback about making the system usable for people with motor or 
  visual impairments. Additionally, requirements around safety, such as 
  confirming system-level actions before execution, were influenced by 
  stakeholder input about avoiding accidental or harmful commands.

  \item \textbf{Which of the courses you have taken, or are currently taking, 
  will help your team to be successful with your capstone project?} \\
  Several courses I have taken will directly support our work on this project. 
  Software Requirements (SFWRENG 3RA3) taught me how to gather and define 
  clear requirements, which is essential for this deliverable. Software 
  Architecture and Design courses will help with structuring the system and 
  designing modular components. Human-Computer Interfaces will be useful 
  for designing an accessible and user-friendly interface. Finally, our 
  programming and systems courses like Object-Oriented Programming and 
  Concurrent Systems Design provide the technical foundation for building 
  and connecting different parts of the system.

  \item \textbf{What knowledge and skills will the team collectively need to 
  acquire to successfully complete this capstone project?} \\
  As a team, we will need to gain deeper knowledge of how to integrate the 
  MCP (Model Context Protocol) effectively to control system-level actions. 
  We also need to improve our understanding of operating system differences 
  to make sure the solution works across Windows, macOS, and Linux. On the 
  skills side, we will need to strengthen our testing practices, improve our 
  technical writing for documentation, and enhance our team coordination and 
  time management skills. Individually, I want to focus on improving my 
  testing and integration skills, as that will be important for the reliability 
  of our system.

  \item \textbf{For each of the knowledge areas and skills identified in the 
  previous question, what are at least two approaches to acquiring the knowledge 
  or mastering the skill? Which will each team member pursue, and why did they 
  make this choice?} \\
  To improve our knowledge of MCP and system-level integration, we can study 
  official documentation and build small prototype projects to experiment with 
  its capabilities. We can also seek help from online forums and developer 
  communities to learn from real-world use cases. For cross-platform knowledge, 
  we can test our software on multiple operating systems throughout development 
  and review OS-specific guidelines. To improve testing skills, we can practice 
  writing unit and integration tests for smaller modules early in the project 
  and use tools like automated testing frameworks. For writing and documentation, 
  we plan to review examples of high-quality documentation and get feedback from 
  instructors and TAs. I will personally focus on building prototypes to learn 
  MCP integration and writing more automated tests, as these are most relevant 
  to the parts of the project I am contributing to.
\end{enumerate}
