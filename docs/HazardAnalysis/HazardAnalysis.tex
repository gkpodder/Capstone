\documentclass{article}

\usepackage{booktabs}
\usepackage{tabularx}
\usepackage{hyperref}
\usepackage{longtable}
\usepackage{array}
\usepackage{geometry}
\usepackage{float} 

\hypersetup{
    colorlinks=true,       % false: boxed links; true: colored links
    linkcolor=red,          % color of internal links (change box color with linkbordercolor)
    citecolor=green,        % color of links to bibliography
    filecolor=magenta,      % color of file links
    urlcolor=cyan           % color of external links
}

\title{Hazard Analysis\\\progname}

\author{\authname}

\date{}

\input{../Comments}
%% Common Parts

\newcommand{\progname}{Software Engineering} % PUT YOUR PROGRAM NAME HERE
\newcommand{\authname}{Team \#23, Project Proxi
\\ Savinay Chhabra
\\ Amanbeer Singh Minhas
\\ Gourob Podder
\\ Ajay Grewal} % AUTHOR NAMES                  

\usepackage{hyperref}
    \hypersetup{colorlinks=true, linkcolor=blue, citecolor=blue, filecolor=blue,
                urlcolor=blue, unicode=false}
    \urlstyle{same}
                                


\begin{document}

\maketitle
\thispagestyle{empty}

~\newpage

\pagenumbering{roman}

\begin{table}[hp]
\caption{Revision History} \label{TblRevisionHistory}
\begin{tabularx}{\textwidth}{llX}
\toprule
\textbf{Date} & \textbf{Developer(s)} & \textbf{Change}\\
\midrule
10/10/2025 & Amanbeer Minhas & Added Sections 2, 4 and reflection\\
10/10/2025 & Gourob Podder & Added Sections 1, 3 and reflection\\
10/10/2025 & Savinay Chhabra & Added Sections 5 and reflection\\
10/10/2025 & Ajay Singh Grewal & Added Sections 6, 7 and reflection\\

\bottomrule
\end{tabularx}
\end{table}

~\newpage

\tableofcontents

~\newpage

\pagenumbering{arabic}

\section{Introduction}

In the context of this project, a hazard is defined as any potential source of 
harm or adverse effect that could result from the use, misuse, or failure of the 
AI-powered assistive technology platform. 
Hazards may arise from interactions between the system and the user, the system 
and the computing environment, or from internal system errors. 
They can manifest as physical, cognitive, or informational risks, and may affect 
user safety, privacy, accessibility, or the integrity of computer operations.

By documenting hazards in a structured and traceable manner, the development team 
can make informed design and operational decisions, while also providing guidance 
for future maintenance, updates, and testing of the system.

\section{Scope and Purpose of Hazard Analysis}
The purpose of this hazard analysis is to identify and evaluate possible
hazards linked to the Proxi system, a voice and text driven assistant
designed to improve accessibility and productivity for all users.
The scope of this analysis includes the software components of Proxi such
as the voice and text input handling, MCP integration, system automation
features, and interaction with the desktop operating system. It also
covers potential issues related to data handling, user commands, and
accessibility features. Hardware failures, external network infrastructure,
and third-party service outages are outside the scope of this analysis.
The main losses that could occur due to these hazards include accidental
file deletion or modification, loss or exposure of user data, compromised
accessibility for people with disabilities, and reduced user trust caused
by incorrect or misleading actions. These issues could also lead to major
productivity loss if important commands fail, run incorrectly, or if key
accessibility features stop working as expected. Even though Proxi does
not control hardware or other safety-critical devices, failures in the
software could still cause real harm through data loss, security breaches,
inefficiency, or user frustration.
The goal of this hazard analysis is to find and understand these risks
early in the project so that effective prevention and mitigation
strategies can be planned and built into the design.


\section{System Boundaries and Components}

\subsection{System Boundaries}

The boundaries of the system are defined by its scope of control and interaction:
\begin{itemize}
    \item The system accepts user input through \textbf{speech} or \textbf{text} 
    interfaces.
    \item It processes these inputs internally using natural language understanding 
    and task planning components.
    \item It interacts with external computer applications and operating system tools
     to perform user-requested actions (e.g., sending emails, opening documents, managing files).
    \item It outputs responses or task results back to the user through \textbf{text} 
    or \textbf{speech synthesis}.
\end{itemize}

Elements outside the system boundary include:
\begin{itemize}
    \item The underlying \textbf{operating system} and \textbf{hardware} on which 
    the platform runs.
    \item Third-party applications or APIs that the platform may invoke.
    \item The user’s physical hardware devices such as microphones, speakers, or 
    assistive peripherals.
    \item Network infrastructure and external AI services (if used for speech 
    recognition or task execution).
\end{itemize}

\subsection{Major System Components}

The system can be conceptually divided into the following major components:

\begin{enumerate}
    \item \textbf{User Interface Module (UI):}
    Handles all communication with the user, including speech-to-text (STT) input, text input, and text-to-speech (TTS) or visual feedback output.

    \item \textbf{Natural Language Understanding (NLU) Engine:}
    Interprets user intent from raw speech or text input and converts it into structured commands for the planning module.

    \item \textbf{Task Planning and Agent Manager:}
    Responsible for decomposing user intent into executable subtasks, invoking specialized software agents, and monitoring task progress.

    \item \textbf{Tool and System Integration Layer:}
    Provides controlled access to the computer’s applications and resources (e.g., file manager, web browser, email client) through APIs or system commands.

    \item \textbf{Knowledge Base and Context Manager:}
    Maintains context about the user’s current session, preferences, and system state to support coherent interactions and task continuity.

    \item \textbf{Data Storage and Logging Component:}
    Stores session logs, error reports, and user preferences while ensuring compliance with privacy and data protection requirements.

    \item \textbf{Security and Permissions Module:}
    Manages access control, ensures safe execution of actions on behalf of the user, and prevents unauthorized or unsafe operations.
\end{enumerate}

These components collectively form the system boundary for hazard analysis. Hazards may arise from failures, misuse, or unexpected interactions among these components or between the system and its external environment.

\section{Critical Assumptions}
The hazard analysis is based on a few key assumptions about how Proxi will
be used and the environment it will run in:
\begin{enumerate}
\item The system will run on supported desktop operating systems like
Windows, macOS, or Linux with the needed permissions already set.
\item Users will give clear and intentional commands. Some commands may be
misunderstood, but we do not assume the system will be used in a harmful way.
\item The MCP integration layer, system APIs, and other tools Proxi works
with are expected to behave as described, even though failures are still
possible.
\item Network issues may slow down some features, but basic local actions
and automation will still work without an internet connection.
\item A working microphone or similar input device is expected to be
available and set up correctly for voice input on the user's device.
\end{enumerate}
These assumptions help us focus on realistic hazards and plan how to handle
them. They guide the analysis and reflect how Proxi is expected to work in
everyday use.



\section{Failure Mode and Effect Analysis}

\begin{table}[H]
\centering
\begin{tiny}
\setlength{\tabcolsep}{2pt}

\caption{Failure Modes and Effects Analysis (FMEA) Table}
\label{tab:fmea}

\begin{tabularx}{\textwidth}{
    |>{\raggedright\arraybackslash}p{1.2cm}
    |>{\raggedright\arraybackslash}X
    |>{\raggedright\arraybackslash}X
    |c 
    |>{\raggedright\arraybackslash}X 
    |c
    |c
    |>{\raggedright\arraybackslash}p{2.3cm}
    |c
    |c|
}
\toprule
\multicolumn{1}{|c|}{\textbf{Component}} &
\multicolumn{1}{c|}{\textbf{Failure Mode}} &
\multicolumn{1}{c|}{\textbf{Effect of Failure}} &
\multicolumn{1}{c|}{\textbf{S}} & % Abbreviated
\multicolumn{1}{c|}{\textbf{Cause of Failure}} &
\multicolumn{1}{c|}{\textbf{O}} & % Abbreviated
\multicolumn{1}{c|}{\textbf{D}} & % Abbreviated
\multicolumn{1}{c|}{\textbf{Recommended Action}} &
\multicolumn{1}{c|}{\textbf{SR Ref.}} &
\multicolumn{1}{c|}{\textbf{Ref. ID}} \\
\midrule

Speech-to-Text (STT)&Misinterprets voice commands&Wrong action or no action 
taken&8&Background noise, accent, poor mic&5&6&Improve model, noise filtering, 
allow correction&POA.R.1&H1-1\\ \hline

Command Interpreter&Fails to understand user intent&Task not completed or 
incorrect action&7&Ambiguous phrasing or missing context&4&6&Add clarification 
prompts, context handling&FUNC.R.3&H2-1\\ \hline

MCP Integration Layer&Wrong tool triggered&Unintended app or function 
runs&8&Incorrect mapping, version mismatch&4&6&Validate inputs, add fallback 
defaults&SAF.R.2&H3-1\\ \hline
OS Automation&Command crashes or freezes&User tasks fail or system 
slows&6&Permission errors, unresponsive apps&5&5&Retry logic, timeouts, 
graceful error handling&ROFT.R.2&H4-1\\ \hline

Confirmation Layer&Fails to ask for confirmation&Accidental deletion or system 
changes&9&Logic bug or skipped validation&3&7&Mandatory confirmation before 
high-risk actions&SAF.R.1&H5-1\\ \hline

Accessibility Layer&Screen reader or assistive tools fail&Users with 
disabilities cannot use system&9&Missing WCAG compliance, poor ARIA&4&6&Test 
with AT, comply with WCAG 2.2 and AODA&ACC.R.2&H6-1\\ \hline

Text-to-Speech Feedback&Response delayed or missing&User repeats commands or is 
confused&5&High load, slow network, TTS errors&5&5&Optimize response, add status 
indicator&SAL.R.1&H7-1\\ \hline

Update \& Deployment&New release breaks features&Commands stop working&6&Lack of
 regression testing&4&5&Automated tests, rollback support&LON.R.2&H8-1\\ \hline

Logging System&Missing logs or corrupted data&Harder to debug and trace 
issues&5&Logging disabled or storage full&4&5&Continuous logging, storage 
monitoring&ROFT.R.1&H9-1\\ \hline

OS Compatibility&Inconsistent performance across OS&Features fail on some 
platforms&7&API differences, OS permissions&4&5&Multi-OS testing, abstraction 
layers&EPE.R.1&H10-1\\ \hline

\end{tabularx}

\vspace{10pt}

% Legend
\begin{tabular}{ll}
\textbf{Legend:} \\
\textbf{S}: & Severity (1-10) \\
\textbf{O}: & Occurrence (1-10) \\
\textbf{D}: & Detection (1-10) \\
\end{tabular}

\end{tiny}
\end{table}

\section{Safety and Security Requirements}

\subsection{Safety Requirements}
Although Proxi doesn't interact with any hardware, a user's files or settings 
could still be impacted by a poorly written or confusing command. We'll keep 
things straightforward and cautious to keep people safe:

\begin{description}
  \item[SAS.R.1 — Confirm before risky changes.]  
  For anything that can alter files or sensitive settings we should 
  show a short message asking the user to press the approve button 
  before proceeding.

  \emph{Why:} This prevents any accidental damage from such requests.  

  \emph{How we’ll check:} In testing, every destructive or high level action 
  should have an approval.

  \item[SAS.R.2 — Easy undo for common file actions.]  
  When possible, there should be a simple undo option that uses the OS recycle 
  bin or copy-back.

  \emph{Why:} This way there is recovery if Proxi makes a mistake. 

  \emph{How we’ll check:} Run tests that can restore the original file in
  at at least 7/10 of cases.

  \item[SAS.R.3 — Least privilege tool scopes.]  
  MCP tools run inside only folders and apps the user has chosen to allow.

  \emph{Why:} This limits from any bad commands affecting unrelated areas.

  \emph{How we’ll check:} Make sure that any attempt to access something out of 
  scope is first requested from the user.

  \item[SAS.R.4 — Pause on low confidence.]  
  If the confidence is low or there is any confusion from Proxi, it can ask a 
  short question to the user before continuing forward.

  \emph{Why:} Don’t guess when unsure it is better to ask if it is safe to 
  proceed.  

  \emph{How we’ll check:} Can test and experiment with low-confidence inputs 
  that never run without clarification/approval.
\end{description}

\subsection{Security Requirements}
We also want to avoid common security issues:

\begin{description}
  \item[SC.R.1 — Validate inputs and use allow lists.]  
  Have clean paths, URLs, and arguments and only allow safe protocols. 
  Random directory traversal should be blocked.

  \emph{Why:} This stops not needed injections and any path traversal bugs.  

  \emph{How we’ll check:} Can run tests that will check to see if there are any 
  injection or traversing issues.

  \item[SC.R.2 — Ask before sending data out.]  
  Sending personal data needs consent from the user, and we should only send the
  least amount of data as possible. Only data needed to complete the task should
  be sent.

  \emph{Why:} This way the user controls what leaves their computer.

  \emph{How we’ll check:} Test flows can show an approval step before any data 
  is sent.

  \item[SC.R.3 — Encrypt everything in transit.]  
  Use TLS/HTTPS for all network traffic to maximize security.

  \emph{Why:} Protects against eavesdropping and man in the middle attacks.

  \emph{How we’ll check:} The packet captures will show HTTPS only traffic. Any 
  bad certifications will get simply rejected.

  \item[SC.R.4 — Keep dependencies clean.]  
  Keep libraries up to date and scan libraries before releasing for any 
  vulnerabilities

  \emph{Why:} This minimizes the risk from any known dependency vulnerabilities.

  \emph{How we’ll check:} Release checklist should have 0 major vulnerabilities.
\end{description}

\section{Roadmap}

\subsection*{What we’ll ship during the capstone}
These are the highest-impact items for safety and trust.

\begin{table}[!h]
\centering
\renewcommand{\arraystretch}{1.15}
\begin{tabularx}{\textwidth}{l X l}
\toprule
\textbf{ID} & \textbf{What we’ll build} & \textbf{When} \\
\midrule
SAS.R.1 & Confirmation step for destructive/system actions & Sprint 1 \\
SAS.R.4 & Clarify when intent confidence is low & Sprint 1 \\
SC.R.1 & Input validation + allow-lists in all adapters & Sprint 1–2 \\
SC.R.2 & Consent flow for any external/network calls & Sprint 2 \\
SAS.R.2 & One-step undo for file moves/renames/deletes & Sprint 2 \\
SC.R.3 & HTTPS-only networking with certificate checks & Sprint 2 \\
\bottomrule
\end{tabularx}
\end{table}

\subsection*{Deferred / later work}
Still useful, but can land after the demo if time is tight.

\begin{table}[!h]
\centering
\renewcommand{\arraystretch}{1.15}
\begin{tabularx}{\textwidth}{l X X}
\toprule
\textbf{ID} & \textbf{What} & \textbf{Why later} \\
\midrule
SAS.R.3 & Deeper OS sandboxing beyond path scoping & More engineering effort; 
too complicated for current scope \\
SC.R.4 & Automated vulnerabilities scanning in CI and signed releases & Better 
for long-term upkeep; not needed right now \\
\bottomrule
\end{tabularx}
\end{table}

\newpage{}

\section*{Appendix --- Reflection}

\input{../Reflection.tex}

\begin{enumerate}
    \item What went well while writing this deliverable? 
    \item What pain points did you experience during this deliverable, and how
    did you resolve them?
    \item Which of your listed risks had your team thought of before this
    deliverable, and which did you think of while doing this deliverable? For
    the latter ones (ones you thought of while doing the Hazard Analysis), how
    did they come about?
    \item Other than the risk of physical harm (some projects may not have any
    appreciable risks of this form), list at least 2 other types of risk in
    software products. Why are they important to consider?
\end{enumerate}

\section{Reflections}

\subsection{Amanbeer Minhas Reflection}

\begin{enumerate}
\item What went well while writing this deliverable?\\
Writing this deliverable went well because we already had a clear idea of
our system from the SRS and FMEA work. Breaking the project into components
helped me think about hazards more systematically and kept the analysis
organized. I also felt more confident in identifying potential risks and
writing about them in a structured way.

\item What pain points did you experience during this deliverable, and how
did you resolve them?\\
One challenge was figuring out how detailed to make each section without
overcomplicating the document. I also found it tricky to write about hazards
without repeating the same points as the SRS. I resolved this by focusing on
specific examples from our project, like MCP integration and OS automation,
and connecting them directly to potential risks.

\item Which of your listed risks had your team thought of before this
deliverable, and which did you think of while doing this deliverable? For
the latter ones (ones you thought of while doing the Hazard Analysis), how
did they come about?\\
We had already thought about risks like incorrect command execution and
permission issues earlier in the project. However, risks such as operating
system compatibility problems and deployment-related failures came up while
writing this hazard analysis. They became more obvious once we broke the
system into components and looked deeper into how each part could fail.

\item Other than the risk of physical harm (some projects may not have any
appreciable risks of this form), list at least 2 other types of risk in
software products. Why are they important to consider?\\
One major risk is data loss or corruption, which can harm user trust and
make the system unreliable. Another risk is accessibility failure, where
users with limitations might not be able to use the software as intended.
Both are important to consider because they affect user confidence,
usability, and the overall success of the product.
\end{enumerate}

\subsection{Gourob Podder Reflection}

\begin{enumerate}
    \item What went well while writing this deliverable?\\
    Identifying system boundaries and major components went smoothly because the project architecture was already clear from prior SRS work. Defining modules like the User Interface, Natural Language Understanding, and Agent Manager helped organize potential hazards systematically. Team collaboration also allowed us to quickly align on how risks could manifest in both software and user interactions.

    \item What pain points did you experience during this deliverable, and how did you resolve them?\\
    Determining the appropriate level of detail for the hazard analysis without drifting into system design was challenging. It was sometimes difficult to distinguish between a hazard and a general usability concern. We resolved this by using safety engineering definitions and focusing on sources of harm or adverse outcomes rather than design flaws. Abstract risks such as privacy or misinterpretation of user intent were addressed using scenario-based brainstorming.

    \item Which of your listed risks had your team thought of before this deliverable, and which did you think of while doing this deliverable? For the latter ones, how did they come about?\\
    We had already recognized usability and privacy risks, like misinterpreted commands or unauthorized data access. New risks emerged during the hazard analysis, especially related to component interactions, such as potential failures in the Task Planning module or dependency on unreliable external APIs. These became apparent as we examined each subsystem and its interface boundaries.

    \item Other than the risk of physical harm, list at least 2 other types of risk in software products. Why are they important to consider?\\
    Privacy and data security risks, such as improper handling of sensitive user data, can harm user trust and violate legal requirements. Reliability and availability risks, like system downtime or malfunctions, can prevent users from completing essential tasks, especially for assistive technologies. Both are critical for maintaining usability, accessibility, and overall user confidence.
\end{enumerate}

\subsection{Savinay Chhabra Reflection}

\begin{enumerate}
    \item What went well while writing this deliverable?\\
    The provided template helped structure our hazards. Collaboration and work distribution went more smoothly, and we could relate hazards directly to the well-structured SRS document.

    \item What pain points did you experience during this deliverable, and how did you resolve them?\\
    Staying solution-neutral while analyzing hazards was difficult. The line between requirements and design is blurred, and we sometimes discussed solutions prematurely. We resolved this by refocusing on hazards rather than proposed solutions.

    \item Which of your listed risks had your team thought of before this deliverable, and which did you think of while doing this deliverable? For the latter ones, how did they come about?\\
    Initially, we focused on privacy and legal risks associated with data breaches. While performing hazard analysis, we discovered additional potential risks and mitigation strategies that were not previously considered.

    \item Other than the risk of physical harm, list at least 2 other types of risk in software products. Why are they important to consider?\\
    Legal risks due to data breaches or violations of user privacy, and compliance risks from accessibility failures in design. Both are important because they affect user trust, legal standing, and the usability of the product.
\end{enumerate}


\end{document}