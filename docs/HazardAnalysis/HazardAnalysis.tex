\documentclass{article}

\usepackage{booktabs}
\usepackage{tabularx}
\usepackage{hyperref}

\hypersetup{
    colorlinks=true,       % false: boxed links; true: colored links
    linkcolor=red,          % color of internal links (change box color with linkbordercolor)
    citecolor=green,        % color of links to bibliography
    filecolor=magenta,      % color of file links
    urlcolor=cyan           % color of external links
}

\title{Hazard Analysis\\\progname}

\author{\authname}

\date{}

\input{../Comments}
%% Common Parts

\newcommand{\progname}{Software Engineering} % PUT YOUR PROGRAM NAME HERE
\newcommand{\authname}{Team \#23, Project Proxi
\\ Savinay Chhabra
\\ Amanbeer Singh Minhas
\\ Gourob Podder
\\ Ajay Grewal} % AUTHOR NAMES                  

\usepackage{hyperref}
    \hypersetup{colorlinks=true, linkcolor=blue, citecolor=blue, filecolor=blue,
                urlcolor=blue, unicode=false}
    \urlstyle{same}
                                


\begin{document}

\maketitle
\thispagestyle{empty}

~\newpage

\pagenumbering{roman}

\begin{table}[hp]
\caption{Revision History} \label{TblRevisionHistory}
\begin{tabularx}{\textwidth}{llX}
\toprule
\textbf{Date} & \textbf{Developer(s)} & \textbf{Change}\\
\midrule
Date1 & Name(s) & Description of changes\\
Date2 & Name(s) & Description of changes\\
... & ... & ...\\
\bottomrule
\end{tabularx}
\end{table}

~\newpage

\tableofcontents

~\newpage

\pagenumbering{arabic}

\wss{You are free to modify this template.}

\section{Introduction}

\wss{You can include your definition of what a hazard is here.}

\section{Scope and Purpose of Hazard Analysis}
The purpose of this hazard analysis is to identify and evaluate potential
hazards associated with the Proxi system, a voice and text driven assistant 
designed to improve accessibility and productivity. The main losses that 
could occur due to these hazards include unintended system actions such as 
accidental file deletion or modification, loss or exposure of user data, 
compromised accessibility for users with disabilities, and reduced user 
trust caused by incorrect or misleading behavior. These hazards could also
result in significant productivity loss if critical commands fail, execute 
incorrectly, or if accessibility features do not function as expected. 
Although Proxi does not control hardware or safety-critical equipment, 
its software failures could still lead to real harm in the form of data loss,
security breaches, inefficiency, or user frustration. The goal of this analysis
is to understand these risks early in development and plan effective ways to 
prevent or reduce them.

\section{System Boundaries and Components}

\wss{Dividing the system into components will help you brainstorm the hazards.
You shouldn't do a full design of the components, just get a feel for the major
ones.  For projects that involve hardware, the components will typically include
each individual piece of hardware.  If your software will have a database, or an
important library, these are also potential components.}

\section{Critical Assumptions}
The hazard analysis is based on a few key assumptions about how Proxi will
be used and the environment it will run in:
\begin{enumerate}
\item The system will run on supported desktop operating systems like
Windows, macOS, or Linux with the needed permissions already set.
\item Users will give clear and intentional commands. Some commands may be
misunderstood, but we do not assume the system will be used in a harmful way.
\item The MCP integration layer, system APIs, and other tools Proxi works
with are expected to behave as described, even though failures are still
possible.
\item Network issues may slow down some features, but basic local actions
and automation will still work without an internet connection.
\item A working microphone or similar input device is expected to be
available and set up correctly for voice input on the user's device.
\end{enumerate}
These assumptions help us focus on realistic hazards and plan how to handle
them. They guide the analysis and reflect how Proxi is expected to work in
everyday use.



\section{Failure Mode and Effect Analysis}

\wss{Include your FMEA table here. This is the most important part of this document.}
\wss{The safety requirements in the table do not have to have the prefix SR.
The most important thing is to show traceability to your SRS. You might trace to
requirements you have already written, or you might need to add new
requirements.}
\wss{If no safety requirement can be devised, other mitigation strategies can be
entered in the table, including strategies involving providing additional
documentation, and/or test cases.}

\section{Safety and Security Requirements}

\wss{Newly discovered requirements.  These should also be added to the SRS.  (A
rationale design process how and why to fake it.)}

\section{Roadmap}

\wss{Which safety requirements will be implemented as part of the capstone timeline?
Which requirements will be implemented in the future?}

\newpage{}

\section*{Appendix --- Reflection}

\wss{Not required for CAS 741}

\input{../Reflection.tex}

\begin{enumerate}
    \item What went well while writing this deliverable? 
    \item What pain points did you experience during this deliverable, and how
    did you resolve them?
    \item Which of your listed risks had your team thought of before this
    deliverable, and which did you think of while doing this deliverable? For
    the latter ones (ones you thought of while doing the Hazard Analysis), how
    did they come about?
    \item Other than the risk of physical harm (some projects may not have any
    appreciable risks of this form), list at least 2 other types of risk in
    software products. Why are they important to consider?
\end{enumerate}

\section*{Reflection – Amanbeer Minhas}

\begin{enumerate}
\item What went well while writing this deliverable?\\
Writing this deliverable went well because we already had a clear idea of
our system from the SRS and FMEA work. Breaking the project into components
helped me think about hazards more systematically and kept the analysis
organized. I also felt more confident in identifying potential risks and
writing about them in a structured way.

\item What pain points did you experience during this deliverable, and how
did you resolve them?\\
One challenge was figuring out how detailed to make each section without
overcomplicating the document. I also found it tricky to write about hazards
without repeating the same points as the SRS. I resolved this by focusing on
specific examples from our project, like MCP integration and OS automation,
and connecting them directly to potential risks.

\item Which of your listed risks had your team thought of before this
deliverable, and which did you think of while doing this deliverable? For
the latter ones (ones you thought of while doing the Hazard Analysis), how
did they come about?\\
We had already thought about risks like incorrect command execution and
permission issues earlier in the project. However, risks such as operating
system compatibility problems and deployment-related failures came up while
writing this hazard analysis. They became more obvious once we broke the
system into components and looked deeper into how each part could fail.

\item Other than the risk of physical harm (some projects may not have any
appreciable risks of this form), list at least 2 other types of risk in
software products. Why are they important to consider?\\
One major risk is data loss or corruption, which can harm user trust and
make the system unreliable. Another risk is accessibility failure, where
users with limitations might not be able to use the software as intended.
Both are important to consider because they affect user confidence,
usability, and the overall success of the product.
\end{enumerate}


\end{document}