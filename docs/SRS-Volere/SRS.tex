% THIS DOCUMENT IS FOLLOWS THE VOLERE TEMPLATE BY Suzanne Robertson and James Robertson
% ONLY THE SECTION HEADINGS ARE PROVIDED
%
% Initial draft from https://github.com/Dieblich/volere
%
% Risks are removed because they are covered by the Hazard Analysis
\documentclass[12pt]{article}

\usepackage{booktabs}
\usepackage{tabularx}
\usepackage{hyperref}
\usepackage{graphicx}
\usepackage{float}
\hypersetup{
    bookmarks=true,         % show bookmarks bar?
      colorlinks=true,      % false: boxed links; true: colored links
    linkcolor=red,          % color of internal links (change box color with linkbordercolor)
    citecolor=green,        % color of links to bibliography
    filecolor=magenta,      % color of file links
    urlcolor=cyan           % color of external links
}

\newcommand{\lips}{\textit{Insert your content here.}}

\input{../Comments}
%% Common Parts

\newcommand{\progname}{Software Engineering} % PUT YOUR PROGRAM NAME HERE
\newcommand{\authname}{Team \#23, Project Proxi
\\ Savinay Chhabra
\\ Amanbeer Singh Minhas
\\ Gourob Podder
\\ Ajay Grewal} % AUTHOR NAMES                  

\usepackage{hyperref}
    \hypersetup{colorlinks=true, linkcolor=blue, citecolor=blue, filecolor=blue,
                urlcolor=blue, unicode=false}
    \urlstyle{same}
                                


\begin{document}

\title{Software Requirements Specification for \progname: AI Voice Assistant} 
\author{\authname}
\date{\today}
	
\maketitle

~\newpage

\pagenumbering{roman}

\tableofcontents

~\newpage

\section*{Revision History}

\begin{tabularx}{\textwidth}{p{3cm}p{2cm}X}
\toprule {\textbf{Date}} & {\textbf{Version}} & {\textbf{Notes}}\\
\midrule
10/10/2025 & 1.0 & Added Initial SRS\\
\bottomrule
\end{tabularx}

~\\

~\newpage
\section{Purpose of the Project}
\subsection{User Business}
Proxi is an AI-powered desktop assistant that lets people operate a computer
entirely through natural speech. It targets users who face barriers with traditional
input devices (keyboard, mouse, complex UIs) and organizations that want to provide
inclusive access to essential digital tasks (communication, learning, work). Proxi
augments independence and reduces the digital divide by turning voice into safe,
precise computer actions. While accessibility is the primary driver, Proxi is
equally intended for general users who want faster, lower‑friction workflows so it
benefits both disabled and non‑disabled users.

\subsection{Goals of the Project}
\begin{description}
\item[G-1 (Latency)] Spoken system responses for common commands shall begin within 
$\leq$ 2.0 s from end-of-speech in a quiet environment.

\item[G-2 (Recognition Accuracy)] Command recognition accuracy for supported 
language(s) in quiet environment shall be $\geq$ 90\% intent-level accuracy.

\item[G-3 (Task Coverage)] Users from the primary user group shall complete $\geq$ 
80\% of a predefined core task suite (open app/file, browse, compose, save, 
schedule) using voice or keyboard.

\item[G-4 (Effectiveness)] Compared to baseline (traditional input for same users), 
Proxi shall improve task completion rate by $\geq$ 20\%.

\item[G-5 (Satisfaction)] Accessibility-focused usability tests shall yield 4.0/5.0 satisfaction score.

\item[G-6 (Stretch Goals)] Voice recognition improvements, offline capabilities, 
multimodal interaction support, personalized profiles, enhanced accessibility
\end{description}
\section{Stakeholders}
\subsection{Client}
The clients for this project are the SFWRENG 4G06A Capstone teaching team at 
McMaster University (course Instructor and assigned Teaching Assistants), serving 
as the product owners on behalf of the department. Their mandate is to ensure the 
solution meets accessibility, usability, and engineering quality standards 
appropriate for a capstone deliverable and potential real‑world piloting within 
academic environments. They provide domain expectations (accessibility best 
practices, privacy/compliance constraints in academic settings) and approve scope 
and milestones.
\subsection{Customer}
Our customers are the end-users and organizations that will deploy Proxi to enable
inclusive and more efficient computer use:

\begin{enumerate}
  \item \textbf{Educational institutions} (libraries, computer labs, accessibility
  services) seeking hands-free or low-friction access to standard desktops and web
  apps.
  \item \textbf{Healthcare and community organizations} supporting users with motor,
  vision, or hearing challenges.
  \item \textbf{General consumers and power users} who prefer faster, voice-first or
  mixed-modality workflows.
\end{enumerate}

\subsection{Other Stakeholders}
Other stakeholders include any person or group with interest beyond the client and
the customer:

\begin{enumerate}
  \item \textbf{Team Proxi (development team):} responsible for requirements, design,
  implementation, testing, and deployment artefacts.
  \item \textbf{Course Staff (Instructor \& TA):} guidance, assessment, feedback, and
  approvals.
  \item \textbf{Accessibility Advisors (if engaged):} best practices for WCAG/AT
  compatibility.
  \item \textbf{Pilot participants:} individuals who will use the system during user
  studies and provide feedback.
\end{enumerate}

\subsection{Hands-On Users of the Project}
Primary hands-on users who will directly interact with Proxi:

\begin{enumerate}
  \item \textbf{Accessibility-focused users:} People with motor impairments or
  temporary/situational limitations needing hands-free or simplified control.
  \item \textbf{General users/power users:} Users seeking faster workflows via
  voice/text commands with optional keyboard/mouse confirmation.
\end{enumerate}

\subsection{Personas}

\begin{enumerate}
  \item \textbf{P1: Amrita (72) - Elderly User:} Amrita is a retired teacher who uses
  her desktop to check emails, pay bills, and video call her family. She struggles
  with small buttons, complicated menus, and remembering multi-step actions, which
  makes her anxious about using technology. She needs a way to perform common tasks
  more easily and with clear guidance to feel confident online.

  \item \textbf{P2: Leo (21) - User with Motor Disability:} Leo is a computer science
  student who finds it difficult to use a keyboard or mouse due to a motor
  impairment. He needs to read PDFs, take notes, and switch between different apps
  for his coursework. He requires a way to interact with his computer hands-free and
  complete his work without relying on physical input.

  \item \textbf{P3: Ari (28) - Power User:} Ari works with spreadsheets, emails, and
  web tools throughout the day and often repeats the same steps over and over.
  Switching between programs slows him down, and remembering different commands and
  shortcuts is frustrating. He needs a more efficient way to complete multi-step
  tasks and manage his work without constant interruptions.
\end{enumerate}

\subsection{Priorities Assigned to Users}
The \textbf{highest priority} users for this project are \textbf{accessibility-focused} users, 
including elderly users like Amrita and users with motor impairments like Leo, as 
the main goal is to improve computer access and usability for people who face 
physical challenges or find current systems difficult to use. Their needs guide 
the core features and design decisions of the project. \textbf{Power users} like Ari are 
the \textbf{secondary priority} because, while they do not face accessibility barriers, 
their focus on speed and efficiency helps shape advanced features that make the 
system useful to a wider audience. Prioritizing these groups ensures the solution 
is both inclusive for those who need accessibility support and valuable for 
everyday users seeking more efficient workflows.
\subsection{User Participation}
Estimated participation during the project will primarily involve end users and the
development team. Accessibility-focused users are expected to participate for about 1
hour per week through remote or in-lab sessions focused on usability testing,
feedback, and formative evaluations. General users and power users will contribute
roughly 1 hour per week by taking part in efficiency and performance testing of new
features. Additionally, the development team will dedicate around 8–10 hours per week
to designing, building, testing, and refining the system based on user feedback and
project milestones.
\subsection{Maintenance Users and Service Technicians}
The primary maintenance users for this project will be the development team. During the
capstone project, they will be responsible for identifying and fixing issues, releasing 
updates and patches, and ensuring that the system continues to function as expected throughout 
development. They will also create and maintain documentation such as installation guides and
user manuals to support future use and potential handoff of the project after completion.

\section{Mandated Constraints}
\subsection{Solution Constraints}
\begin{itemize}
    \item \textbf{SOL-01}: The application must be lightweight and run without requiring any 
    speacial hardware
    \item \textbf{SOL-02}: The application will operate without requiring user accounts or 
    signups.
\end{itemize}

\subsection{Implementation Environment of the Current System}
\begin{itemize}
    \item \textbf{IMECS-01}: The application is designed to run on the personal computers of the 
    users running mainstream operating systems like Windows and MacOS.
    \item \textbf{IMECS-02}: The application uses standard I/O devices like mice, keyboards, mics, 
    and screens to interact with the user.
    \item \textbf{IMECS-03}: The software bundle includes JavaScript support, the MCP (modular 
    control platform) and any other required libraries.
\end{itemize}

\subsection{Partner or Collaborative Applications}
\begin{itemize}
    \item \textbf{PCA-01}: The application may use AI services via secure APIs for natural 
    language processing and logic.
    \item \textbf{PCA-02}: Integration with external accessibility tools like screen readers may 
    be supported to enhance usability for older adults and users with 
    disabilities.
\end{itemize}

\subsection{Off-the-Shelf Software}
The application will use off-the-shelf software components to reduce devlopement
 time:
 \begin{itemize}
    \item \textbf{OTSS-01}: MCP (Modular Control Platform) will be used for accessibility and 
    interface  management.
    \item \textbf{OTSS-02}: JavaScript libraries and frameworks will be used to implement 
    interactive elements and handle input/output operations.
    \item \textbf{OTSS-03}: LLM services will be used to provide natural language responses and 
    reasoning capabilities. The application does not require a specialized 
    model, so no additional training will be necessary. An off-the-shelf LLM is 
    sufficient to deliver accurate and reliable results for the intended 
    functionality.
\end{itemize}

\subsection{Anticipated Workplace Environment}
\begin{itemize}
    \item \textbf{AWE-01}: The application is expected to run on the personal computers of the 
    users at their homes or assited-living facilities.
    \item \textbf{AWE-02}: The application will be used primarily on Windows and MacOS operating 
    systems.
    \item \textbf{AWE-03}: The system is designed for users with limited technical experience, 
    including older adults and people with disabilities, requiring a simple, 
    intuitive interface.
    \item \textbf{AWE-04}: Devices may vary in screen size, input methods, and processing power, 
    so the app must function on standard consumer computers without 
    specialized hardware.
    \item \textbf{AWE-05}: Environmental factors such as background noise or poor acoustics may 
    affect voice interactions. Therefore, the application shall provide 
    alternative input methods, such as text or on-screen controls, to ensure 
    usability when voice input is infeasible.
\end{itemize}


\subsection{Schedule Constraints}
Development will follow the planned project timeline ensuring that deliverables 
are completed before the deadlines.

\begin{table}[!h]
\caption{Project Timeline} \label{ProjScheduling}
\begin{tabularx}{\textwidth}{lXl}
\toprule
\textbf{Deliverable} & \textbf{Deadline} \\
\midrule
Req. Doc. and Hazard Analysis Revision 0 & October 6, 2025 \\
V\&V Plan Revision 0 & October 27, 2025 \\
Design Document Revision -1 & November 10, 2025 \\
Design Document Revision 0 & January 19, 2026 \\
Revision 0 Demonstration &  February 13, 2026\\
V\&V Report and Extras Revision 0 & March 09, 2026 \\
Final Demonstration (Revision 1)  &  March 29, 2026 \\
EXPO Demonstration & TBD \\
Final Documentation (Revision 1) & April 06, 2026 \\
\bottomrule
\end{tabularx}
\end{table}

\subsection{Budget Constraints}
The project budget is limited to \$500 in total expenditure. Of this \$500, 
\$125 will be provided by the CAS department for approved expenses.

\subsection{Enterprise Constraints}
There are no enterprise constraints for this application. The application will 
be allowed to use any service or LLM as required as long as it adheres to the 
budget constraints. This ensures flexibility and future adaptability.

\section{Naming Conventions and Terminology}
\subsection{Glossary of All Terms, Including Acronyms, Used by Stakeholders
involved in the Project}
\begin{description}
  \item[MCP (Model Context Protocol)] A contract with which an AI agent can find 
  and invoke a toolset to accomplish specific tasks.

  \item[MCP Server] A local server that implements the MCP to allow AI agents
  to discover and use the available tools.

  \item[MCP Client] An AI agent that connects to the MCP server to use the 
  provided tools.

  \item[AI Agent] A software (LLM) that uses artificial intelligence to perform 
  tasks.

  \item[LLM (Large Language Model)] A type of AI model that can understand and
  generate human language.

  \item[Token] A unit of text used in LLMs to process and generate language.

  \item[Sandbox] A secure environment in which code can be run and tested 
  without affecting the rest of the system.

  \item[Action] A single execution of a tool with validated parameters.

  \item[Plan] An ordered sequence of actions to fulfill a user intent.

  \item[Command] An instruction given by the user to the AI agent.

  \item[Permission Scope] Level of access that is given to the AI agent for 
  completing actions.

  \item[Confirmation Gate] A approval prompt given to the user by the AI before 
  completing risky actions.

  \item[TTS (Text-to-Speech)] Tool that converts written text into spoken words.

  \item[STT (Speech-to-Text)] Tool that converts spoken words into written text.

  \item[Audit Log] A record of all actions taken by the AI agent.

  \item[Accessibility] It is usable by people with a wide range of 
  abilities/disabilities.

  \item[User Interface (UI)] The visual elements of the software (frontend) in 
  which a user will engage with the system.

  \item[PII (Personally Identifiable Information)] Any data that could identify 
  a specific individual.

  \item[POLP (Principle of Least Privilege)] A security concept in which the 
  AI agent will have minimum levels of access necessary to perform certain tasks.

  \item[OS Automation Tools] Software on one's computer that gives the AI agent 
  the ability to control and engage with their OS.
\end{description}

\section{Relevant Facts And Assumptions}
\subsection{Relevant Facts}
\begin{description}
  \item[1:] The Proxi's main goal is to enable hands-free computer control
  through natural language commands for disabled users. It also provides
  users with a more efficient workflow. Tasks include opening applications,
  reading and writing files, browsing the web, and scheduling events.
  Interaction with Proxi is via voice command, with a textual input option.

  \item[2:] The system will run on desktop operating systems (Windows, macOS)
  and will engage with many common applications (web browsers, email clients,
  and documents/file browsers).

  \item[3:] Since certain actions can be considered high risk, such as deleting
  files or changing system settings, the system will confirm with the user
  before performing such risky actions that can potentially have destructive
  or sensitive outcomes.
\end{description}
\subsection{Business Rules}
Proxi must get user consent before accessing or handling any personal data.
It must minimize the amount of data shared with external sources, sending only
what is strictly needed to fulfill the user's command.
Any command or action that involves high risk (deleting data, changing system
settings, installing software, etc.) must be confirmed by the user before
execution.
All actions performed should be recorded in an audit log, allowing tracking of
what the AI has done so incorrect actions can be traced and reversed if needed.
Tools used by the AI can only run within a user permission scope, meaning they
can only access and perform actions on resources the user has explicitly
allowed.

\subsection{Assumptions}
\begin{description}
  \item[Hardware] The user has a working computer with a microphone and audio
  output, capable of running the software on a modern OS (Windows, macOS).
  \item[Permissions] The user can grant the permissions needed by the AI agent
  (microphone, screen access, automation APIs) and can run the MCP server.
  \item[Internet] The user has a reliable internet connection to use the AI
  agent, as the LLM and online tools (browser, web search) require connectivity.
  \item[Environment and Language] The user is in a quiet environment where the
  microphone can pick up sound clearly, and will speak in a supported language
  (English).
\end{description}

\section{The Scope of the Work}
\subsection{The Current Situation}
In the current computing landscape, many individuals face barriers when interacting 
with traditional computer interfaces such as keyboards, mice, and complex graphical 
user interfaces. People with motor disabilities, visual impairments, elderly users, 
or those lacking technical proficiency often struggle to perform basic computer tasks 
efficiently. Existing digital assistants (e.g., Siri, Alexa, or Google Assistant) 
provide limited task execution and lack deep integration with desktop environments 
or productivity applications. As a result, accessibility remains a significant challenge, 
and there is a growing demand for more adaptive, intelligent, and inclusive systems 
that can understand user intent through natural language and perform meaningful 
actions across software environments.
\subsection{The Context of the Work}
The proposed project aims to develop an AI-powered assistive technology platform 
designed to improve computer accessibility by leveraging natural language processing, 
speech recognition, and intelligent task automation. The system will act as an intermediary 
between the user and the computer, interpreting spoken or textual commands and 
executing them through modular software agents. These agents will be capable of 
reasoning, planning, and performing actions such as sending emails, browsing the 
web, or managing files. The work will combine accessibility design principles with 
AI technologies to create an inclusive experience that empowers users who cannot 
easily interact with conventional input devices.
\subsection{Work Partitioning}
The work will be divided into the following major components:
\begin{enumerate}
    \item \textbf{Input and Recognition Layer:} Responsible for capturing and processing user input through speech recognition or text interfaces.
    \item \textbf{Natural Language Understanding (NLU):} Converts the user’s intent into structured commands that can be acted upon by the agent system.
    \item \textbf{Agent Planning and Execution Layer:} Deploys intelligent agents capable of reasoning, planning, and interacting with tools or APIs to achieve user goals.
    \item \textbf{Feedback and Output Layer:} Returns the results of executed actions to the user through text or synthesized speech, ensuring clarity and accessibility.
    \item \textbf{Accessibility and Usability Design:} Focused on interface simplicity, adaptive behavior, and inclusive design for users with various needs.
    \item \textbf{Evaluation and Testing:} Involves user studies and performance evaluation to ensure reliability, accuracy, and user satisfaction.
\end{enumerate}
\subsection{Specifying a Business Use Case (BUC)}
\textbf{BUC–01: Intelligent Accessibility Assistant}

\noindent
\textbf{Primary Actor:} User (individual with limited mobility or low computer literacy)\\
\textbf{Goal:} To perform computer-based tasks through natural speech or text commands without relying on traditional input devices.

\noindent
\textbf{Preconditions:}
\begin{itemize}
    \item The user has access to a device with a microphone, speaker, and internet connectivity.
    \item The assistive system is installed and running in the background.
\end{itemize}

\noindent
\textbf{Main Flow:}
\begin{enumerate}
    \item The user issues a command such as “Open my email and send a message to my doctor.”
    \item The system processes the speech or text input and identifies the intended task.
    \item The planning agent determines which actions or tools are required to complete the task.
    \item The system executes the corresponding steps (e.g., opens an email client, composes a message, and sends it).
    \item The system provides feedback to the user confirming the task completion.
\end{enumerate}

\noindent
\textbf{Alternative Flows:}
\begin{itemize}
    \item If the system cannot interpret the user’s request, it will ask clarifying questions.
    \item If a required tool or permission is missing, the system will notify the user and suggest corrective actions.
\end{itemize}

\noindent
\textbf{Postconditions:}
\begin{itemize}
    \item The requested action is completed successfully or an appropriate message is returned.
    \item The system logs the interaction for future learning and performance improvement.
\end{itemize}

\section{Business Data Model and Data Dictionary}
\subsection{Business Data Model}
\begin{figure}[htbp]
  \centering
  \includegraphics[width=0.9\textwidth]{Business_Model.png}
  \caption{Business Data Model}
  \label{fig:business-data-model}
\end{figure}
\subsection{Data Dictionary}

\subsubsection*{User}
\noindent\begin{tabularx}{\textwidth}{@{} l l X @{}}
\toprule
\textbf{Attribute} & \textbf{Type} & \textbf{Description} \\
\midrule
userId & UUID & Unique identifier for a user \\
displayName & String & Name shown in confirmations/UI \\
role & Enum & Authorization role (\textit{Standard}, \textit{Admin}) \\
\bottomrule
\end{tabularx}\par\medskip

\subsubsection*{Device}
\noindent\begin{tabularx}{\textwidth}{@{} l l X @{}}
\toprule
\textbf{Attribute} & \textbf{Type} & \textbf{Description} \\
\midrule
deviceId & UUID & Logical device identity \\
osFamily & Enum & OS family (\textit{Windows}/\textit{macOS}) \\
osVersion & String & OS version string \\
\bottomrule
\end{tabularx}\par\medskip

\subsubsection*{Session}
\noindent\begin{tabularx}{\textwidth}{@{} l l X @{}}
\toprule
\textbf{Attribute} & \textbf{Type} & \textbf{Description} \\
\midrule
sessionId & UUID & Logical dialog/control session \\
startedAt / endedAt & Timestamp & Session start/end \\
status & Enum & \textit{Active}, \textit{Ended}, or \textit{Error} \\
\bottomrule
\end{tabularx}\par\medskip

\subsubsection*{Command}
\noindent\begin{tabularx}{\textwidth}{@{} l l X @{}}
\toprule
\textbf{Attribute} & \textbf{Type} & \textbf{Description} \\
\midrule
commandId & UUID & Captured user request \\
modality & Enum & \textit{Voice}, \textit{Text}, or \textit{Mixed} \\
rawText & String & Transcribed/typed command text \\
timestamp & Timestamp & Capture time \\
\bottomrule
\end{tabularx}\par\medskip

\subsubsection*{Intent}
\noindent\begin{tabularx}{\textwidth}{@{} l l X @{}}
\toprule
\textbf{Attribute} & \textbf{Type} & \textbf{Description} \\
\midrule
intentId & UUID & Normalized goal \\
name & Enum/String & Label \\
confidence & Float & Classifier confidence (0.0--1.0) \\
\bottomrule
\end{tabularx}\par\medskip

\subsubsection*{Plan}
\noindent\begin{tabularx}{\textwidth}{@{} l l X @{}}
\toprule
\textbf{Attribute} & \textbf{Type} & \textbf{Description} \\
\midrule
planId & UUID & Ordered set of actions \\
summary & String & One-line description \\
status & Enum & \textit{Proposed}, \textit{Running}, \textit{Succeeded}, \textit{Failed} \\
\bottomrule
\end{tabularx}\par\medskip

\subsubsection*{Action}
\noindent\begin{tabularx}{\textwidth}{@{} l l X @{}}
\toprule
\textbf{Attribute} & \textbf{Type} & \textbf{Description} \\
\midrule
actionId & UUID & Atomic operation \\
toolId & UUID & Tool to invoke \\
parameters & JSON & Tool inputs (schema-validated) \\
result & JSON & Tool outputs/return \\
riskLevel & Enum & \textit{Low}, \textit{Medium}, \textit{High} \\
\bottomrule
\end{tabularx}\par\medskip

\subsubsection*{Tool}
\noindent\begin{tabularx}{\textwidth}{@{} l l X @{}}
\toprule
\textbf{Attribute} & \textbf{Type} & \textbf{Description} \\
\midrule
toolId & UUID & Unique tool capability \\
name & String & Human-friendly tool name \\
description & String & Purpose/constraints \\
isDestructive & Boolean & Writes/installs/deletes \\
\bottomrule
\end{tabularx}\par\medskip

\subsubsection*{Confirmation}
\noindent\begin{tabularx}{\textwidth}{@{} l l X @{}}
\toprule
\textbf{Attribute} & \textbf{Type} & \textbf{Description} \\
\midrule
confirmationId & UUID & Approval record \\
method & Enum & \textit{GUI} or \textit{Voice} \\
approvedAt & Timestamp & Time of approval \\
\bottomrule
\end{tabularx}\par\medskip

\subsubsection*{AuditLog}
\noindent\begin{tabularx}{\textwidth}{@{} l l X @{}} 
\toprule
\textbf{Attribute} & \textbf{Type} & \textbf{Description} \\
\midrule
auditId & UUID & Append-only log entry \\
eventType & Enum & example: \textit{CommandCaptured}, \textit{ActionSucceeded} \\
details & String/JSON & Minimal facts about the event \\
createdAt & Timestamp & When the entry was written \\
\bottomrule
\end{tabularx}\par\medskip


\section{The Scope of the Product}

\subsection{Product Boundary}
Proxi is a desktop helper that lets people run their computer with short voice or
text commands. In this project, Proxi listens, figures out what the user means,
asks for a quick confirmation when needed, and then carries out the task on the
local machine through tools (MCP).

\textbf{In-Scope Features:}
\begin{itemize}
  \item \textbf{Input and understanding:} Listen for a request, interpret the
  intent, and ask a brief follow-up if there are multiple matches. Learn about
  the user's system and common tasks as time goes on to make Proxi more
  efficient.
  \item \textbf{Desktop actions via MCP:} Perform common tasks such as: open an
  app or file, learn and interact with user apps, browse/search, save/move/rename
  files using local tools with least privilege access.
  \item \textbf{Safety and privacy:} Ask before anything destructive, offer a
  simple undo when possible, and keep a short audit trail of what happened.
  \item \textbf{Accessible, plain language:} Show captions for spoken output, use
  readable defaults (contrast/size), and support keyboard or voice for every step.
\end{itemize}

\textbf{Out-of-Scope Features:}
\begin{itemize}
  \item Replacing full third-party apps (complete email client, calendar, or
  browser).
  \item Unattended background automation, remote administration, or multiple users
  controlling the same desktop session.
  \item Offline language understanding/generation, enterprise policy management,
  or default long-term cloud storage of user data.
\end{itemize}

\subsection{Product Use Case Table}
\begin{table}[H]
\centering
\renewcommand{\arraystretch}{1.15}
\begin{tabular}{|p{2.8cm}|p{2.0cm}|p{3.2cm}|p{3.2cm}|p{2.6cm}|}
\hline
\textbf{Use Case} & \textbf{Actors} & \textbf{Description} & \textbf{Preconditions}
& \textbf{Outcome} \\
\hline
PUC-01: Open Application & User & Open a named app. & App installed; automation
allowed. & App launched or focused. \\
\hline
PUC-02: Open and Read File & User & Open a file; preview/read. & File exists;
within scope. & File opened; content visible. \\
\hline
PUC-03: Browse and Search & User & Go to a URL or search. & Browser available;
internet on. & Page or results shown. \\
\hline
PUC-04: Compose Draft & User & Create a short email/message draft. & Default
mail/editor set. & Draft opens prefilled. \\
\hline
PUC-05: Save/Organize File & User & Save As, move, or rename. & Paths allowed;
confirm overwrite. & Operation completed; confirmed. \\
\hline
PUC-06: Interact with User Apps & User & Perform simple actions in an open app
(click, type, select). & Target app is open; within allowed scope. & Action done
or clear reason shown. \\
\hline
\end{tabular}
\caption{Product Use Case Table}
\label{tab:puc-table}
\end{table}

\subsection{Individual Product Use Cases (PUC's)}

\textbf{PUC-01: Open Application}  
The user asks Proxi to open an application by name. If there are several matches,
Proxi lists them and the user picks one. Proxi checks permissions, opens or
focuses the app, and confirms success. If the app isn’t available, Proxi explains
why and suggests the next step.

\textbf{PUC-02: Open and Read File}  
The user asks to open or read a file. Proxi resolves the path or name within the
allowed scope, opens it in the default viewer, and can read a section aloud or
give a short summary on request. If the file can’t be accessed, Proxi says why
and offers alternatives.

\textbf{PUC-03: Browse and Search}  
The user says to go to a certain URL or to search for something, Proxi validates
the URL or forms a search, opens the default browser, and lands on the right
page. If the link looks unsafe, Proxi warns the user and asks for confirmation.

\textbf{PUC-04: Compose Draft}  
The user dictates a short message or email. Proxi extracts recipients and
content, opens a new draft in the default app, and fills in the fields. The user
reviews and sends. If a recipient is unknown, Proxi asks for the address.

\textbf{PUC-05: Save/Organize File}  
The user asks to Save as, Move, or Rename. Proxi checks the target path, handles
name collisions, and asks before any overwrite. After the operation, Proxi
confirms what changed; if it couldn’t proceed, it explains the reason and
options.

\textbf{PUC-06: Interact with User Apps}  
The user asks Proxi to do a simple action inside an app that’s already open (for
example: click a menu item, type a short phrase, press a button, or toggle a
setting). If there’s any ambiguity, Proxi asks a brief follow-up (“Which
button?”) and then performs the action. If the app can’t be controlled, Proxi
explains why and suggests the next step.

\section{Functional Requirements}
\subsection{Functional Requirements}
\begin{table}[H]
\centering
\renewcommand{\arraystretch}{1.3}
\begin{tabular}{|p{2.5cm}|p{8cm}|p{5cm}|}
\hline
\textbf{Req\#} & \textbf{Requirement} & \textbf{Fit Criterion} \\ \hline

FUNC.R.1 & The system shall accept user input through both speech and text 
interfaces. & Successful recognition and processing of at least 95\% of valid 
user speech or text commands during testing. \\ \hline

FUNC.R.2 & The system shall convert speech input into text using an automatic 
speech recognition (ASR) module. & The transcribed text shall match user speech 
with at least 90\% word accuracy in controlled test conditions. \\ \hline

FUNC.R.3 & The system shall interpret user intent from natural language input 
to determine the desired task. & In 90\% of test cases, the interpreted intent 
shall match the human-labeled ground truth task. \\ \hline

FUNC.R.4 & The system shall plan and execute tasks by invoking one or more 
software agents that perform actions on behalf of the user. & The system shall 
successfully complete end-to-end task execution in at least 85\% of test 
scenarios. \\ \hline

FUNC.R.5 & The system shall provide textual or spoken feedback describing 
the status or results of the requested task. & Each executed command shall 
generate a confirmation message or output within 2 seconds of task completion. 
\\ \hline

FUNC.R.6 & The system shall maintain a short-term conversational memory to
 handle contextual follow-up commands. & In at least 90\% of test cases, 
 follow-up commands dependent on prior context shall be correctly interpreted. 
 \\ \hline
\end{tabular}
\caption{Functional Requirements and Fit Criteria}
\end{table}

\begin{table}[H]
\centering
\renewcommand{\arraystretch}{1.3}
\begin{tabular}{|p{2.5cm}|p{8cm}|p{5cm}|}
\hline
\textbf{Req\#} & \textbf{Requirement} & \textbf{Fit Criterion} \\ \hline
FUNC.R.7 & The system shall log all user interactions and task results for 
analysis and traceability purposes. & Each user session shall generate a 
verifiable log entry containing timestamped input and outcome data. \\ \hline

FUNC.R.8 & The system shall operate without requiring the user to interact 
with low-level system interfaces (e.g., OS file paths or command-line tools). 
& In usability tests, 100\% of user operations shall be achievable through 
high-level natural language commands only. \\ \hline
FUNC.R.9 & The system shall ensure that all agent actions requiring system-level 
access request user confirmation before execution. & Each privileged action 
(e.g., deleting files) shall trigger a confirmation request and require explicit 
user approval. \\ \hline
\end{tabular}
\caption{Functional Requirements and Fit Criteria - Continued}
\end{table}

\section{Look and Feel Requirements}

\subsection{Appearance Requirements}

\begin{description}
  \item[APP.1] The interface should look simple and familiar, like a normal 
  desktop tool with a small status bar that shows when the system is 
  listening, thinking, or ready.
  
  \item[APP.2] Main actions such as listen, stop, undo, and confirm should 
  be clearly visible as buttons on the main screen.
  
  \item[APP.3] The screen should not feel crowded; only the most important 
  options should be shown, and advanced options hidden in a simple menu.
  
  \item[APP.4] Text and buttons should be large and easy to read or click, 
  especially for new or elderly users.
  
  \item[APP.5] A light and dark theme toggle should be available so users 
  can choose what is most comfortable for them.
\end{description}

\subsection{Style Requirements}

\begin{description}
  \item[STY.1] Labels and messages should use short, plain language with no 
  technical hard language.
  
  \item[STY.2] When the system speaks, the text should also appear on 
  screen as captions.
  
  \item[STY.3] Icons should stay consistent and have a short text label 
  underneath to avoid confusion.
\end{description}

\section{Usability and Humanity Requirements}

\subsection{Ease of Use Requirements}

\begin{itemize}
  \item \textbf{EOU.R.1} The system shall be easy for new users to start using. 
  Basic tasks like opening an application or reading a file should require only 
  one simple command or button press.

  \item \textbf{EOU.R.2} The system shall have an intuitive interface that does 
  not overwhelm users. Main actions such as listen, stop, undo, and confirm 
  must be clearly visible and easy to access.

  \item \textbf{EOU.R.3} After a short introduction, users should be able to 
  complete most everyday tasks without assistance, and tasks should become 
  faster with practice.
\end{itemize}

\subsection{Personalization and Internationalization Requirements}

\begin{itemize}
  \item \textbf{PER.R.1} The system shall adapt to a user's speech patterns over 
  time to improve recognition accuracy.

  \item \textbf{PER.R.2} The system shall support Canadian English standards 
  for language, date, and time formatting and allow future translation if needed.
\end{itemize}

\subsection{Learning Requirements}

\begin{itemize}
  \item \textbf{LEA.R.1} The system shall have a short learning curve, allowing 
  new users to understand and use basic commands within 30 minutes.

  \item \textbf{LEA.R.2} The system shall provide a list of example commands 
  or a ``what can I say'' option to help users learn available features quickly.
\end{itemize}

\subsection{Understandability and Politeness Requirements}

\begin{itemize}
  \item \textbf{UAP.R.1} The system shall use simple and clear language for all 
  text and messages, avoiding technical jargon.

  \item \textbf{UAP.R.2} The system shall confirm actions that could change or 
  delete important files or settings before executing them.

  \item \textbf{UAP.R.3} Error messages shall be polite, explain the issue in 
  plain language, and suggest a next step to fix it.
\end{itemize}

\subsection{Accessibility Requirements}

\begin{itemize}
  \item \textbf{ACC.R.1} The system shall allow all actions, including setup and 
  exit, to be performed using voice commands without requiring a keyboard 
  or mouse.

  \item \textbf{ACC.R.2} The system shall comply with recognized accessibility 
  standards such as \textbf{WCAG 2.2} and the \textbf{Accessibility for Ontarians 
  with Disabilities Act (AODA)}. All interface elements must provide captions for 
  spoken responses, support text scaling, and ensure sufficient color contrast to 
  assist users with visual impairments.
\end{itemize}


\section{Performance Requirements}

\subsection{Speed and Latency Requirements}

\begin{itemize}
  \item \textbf{SAL.R.1} The system shall begin responding to a voice command 
  within 2 seconds under normal conditions.
  
  \item \textbf{SAL.R.2} Common actions, such as opening an application or 
  navigating files, shall complete within 3 seconds on standard hardware.
\end{itemize}

\subsection{Safety-Critical Requirements}

\begin{itemize}
  \item \textbf{SAF.R.1} Any action that changes files or system settings shall 
  require user confirmation before execution.
  
  \item \textbf{SAF.R.2} The system shall allow users to undo or roll back 
  critical actions in a single step to prevent accidental changes.
\end{itemize}

\subsection{Precision or Accuracy Requirements}

\begin{itemize}
  \item \textbf{POA.R.1} The system shall correctly recognize at least 90\% of 
  supported voice commands in a quiet environment.
  
  \item \textbf{POA.R.2} In a moderately noisy environment, the system shall 
  maintain at least 80\% command recognition accuracy.
\end{itemize}

\subsection{Robustness or Fault-Tolerance Requirements}

\begin{itemize}
  \item \textbf{ROFT.R.1} The system shall log and report input errors without 
  crashing or losing user data.
  
  \item \textbf{ROFT.R.2} The system shall continue running even if it receives 
  an invalid or incomplete command.
\end{itemize}

\subsection{Capacity Requirements}

\begin{itemize}
  \item \textbf{CAP.R.1} The system shall support continuous operation for at 
  least 4 hours without performance degradation.
  
  \item \textbf{CAP.R.2} CPU usage shall remain under 30\% during normal 
  operation on standard hardware.
\end{itemize}

\subsection{Scalability or Extensibility Requirements}

\begin{itemize}
  \item \textbf{SOE.R.1} The system shall allow new features or modules to be 
  added without major changes to existing functionality.
  
  \item \textbf{SOE.R.2} Any new feature or skill shall load and become 
  available for use within 200 milliseconds.
\end{itemize}

\subsection{Longevity Requirements}

\begin{itemize}
  \item \textbf{LON.R.1} The system shall retain all user settings and 
  personalization data after updates.
  
  \item \textbf{LON.R.2} The system shall allow rollback to a previous version 
  within 60 seconds if an update fails.
\end{itemize}


\section{Operational and Environmental Requirements}
\subsection{Expected Physical Environment}
\begin{itemize}
    \item \textbf{EPE.R.1} The system shall operate on personal computing devices
     such as laptops, desktop computers, equipped with standard hardware
      components, including a microphone, speakers, and a stable internet connection.
    \item \textbf{EPE.R.2} The system shall be designed to work optimally for indoor environments
     such as homes, offices where noise levels are moderate(under 65dB).
    \item \textbf{EPE.R.3} The speech recognition capability shall achieve reliable
     accuracy when the user speaks within approximately one meter of the microphone input.
    \item \textbf{EPE.R.4} The system shall be able to operate effectively under
     normal indoor environmental conditions, including temperatures between 15°C
      and 30°C.
\end{itemize}
\subsection{Wider Environment Requirements}
\begin{itemize} 
    \item \textbf{WER.R.1} The system shall integrate safely within digital ecosystems, 
    ensuring that its automation does not interfere with other running applications.  
    \item \textbf{WER.R.2} The system shall be designed to minimize environmental impact 
    by reducing unnecessary computational load and optimizing resource usage.  
\end{itemize}
\subsection{Requirements for Interfacing with Adjacent Systems}
\begin{itemize}
    \item \textbf{IAS.R.1} The system shall provide secure APIs or middleware for integration
     with third-party tools and applications (e.g., email, web browsers, file systems).  
    \item \textbf{IAS.R.2} The system shall support standard communication protocols such
     as HTTPS, MCP and RESTful API calls.  
    \item \textbf{IAS.R.3} The system shall interact with speech recognition engines
     (e.g., Whisper, Google ASR) and text-to-speech systems through well-defined interfaces.  
    \item \textbf{IAS.R.4} The system shall log interactions with external systems for
     traceability and debugging.  
\end{itemize}
\subsection{Productization Requirements}
\begin{itemize}
    \item \textbf{PRD.R.1} The system shall be packaged as a deployable desktop or web-based
     application with minimal setup steps.  
    \item \textbf{PRD.R.2} The product shall include installation documentation and user
     onboarding instructions in the github repo  
    \item \textbf{PRD.R.3} The design shall allow modular upgrades, enabling future
     integration of improved AI models or additional tools.   
\end{itemize}
\subsection{Release Requirements}
\begin{itemize}
    \item \textbf{REL.R.1} The releases will follow the MAJOR.MINOR.PATCH versioning scheme.
    \item \textbf{REL.R.2} The release package shall be version-controlled and archived for reproducibility.  
\end{itemize}

\section{Maintainability and Support Requirements}
\subsection{Maintenance Requirements}
\begin{itemize}
  \item \textbf{1:} The project shall include a short README and setup steps so new
  contributors can build and run it without help.
  \item \textbf{2:} Adding a new MCP tool shall only require a new module and a
  registry entry; existing tools do not need to be changed.
\end{itemize}

\subsection{Supportability Requirements}
\begin{itemize}
  \item \textbf{1:} The release shall include a simple user guide (quick start,
  common commands, basic troubleshooting).
  \item \textbf{2:} The app shall offer a Report Problem button that bundles recent
  logs and basic system info into a ZIP with the user’s consent.
\end{itemize}

\subsection{Adaptability Requirements}
\begin{itemize}
  \item \textbf{1:} The system shall run on current versions of Windows and macOS
  on standard hardware.
  \item \textbf{2:} The system shall keep working if the user switches default apps
  by updating configuration only, no code changes.
\end{itemize}

\section{Security Requirements}
\subsection{Access Requirements}
\begin{itemize}
    \item \textbf{ACS-01} :The application will be available publicly 
    and no user registration or account is required.
    \item \textbf{ACS-02} :Any necessary permissions, such as 
    itemize or screen access, must be explicitly granted by the user.
\end{itemize}


\subsection{Integrity Requirements}
\begin{itemize}
    \item \textbf{INT-01}: The system shall ensure that any stored 
    or transmitted data is protected from unauthorized alteration or corruption.
\end{itemize}



\subsection{Privacy Requirements}
\begin{itemize}
    \item \textbf{PRIV-01}: Any stored user data will be stored 
    acording to PIPEDA.
    \item \textbf{PRIV-01}: If any external APIs are used, only 
    minimal, non identifiable data may be transmitted. These services must 
    adhere to GDPR. The user must also explicitly accept the privacy policies 
    of these third party services.
\end{itemize}

\subsection{Audit Requirements}
There are no audit requirements for the application. The application does not 
store user data, maintain user accounts or perform any actions that require 
tracking or verification. There is nothing that requires audits or security 
reviews. The privacy focused design eliminates the need for audit trails.

\subsection{Immunity Requirements}
\begin{itemize}
    \item \textbf{IMM-01}: The application must be to prevent DDos 
    Attacks, SQL injection, and unauthorized access.
    \item \textbf{IMM-02}: All third party libraries and 
    dependencies must be up to date and scanned for vulnerabilities.
\end{itemize}

\section{Cultural Requirements}
\subsection{Cultural Requirements}
\begin{itemize}
    \item \textbf{CULR-01}: The application shall use
    polite, culurally neutral language, avoid slangs and any references that may
    confuse or offend users.
    \item \textbf{CULR-02}: The application shall support multiple
    languages and will allow users to interact in their preferred language.
\end{itemize}

\section{Compliance Requirements}
\subsection{Legal Requirements}
\begin{itemize}
    \item \textbf{LGL-01}: The application will adhere to PIPEDA 
    for any stored user data.
    \item \textbf{LGL-02}: The application shall be distributed under
    the MIT license; allowing for free use, modification and distribution as
    long as the original license and copyright notice is also included.
\end{itemize}


\subsection{Standards Compliance Requirements}
\begin{itemize}
    \item \textbf{STDCOMP-01}: The application will
    follow WCAG 2 to ensure accessibility for users with disabilities.
    \item \textbf{STDCOMP-02}: The application should 
    follow ISO 25010 and ISO 27001 to ensure delivery of high quality and secure
    software.
\end{itemize}


\section{Open Issues}

\begin{itemize}
  \item \textbf{OI.1} The integration of the MCP (Model Context Protocol) for 
  controlling system-level functions is still in progress, and further testing is 
  needed to ensure reliable and safe execution of commands across different 
  applications.

  \item \textbf{OI.2} The system’s compatibility and consistent performance 
  across different operating systems (such as Windows, macOS, and Linux) 
  have not yet been fully tested or confirmed.

  \item \textbf{OI.3} Ensuring that user commands do not unintentionally trigger 
  harmful or unauthorized actions on the device is still under investigation 
  and requires additional safety checks and permission handling.
\end{itemize}


\section{Off-the-Shelf Solutions}
\subsection{Ready-Made Products}
Several existing tools can cover key parts of Proxi so we do not reinvent them:
\begin{itemize}
  \item \textbf{Speech-to-Text (STT):} Many STT engines can perform microphone
  input and return text with timestamps. Many computers have built-in STT
  software that is free to use.
  \item \textbf{Text-to-Speech (TTS):} OpenAI has its own TTS model that can
  convert text to speech, such as GPT-4o mini TTS.
  \item \textbf{Desktop Automation:} OS-level automation libraries can open
  apps, switch windows, and engage with controls.
  \item \textbf{Language Understanding:} An API to a LLM such as GPT-4o Mini
  can be the head AI agent for understanding user commands and generating plans.
  \item \textbf{Storage and Logging:} A lightweight database (SQLite) or a log
  file can store audit logs and MCP settings with minimal setup.
\end{itemize}

\subsection{Reusable Components}
We can reuse or lightly adapt components that are common across desktop
assistants:
\begin{itemize}
  \item \textbf{MCP tool skeletons:} A small template for tools and a registry
  storer so new tools are added by convention.
  \item \textbf{OS adapters:} Adapters for Windows/macOS that present the same
  functions (open app, focus window, file operations) behind one interface.
  \item \textbf{Prompt templates:} We can reuse templates for short, reusable
  prompts to open apps, open files, browse/search, and save/move/rename.
\end{itemize}

\subsection{Products That Can Be Copied}
We shall adopt proven patterns from familiar tools to speed delivery:
\begin{itemize}
  \item \textbf{Command palette pattern:} A single search box that finds actions
  and recent items, with arrow-key selection and enter to run it.
  \item \textbf{Clear confirmation modal:} A dialog that states the action in
  simple language, showing what will change, and offers approve/cancel options.
  \item \textbf{Pre-run checklist:} A simple setup that asks for microphone
  permission, automation permission, and desktop control on first use. Whenever
  app access is needed, permission will be requested from the user again before
  executing any further commands.
\end{itemize}

\section{New Problems}
\subsection{Effects on the Current Environment}
The introduction of the AI-powered assistive technology platform may alter how 
users interact with their existing computing environment. Potential effects include:
\begin{itemize}
    \item Users may become dependent on natural language commands, reducing direct 
    interaction with traditional interfaces.
    \item Additional system resources (CPU, memory) will be used to run AI agents, 
    which could impact performance of other applications.
    \item There may be changes in user workflows, requiring adaptation to the system’s 
    operational model.
    \item The presence of AI-driven agents could introduce unexpected interactions 
    with other software if not properly sandboxed.
\end{itemize}

\subsection{Effects on the Installed Systems}
Integrating the system with existing hardware and software may create compatibility 
challenges:
\begin{itemize}
    \item Existing applications may require updated permissions or APIs to allow 
    agent interactions.
    \item System updates (OS or applications) may temporarily disrupt AI agent 
    functionality.
    \item The system may necessitate installation of additional software libraries 
    or frameworks, which could conflict with pre-existing software.
    \item Increased network usage for cloud-based AI services could affect other 
    applications relying on the same connectivity.
\end{itemize}

\subsection{Potential User Problems}
While the system is designed to enhance accessibility, users may encounter challenges:
\begin{itemize}
    \item Misinterpretation of voice commands may lead to incorrect task execution.
    \item Users may feel discomfort or distrust if the system performs unexpected 
    actions.
    \item Cognitive overload may occur if the system provides too much feedback 
    or prompts.
    \item Users with atypical speech patterns or accents may experience reduced 
    recognition accuracy.
\end{itemize}

\subsection{Limitations in the Anticipated Implementation Environment That May Inhibit the New Product}
Certain environmental and technical limitations may restrict the effectiveness of 
the platform:
\begin{itemize}
    \item Limited or unstable internet connectivity could hinder cloud-based speech 
    recognition or AI planning.
    \item Hardware constraints (low-end processors, insufficient RAM) may slow down 
    task execution.
    \item Applications with restrictive APIs/security measures may not allow full 
    agent interaction.
\end{itemize}

\subsection{Follow-Up Problems}
After deployment, additional challenges may emerge that require monitoring and mitigation:
\begin{itemize}
    \item Continuous maintenance will be needed to ensure compatibility with 
    updated applications and OS versions.
    \item Security and privacy risks from storing or processing user data must be 
    actively managed.
\end{itemize}

\section{Tasks}
\subsection{Project Planning}
\begin{enumerate}
    \item The project will follow a iterative and incremental development plan. 
    Early prototyping along with continuous feedback from beta users will help 
    address defects early.
    \item Core stages will include requirements, system design, development, 
    testing, and deployment.
    \item Developers will be responsible for coming up with test cases (unit 
    and functional tests), so sufficient time should also be allocated to 
    writing test cases.
\end{enumerate}

\subsection{Planning of the Development Phases}
Since development will be interative and incremental, the phases are not 
strictly linear. Development and testing, in particular, will be repeated 
iteratively as part of an incremental cycle. The Development phases will be 
broadly divided into these phases:
\begin{itemize}
    \item \textbf{Requirements Phase}
    \begin{itemize}
        \item Identify what the system must do and it's constraints.
        \item Come up with Functional, Non-Functional requirements.
        \item Gather input from target users.
    \end{itemize}

    \item \textbf{System Design Phase}
    \begin{itemize}
        \item Create overall system architecture, high level component 
        interactions and integrations.
        \item Select appropriate off-the-shelf LLM model for natural language
        processing and logic
        \item Brainstrom and pick appropriate technology stack.
        \item Create diagrams, mockups and data flow diagrams to aide 
        development.
    \end{itemize}

    \item \textbf{Development Phase}
    \begin{itemize}
        \item Implement system according to design and feedback from previous
        iterations.
        \item Build the user interface using selected technology stack and 
        integrate partner applications and technologies.
        \item Come up with automated and manual test cases for any new features
        added or any change in functionality. These can be added to the 
        automated testing pipelines.
    \end{itemize}

    \item \textbf{Testing Phase}
    \begin{itemize}
        \item Create and conduct comprehensive test suites that include testing
        objectives, failures, successes and stability.
        \item Test the system with representative users, including older adults 
        and users with disabilities, to confirm the interface is intuitive and 
        meets accessibility standards.
        \item Evaluate responsiveness and stability across different hardware 
        setups, including low-spec devices.
        \item Ensure regressions aren't introduced between successive 
        iteractions and that new changes don't break existing features.
        \item Document all testing results and come up with possible 
        improvements for next development iteration.
    \end{itemize}

    \item \textbf{Deployment Phase}
    \begin{itemize}
        \item Ensure that instruction manuals are up to date with latest 
        features.
        \item Ensure installation and initial setup are simple and do not 
        require technical expertise.
        \item Deploy the application using the selected channel. Continuously 
        monitor system performance and user feedback for maintenance and 
        updates.
        \item Maintain version control and update logs so improvements and bug 
        fixes can be tracked systematically
    \end{itemize}
\end{itemize}


\section{Migration to the New Product}
\subsection{Requirements for Migration to the New Product}
There are no special migration steps beyond just installing Proxi and granting
the usual permissions on the first run (microphone, automation, etc.). After
that, for access to any apps and folders, Proxi will request permission again.
Existing workflows and applications will remain the same; users only need to
familiarize themselves with the new voice/text command interface.

\subsection{Data That Has to be Modified or Translated for the New System}
No data conversion is needed. This tool works with the user’s existing files and
applications as they are. The only new data created is potentially a local
configuration file generated. Nothing else needs to be imported or translated
from previous tools.

\section{Costs}
To minimize costs, the project will aim to use open source and off-the-shelf 
solutions whenever possible. Since this project is a student project as part of
a course, development costs will not be taken into consideration. The only costs
associated with development are related to any third-party libraries or any APIs
used by the application. The application itself will be free to use and publicly
available.

\section{User Documentation and Training}
\subsection{User Documentation Requirements}
\begin{itemize}
    \item The system shall include a user manual explaining installation, setup,
     and configuration procedures.  
    \item The documentation shall provide step-by-step instructions for performing
     common tasks via speech and text commands.  
    \item The manual shall include troubleshooting guidance for common issues such
     as misinterpreted commands, failed task execution, or connectivity problems.  
    \item The documentation shall provide an explanation of accessibility features,
     including options for speech rate, font size, and interface contrast.  
    \item The system shall include inline help or tooltips within the interface for
     immediate guidance during use.  
    \item Documentation shall be available in multiple formats, including PDF, 
    HTML, and context-sensitive help within the software.
\end{itemize}

\subsection{Training Requirements}
There isn't a dedicated training program for the system but the user will be 
provided with example scenarios that they can follow to build intuition on how 
to direct the agentic system.

\begin{itemize}
    \item Users shall be able to complete an introductory tutorial that demonstrates
     how to issue commands and interact with the AI agents.  
    \item The system shall provide interactive training sessions or guided exercises
     to familiarize users with core functionalities such as sending emails, managing 
     files, and performing web searches.  
    \item Training materials shall include examples of natural language commands, 
    including variations in phrasing to account for diverse user speech patterns.  

\end{itemize}

\section{Waiting Room}
These are good ideas we’re keeping on the side for now, as they are not the most
urgent.
\begin{itemize}
  \item \textbf{Offline mode:} Basic voice commands without internet using a small
  local model.
  \item \textbf{App plugins:} Simple SDK so other people can add new MCP tools.
  \item \textbf{Interactions with multiple devices:} Start a task on one computer
  and finish on another. For example, computer to computer connection.
  \item \textbf{Custom voice:} Let users pick a voice for Text-to-Speech. A lot of
  TTS engines have many voices to choose from. For now we shall keep it simple
  with one voice.
  \item \textbf{More languages:} Support more languages for voice commands and
  responses.
\end{itemize}

\section{Ideas for Solution}
Proxi should stay simple and safe. The main idea is a small desktop app that
listens for a request, confirms anything risky in plain language, and then runs a
scoped action through a set of MCP tools.
\begin{itemize}
  \item \textbf{Local MCP server:} A set of processes expose a bunch of safe tools
  (open app, open file, browse, save/move/rename, click buttons on screen, etc.).
  Each tool runs with low privilege and only within user approved scopes.
  \item \textbf{Lightweight UI:} A simple AI with options for users to click, type,
  or speak. The UI shows status (listening, thinking, ready), captions for spoken
  output, and a few big buttons (listen, stop, undo, approve).
  \item \textbf{Voice in, text out (with captions):} Use a STT engine for
  microphone input and TTS for spoken output. Always show captions to the text that
  is coming out/in.
  \item \textbf{Simple intent handling:} Map common phrases to a small set of
  intents using prompts/templates (Open app, Open file, browse, save/move/rename).
  If unclear, ask one short follow up.
  \item \textbf{Safety by default:} Permission scopes (folders/apps) are chosen at
  setup and can be edited or approved by the user later. Destructive operations
  always require an explicit approval from the user.
  \item \textbf{Traceability:} Keep a short, local audit log. Any sensitive
  information should be kept out of the audit logs.
\end{itemize}


\newpage{}
\section*{Appendix --- Reflection}

\input{../Reflection.tex}

\begin{enumerate}
  \item What went well while writing this deliverable? 
  \item What pain points did you experience during this deliverable, and how did
  you resolve them?
  \item How many of your requirements were inspired by speaking to your
  client(s) or their proxies (e.g. your peers, stakeholders, potential users)?
  \item Which of the courses you have taken, or are currently taking, will help
  your team to be successful with your capstone project.
  \item What knowledge and skills will the team collectively need to acquire to
  successfully complete this capstone project?  Examples of possible knowledge
  to acquire include domain specific knowledge from the domain of your
  application, or software engineering knowledge, mechatronics knowledge or
  computer science knowledge.  Skills may be related to technology, or writing,
  or presentation, or team management, etc.  You should look to identify at
  least one item for each team member.
  \item For each of the knowledge areas and skills identified in the previous
  question, what are at least two approaches to acquiring the knowledge or
  mastering the skill?  Of the identified approaches, which will each team
  member pursue, and why did they make this choice?
\end{enumerate}

\section{Reflections}

\subsection{Amanbeer Minhas Reflection}

\begin{enumerate}
  \item \textbf{What went well while writing this deliverable?} \\
  One thing that went well while writing this deliverable was that I had a 
  much clearer understanding of what was expected compared to our previous 
  submissions. The Volere template helped a lot because it gave a clear 
  structure and description of what needed to be included in each section, 
  so I was able to focus on writing instead of figuring out the format. 
  As a team, we also communicated better and stayed consistent in how we 
  wrote different sections, which made the final document more cohesive.

  \item \textbf{What pain points did you experience during this deliverable, 
  and how did you resolve them?} \\
  The biggest challenge we faced was dividing up the work evenly across the 
  team. At first, some team members started earlier and took on more tasks, 
  while others joined later, leading to an imbalance in workload. This caused 
  some sections to be rushed near the end. To resolve this, we agreed to plan 
  task assignments ahead of time for future deliverables so that everyone has 
  clear responsibilities and the work is distributed more fairly. This will 
  help us stay more organized and avoid last-minute issues.

  \item \textbf{How many of your requirements were inspired by speaking to 
  your client(s) or their proxies?} \\
  It is hard to put an exact number on how many requirements were inspired 
  by client conversations, but several key ones were shaped by those 
  discussions. Our focus on accessibility and ease of use came directly from 
  client feedback about making the system usable for people with motor or 
  visual impairments. Additionally, requirements around safety, such as 
  confirming system-level actions before execution, were influenced by 
  stakeholder input about avoiding accidental or harmful commands.

  \item \textbf{Which of the courses you have taken, or are currently taking, 
  will help your team to be successful with your capstone project?} \\
  Several courses I have taken will directly support our work on this project. 
  Software Requirements (SFWRENG 3RA3) taught me how to gather and define 
  clear requirements, which is essential for this deliverable. Software 
  Architecture and Design courses will help with structuring the system and 
  designing modular components. Human-Computer Interfaces will be useful 
  for designing an accessible and user-friendly interface. Finally, our 
  programming and systems courses like Object-Oriented Programming and 
  Concurrent Systems Design provide the technical foundation for building 
  and connecting different parts of the system.

  \item \textbf{What knowledge and skills will the team collectively need to 
  acquire to successfully complete this capstone project?} \\
  As a team, we will need to gain deeper knowledge of how to integrate the 
  MCP (Model Context Protocol) effectively to control system-level actions. 
  We also need to improve our understanding of operating system differences 
  to make sure the solution works across Windows, macOS, and Linux. On the 
  skills side, we will need to strengthen our testing practices, improve our 
  technical writing for documentation, and enhance our team coordination and 
  time management skills. Individually, I want to focus on improving my 
  testing and integration skills, as that will be important for the reliability 
  of our system.

  \item \textbf{For each of the knowledge areas and skills identified in the 
  previous question, what are at least two approaches to acquiring the knowledge 
  or mastering the skill? Which will each team member pursue, and why did they 
  make this choice?} \\
  To improve our knowledge of MCP and system-level integration, we can study 
  official documentation and build small prototype projects to experiment with 
  its capabilities. We can also seek help from online forums and developer 
  communities to learn from real-world use cases. For cross-platform knowledge, 
  we can test our software on multiple operating systems throughout development 
  and review OS-specific guidelines. To improve testing skills, we can practice 
  writing unit and integration tests for smaller modules early in the project 
  and use tools like automated testing frameworks. For writing and documentation, 
  we plan to review examples of high-quality documentation and get feedback from 
  instructors and TAs. I will personally focus on building prototypes to learn 
  MCP integration and writing more automated tests, as these are most relevant 
  to the parts of the project I am contributing to.
\end{enumerate}

\subsection{Gourob Podder Reflection}

\begin{enumerate}
  \item \textbf{What went well while writing this deliverable?} \\
  While writing this deliverable, we were able to clearly define the functional 
  and operational requirements of our AI-powered assistive technology platform. 
  Structuring requirements as black-box and verifiable statements ensured clarity 
  and made it easier to anticipate testing criteria. Breaking down the system into 
  modular components such as input handling, agent planning, and feedback mechanisms 
  allowed us to comprehensively cover potential user interactions and system behaviors. 
  Additionally, using a structured LaTeX format helped maintain consistency across sections.

  \item \textbf{What pain points did you experience during this deliverable, and how did you resolve them?} \\
  A key challenge was ensuring that all requirements were verifiable and free of design 
  decisions. Initially, some requirements were too abstract (e.g., “the system shall 
  be user-friendly”), which was not testable. We resolved this by reframing each 
  requirement to include measurable criteria, such as task completion rates, recognition 
  accuracy, or response times. Another pain point was anticipating all potential user 
  problems and environmental constraints without making assumptions. We addressed this 
  by reviewing accessibility guidelines, examining similar assistive technologies, 
  and considering diverse user scenarios, including elderly users and people with disabilities.

  \item \textbf{How many of your requirements were inspired by speaking to your client(s) or their proxies?} \\
  We don't have an exact measure of client impact, but the core requirements were either 
  constructed or revised based on talking to potential clients. Discussions with accessibility-focused 
  community members provided insight into what tasks are most important, potential user errors, 
  and expectations for feedback mechanisms. This input was crucial for shaping functional requirements 
  such as conversational memory and clarification prompt features.

  \item \textbf{Which of the courses you have taken, or are currently taking, will help your team to be successful with your capstone project?} \\
  Several courses will directly support our work on this project: 
  \begin{itemize}
      \item \textbf{Human-Computer Interaction (4HC3):} Understanding usability principles and accessibility guidelines.  
      \item \textbf{Software Engineering Requirements (3RA3):} Writing SRS, functional and non-functional requirements, and system design.  
      \item \textbf{Operating Systems (3SH3):} Managing agent execution, concurrency, and integration with host systems.  
      \item \textbf{Databases and Networking (4C03):} Handling logging, cloud-based APIs, and data persistence for the system.  
  \end{itemize}

  \item \textbf{What knowledge and skills will the team collectively need to acquire to successfully complete this capstone project?} \\
  To successfully complete this project, the team will need:
  \begin{itemize}
      \item \textbf{Natural Language Processing (NLP) knowledge:} Understanding speech-to-text, intent recognition, and dialogue management.  
      \item \textbf{Agent-based system design:} Planning, execution, and tool integration for autonomous agents.  
      \item \textbf{Accessibility and usability skills:} Designing for diverse users, including those with disabilities or low technical proficiency.   
  \end{itemize}

  \item \textbf{For each of the knowledge areas and skills identified in the previous question, what are at least two approaches to acquiring the knowledge or mastering the skill? Which will each team member pursue, and why did they make this choice?} \\
  To acquire these skills:
  \begin{itemize}
      \item \textbf{NLP knowledge:}  
      \begin{enumerate}
          \item Complete online tutorials and courses on speech recognition and intent detection.  
          \item Implement small prototypes of text and speech processing modules for hands-on experience.  
      \end{enumerate}
      \item \textbf{Agent-based system design:}  
      \begin{enumerate}
          \item Study academic papers and open-source projects on multi-agent systems.  
          \item Develop proof-of-concept agents in a controlled environment to understand task planning and execution.  
      \end{enumerate}
      \item \textbf{Accessibility and usability skills:}  
      \begin{enumerate}
          \item Review accessibility guidelines such as WCAG 2.1 and analyze existing assistive software.  
          \item Conduct user testing sessions with peers or volunteers to identify usability issues.  
      \end{enumerate}
  \end{itemize}
  Each team member will focus on specific areas while gaining a breadth-first understanding of other areas. 
  In my case, I have prior knowledge with NLP and basic agentic systems, so I will focus on reviewing accessibility studies 
  related to software design.
\end{enumerate}

\subsection{Ajay Grewal Reflection}

\begin{enumerate}
  \item \textbf{What went well while writing this deliverable?} \\
  We agreed early on what Proxi should do, which is to turn prompts from a user into safe actions. 
  That made the writing easier and overall requirements easier to understand and write down. This 
  resulted in the sections coming together quickly. We also had good team communication, with everyone contributing ideas.
  \item \textbf{What pain points did you experience during this deliverable, and how did you resolve them?} \\
  We kept slipping into how this would be built instead of what Proxi would do. We fixed this by asking ourselves if this 
  can be ideated properly without going too deep into the technology side. We also had many LaTeX issues when writing this 
  regarding formatting. This was resolved through heavy trial and error.
  \item \textbf{How many of your requirements were inspired by speaking to your client(s) or their proxies?} \\
  It is difficult to put an exact number on this, but many important ones were from those discussions. Making Proxi accessible 
  and simple came from obvious client needs. Safety requirements, such as approving system level actions, were also heavily 
  influenced by stakeholders.
  \item \textbf{Which of the courses you have taken, or are currently taking, will help your team to be successful with your 
  capstone project?} \\
  Data structures and algorithms will help us in having efficient code and data processing. Software requirements will help in 
  defining structured requirements. Human computer interactions will help in making an effective and simple to use UI design for 
  Proxi.
  \item \textbf{What knowledge and skills will the team collectively need to acquire to successfully complete this capstone 
  project?} \\
  Better understanding of MCP tools, how to do OS automation, and how to build reliable tests/logs for our software. 
  Personally, I really want to get better at MCP and OS automation as this will define how well Proxi works. We will also need to 
  get better at writing tests and logs to ensure Proxi is reliable and safe. We also need to better work as a group to ensure the 
  code work is separated well and everyone is contributing equally.
  \item \textbf{For each of the knowledge areas and skills identified in the 
  previous question, what are at least two approaches to acquiring the knowledge 
  or mastering the skill? Which will each team member pursue, and why did they 
  make this choice?} \\
  Firstly, for this to work, a lot of trial and error will need to happen. Multiple prototypes will have to be built 
  to see what works best, furthering our understanding of how MCP, and OS automation works, as well as writing effective test 
  cases. I and each team member will also watch and read tutorials on MCP and OS automation to further understanding of these 
  topics. This way we can all equally contribute to the technical design of Proxi. 
\end{enumerate}

\subsection{Savinay Chhabra Reflection}
\begin{enumerate}
  \item \textbf{What went well while writing this deliverable?} \\
  The templates helped a lot as these explained what was expected of us. The 
  weekly team meetings also helped us with coordinate with other team members 
  for inspiration and ideas.

  \item \textbf{What pain points did you experience during this deliverable, 
  and how did you resolve them?} \\
  Dividing the work proved challenging once again. The previous deliverable
  also suffered from an imbalanced workload. The problem is that we don't know
  the time commitment each section requires until we actually do that section. 
  We split out the workload based on what we thougth were our individual 
  strengths however sections are not necessarily even in their workload. Another
  challenge was staying solution neutral while coming up with requirements.
  

  \item \textbf{How many of your requirements were inspired by speaking to 
  your client(s) or their proxies?} \\
  I don't recall excatly how many requirements were inspired by the clients, 
  but the core requirements were made after understanding the needs of older
  adults. Other requirements were inferred from general best practices and what
  we thought would make sense.

  \item \textbf{Which of the courses you have taken, or are currently taking, 
  will help your team to be successful with your capstone project?} \\
  SFWRENG 2AA4 and SFWRENG 3RA3 are two course in particular that will help our
  team to be successful. These courses include requirements and software 
  delivery which are particularly important for this course.
  

  \item \textbf{What knowledge and skills will the team collectively need to 
  acquire to successfully complete this capstone project?} \\
  As a team, we will need to learn more about MCP and how to use it to control
  the user's system. We will also need to learn about LLM integration and 
  voice to text features. We will also need to upskill in the testing domain 
  as none of us have exhaustive knowledge on end-to-end system testing.
 

  \item \textbf{For each of the knowledge areas and skills identified in the 
  previous question, what are at least two approaches to acquiring the knowledge 
  or mastering the skill? Which will each team member pursue, and why did they 
  make this choice?} \\
  For MCP, documentation and smaller projects are good approaches to familiarize
  ourselves with the technology. For LLM and Voice to speech integrations, 
  online courses and exmaple projects on github are good real world examples we 
  can learn from. Projects are always a good learning method as they provide 
  hands on learning.

\end{enumerate}

\end{document}