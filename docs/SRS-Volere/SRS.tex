% THIS DOCUMENT IS FOLLOWS THE VOLERE TEMPLATE BY Suzanne Robertson and James Robertson
% ONLY THE SECTION HEADINGS ARE PROVIDED
%
% Initial draft from https://github.com/Dieblich/volere
%
% Risks are removed because they are covered by the Hazard Analysis
\documentclass[12pt]{article}

\usepackage{booktabs}
\usepackage{tabularx}
\usepackage{hyperref}
\hypersetup{
    bookmarks=true,         % show bookmarks bar?
      colorlinks=true,      % false: boxed links; true: colored links
    linkcolor=red,          % color of internal links (change box color with linkbordercolor)
    citecolor=green,        % color of links to bibliography
    filecolor=magenta,      % color of file links
    urlcolor=cyan           % color of external links
}

\newcommand{\lips}{\textit{Insert your content here.}}

\input{../Comments}
%% Common Parts

\newcommand{\progname}{Software Engineering} % PUT YOUR PROGRAM NAME HERE
\newcommand{\authname}{Team \#23, Project Proxi
\\ Savinay Chhabra
\\ Amanbeer Singh Minhas
\\ Gourob Podder
\\ Ajay Grewal} % AUTHOR NAMES                  

\usepackage{hyperref}
    \hypersetup{colorlinks=true, linkcolor=blue, citecolor=blue, filecolor=blue,
                urlcolor=blue, unicode=false}
    \urlstyle{same}
                                


\begin{document}

\title{Software Requirements Specification for \progname: subtitle describing software} 
\author{\authname}
\date{\today}
	
\maketitle

~\newpage

\pagenumbering{roman}

\tableofcontents

~\newpage

\section*{Revision History}

\begin{tabularx}{\textwidth}{p{3cm}p{2cm}X}
\toprule {\textbf{Date}} & {\textbf{Version}} & {\textbf{Notes}}\\
\midrule
Date 1 & 1.0 & Notes\\
Date 2 & 1.1 & Notes\\
\bottomrule
\end{tabularx}

~\\

~\newpage
\section{Purpose of the Project}
\subsection{User Business}
Proxi is an AI-powered desktop assistant that lets people operate a computer
entirely through natural speech. It targets users who face barriers with traditional
input devices (keyboard, mouse, complex UIs) and organizations that want to provide
inclusive access to essential digital tasks (communication, learning, work). Proxi
augments independence and reduces the digital divide by turning voice into safe,
precise computer actions. While accessibility is the primary driver, Proxi is
equally intended for general users who want faster, lower‑friction workflows so it
benefits both disabled and non‑disabled users.

\subsection{Goals of the Project}
\begin{description}
\item[G-1 (Latency)] Spoken system responses for common commands shall begin within 
$\leq$ 2.0 s from end-of-speech in a quiet environment.

\item[G-2 (Recognition Accuracy)] Command recognition accuracy for supported 
language(s) in quiet environment shall be $\geq$ 90\% intent-level accuracy.

\item[G-3 (Task Coverage)] Users from the primary user group shall complete $\geq$ 
80\% of a predefined core task suite (open app/file, browse, compose, save, 
schedule) using voice or keyboard.

\item[G-4 (Effectiveness)] Compared to baseline (traditional input for same users), 
Proxi shall improve task completion rate by $\geq$ 20\%.

\item[G-5 (Satisfaction)] Accessibility-focused usability tests shall yield 4.0/5.0 satisfaction score.

\item[G-6 (Stretch Goals)] Voice recognition improvements, offline capabilities, 
multimodal interaction support, personalized profiles, enhanced accessibility
\end{description}
\section{Stakeholders}
\subsection{Client}
The clients for this project are the SFWRENG 4G06A Capstone teaching team at 
McMaster University (course Instructor and assigned Teaching Assistants), serving 
as the product owners on behalf of the department. Their mandate is to ensure the 
solution meets accessibility, usability, and engineering quality standards 
appropriate for a capstone deliverable and potential real‑world piloting within 
academic environments. They provide domain expectations (accessibility best 
practices, privacy/compliance constraints in academic settings) and approve scope 
and milestones.
\subsection{Customer}
Our customers are the end-users and organizations that will deploy Proxi to enable 
inclusive and more efficient computer use:
\begin{itemize}
\item \textbf{Educational institutions} (libraries, computer labs, accessibility services) 
  seeking hands-free or low-friction access to standard desktops and web apps
\item \textbf{Healthcare and community organizations} supporting users with motor/vision/
  hearing challenges
\item \textbf{General consumers and power users} who prefer faster, voice-first or mixed-
  modality workflows

\end{itemize}

\subsection{Other Stakeholders}
Other stakeholders include any person or group with interest beyond the client and 
the customer:
\begin{itemize}
\item \textbf{Team Proxi (development team):} responsible for requirements, design, 
  implementation, testing, and deployment artefacts
\item \textbf{Course Staff (Instructor \& TA):} guidance, assessment, feedback, and 
  approvals
\item \textbf{Accessibility Advisors (if engaged):} best practices for WCAG/AT compatibility
\item \textbf{Pilot participants:} individuals who will use the system during user studies 
  and provide feedback
\end{itemize}
\subsection{Hands-On Users of the Project}
Primary hands-on users who will directly interact with Proxi:
\begin{itemize}
\item \textbf{Accessibility-focused users:} People with motor impairments or 
  temporary/situational limitations needing hands-free or simplified control
\item \textbf{General users/power users:} Users seeking faster workflows via 
  voice/text commands with optional keyboard/mouse confirmation
\end{itemize}
\subsection{Personas}
\begin{itemize}
\item \textbf{P1: Amrita (72) - Elderly User:} Amrita is a retired teacher who uses her desktop to 
check emails, pay bills, and video call her family. She struggles with small buttons, 
complicated menus, and remembering multi-step actions, which makes her anxious about 
using technology. She needs a way to perform common tasks more easily and with clear 
guidance to feel confident online.

\item \textbf{P2: Leo (21) - User with Motor Disability:} Leo is a computer science student who finds
it difficult to use a keyboard or mouse due to a motor impairment. He needs to read PDFs,
take notes, and switch between different apps for his coursework. He requires a way to 
interact with his computer hands-free and complete his work without relying on physical 
input.

\item \textbf{P3: Ari (28) - Power User:} Ari works with spreadsheets, emails, and web tools 
throughout the day and often repeats the same steps over and over. Switching between 
programs slows him down, and remembering different commands and shortcuts is frustrating. 
He needs a more efficient way to complete multi-step tasks and manage his work without 
constant interruptions.
\end{itemize}
\subsection{Priorities Assigned to Users}
The \textbf{highest priority} users for this project are \textbf{accessibility-focused} users, 
including elderly users like Amrita and users with motor impairments like Leo, as 
the main goal is to improve computer access and usability for people who face 
physical challenges or find current systems difficult to use. Their needs guide 
the core features and design decisions of the project. \textbf{Power users} like Ari are 
the \textbf{secondary priority} because, while they do not face accessibility barriers, 
their focus on speed and efficiency helps shape advanced features that make the 
system useful to a wider audience. Prioritizing these groups ensures the solution 
is both inclusive for those who need accessibility support and valuable for 
everyday users seeking more efficient workflows.
\subsection{User Participation}
Estimated participation during the project will primarily involve end users and the
development team. Accessibility-focused users are expected to participate for about 1
hour per week through remote or in-lab sessions focused on usability testing,
feedback, and formative evaluations. General users and power users will contribute
roughly 1 hour per week by taking part in efficiency and performance testing of new
features. Additionally, the development team will dedicate around 8–10 hours per week
to designing, building, testing, and refining the system based on user feedback and
project milestones.
\subsection{Maintenance Users and Service Technicians}
The primary maintenance users for this project will be the development team. During the
capstone project, they will be responsible for identifying and fixing issues, releasing 
updates and patches, and ensuring that the system continues to function as expected throughout 
development. They will also create and maintain documentation such as installation guides and
user manuals to support future use and potential handoff of the project after completion.

\section{Mandated Constraints}
\subsection{Solution Constraints}
\lips
\subsection{Implementation Environment of the Current System}
\lips
\subsection{Partner or Collaborative Applications}
\lips
\subsection{Off-the-Shelf Software}
\lips
\subsection{Anticipated Workplace Environment}
\lips
\subsection{Schedule Constraints}
\lips
\subsection{Budget Constraints}
\lips
\subsection{Enterprise Constraints}
\lips

\section{Naming Conventions and Terminology}
\subsection{Glossary of All Terms, Including Acronyms, Used by Stakeholders
involved in the Project}
\begin{description}
  \item[MCP (Model Context Protocol)] A contract with which an AI agent can find 
  and invoke a toolset to accomplish specific tasks.

  \item[MCP Server] A local server that implements the MCP to allow AI agents
  to discover and use the available tools.

  \item[MCP Client] An AI agent that connects to the MCP server to use the provided tools.

  \item[AI Agent] A software (LLM) that uses artificial intelligence to perform tasks.

  \item[LLM (Large Language Model)] A type of AI model that can understand and
  generate human language.

  \item[Token] A unit of text used in LLMs to process and generate language.

  \item[Sandbox] A secure environment in which code can be run and tested without
  affecting the rest of the system.

  \item[Action] A single execution of a tool with validated parameters.

  \item[Plan] An ordered sequence of actions to fulfill a user intent.

  \item[Command] An instruction given by the user to the AI agent.

  \item[Permission Scope] Level of access that is given to the AI agent for completing actions.

  \item[Confirmation Gate] A approval prompt given to the user by the AI before cpmpleting
  risky actions.

  \item[TTS (Text-to-Speech)] Tool that converts written text into spoken words.

  \item[STT (Speech-to-Text)] Tool that converts spoken words into written text.

  \item[Audit Log] A record of all actions taken by the AI agent.

  \item[Accessibility] It is usable by people with a wide range of abilities/disabilities.

  \item[User Interface (UI)] The visual elements of the software (frontend) in which a user will engage with the system.

  \item[PII (Personally Identifiable Information)] Any data that could identify a specific individual.

  \item[POLP (Principle of Least Privilege)] A security concept in which the AI agent will have minimum levels of access 
  necessary to perform certain tasks.

  \item[OS Automation Tools] Software on one's computer that gives the AI agent the ability to control and engage with their OS.
\end{description}

\section{Relevant Facts And Assumptions}
\subsection{Relevant Facts}
\lips
\subsection{Business Rules}
\lips
\subsection{Assumptions}
\lips

\section{The Scope of the Work}
\subsection{The Current Situation}
\lips
\subsection{The Context of the Work}
\lips
\subsection{Work Partitioning}
\lips
\subsection{Specifying a Business Use Case (BUC)}
\lips

\section{Business Data Model and Data Dictionary}
\subsection{Business Data Model}
\lips
\subsection{Data Dictionary}
\lips

\section{The Scope of the Product}
\subsection{Product Boundary}
\lips
\subsection{Product Use Case Table}
\lips
\subsection{Individual Product Use Cases (PUC's)}
\lips

\section{Functional Requirements}
\subsection{Functional Requirements}
\lips

\section{Look and Feel Requirements}

\subsection{Appearance Requirements}

\begin{description}
  \item[APP.1] The interface should look simple and familiar, like a normal 
  desktop tool with a small status bar that shows when the system is 
  listening, thinking, or ready.
  
  \item[APP.2] Main actions such as listen, stop, undo, and confirm should 
  be clearly visible as buttons on the main screen.
  
  \item[APP.3] The screen should not feel crowded; only the most important 
  options should be shown, and advanced options hidden in a simple menu.
  
  \item[APP.4] Text and buttons should be large and easy to read or click, 
  especially for new or elderly users.
  
  \item[APP.5] A light and dark theme toggle should be available so users 
  can choose what is most comfortable for them.
\end{description}

\subsection{Style Requirements}

\begin{description}
  \item[STY.1] Labels and messages should use short, plain language with no 
  technical hard language.
  
  \item[STY.2] When the system speaks, the text should also appear on 
  screen as captions.
  
  \item[STY.3] Icons should stay consistent and have a short text label 
  underneath to avoid confusion.
\end{description}

\section{Usability and Humanity Requirements}

\subsection{Ease of Use Requirements}

\begin{itemize}
  \item \textbf{EOU.R.1} The system shall be easy for new users to start using. 
  Basic tasks like opening an application or reading a file should require only 
  one simple command or button press.

  \item \textbf{EOU.R.2} The system shall have an intuitive interface that does 
  not overwhelm users. Main actions such as listen, stop, undo, and confirm 
  must be clearly visible and easy to access.

  \item \textbf{EOU.R.3} After a short introduction, users should be able to 
  complete most everyday tasks without assistance, and tasks should become 
  faster with practice.
\end{itemize}

\subsection{Personalization and Internationalization Requirements}

\begin{itemize}
  \item \textbf{PER.R.1} The system shall adapt to a user's speech patterns over 
  time to improve recognition accuracy.

  \item \textbf{PER.R.2} The system shall support Canadian English standards 
  for language, date, and time formatting and allow future translation if needed.
\end{itemize}

\subsection{Learning Requirements}

\begin{itemize}
  \item \textbf{LEA.R.1} The system shall have a short learning curve, allowing 
  new users to understand and use basic commands within 30 minutes.

  \item \textbf{LEA.R.2} The system shall provide a list of example commands 
  or a ``what can I say'' option to help users learn available features quickly.
\end{itemize}

\subsection{Understandability and Politeness Requirements}

\begin{itemize}
  \item \textbf{UAP.R.1} The system shall use simple and clear language for all 
  text and messages, avoiding technical jargon.

  \item \textbf{UAP.R.2} The system shall confirm actions that could change or 
  delete important files or settings before executing them.

  \item \textbf{UAP.R.3} Error messages shall be polite, explain the issue in 
  plain language, and suggest a next step to fix it.
\end{itemize}

\subsection{Accessibility Requirements}

\begin{itemize}
  \item \textbf{ACC.R.1} The system shall allow all actions, including setup and 
  exit, to be performed using voice commands without requiring a keyboard 
  or mouse.

  \item \textbf{ACC.R.2} The system shall comply with recognized accessibility 
  standards such as \textbf{WCAG 2.2} and the \textbf{Accessibility for Ontarians 
  with Disabilities Act (AODA)}. All interface elements must provide captions for 
  spoken responses, support text scaling, and ensure sufficient color contrast to 
  assist users with visual impairments.
\end{itemize}


\section{Performance Requirements}

\subsection{Speed and Latency Requirements}

\begin{itemize}
  \item \textbf{SAL.R.1} The system shall begin responding to a voice command 
  within 2 seconds under normal conditions.
  
  \item \textbf{SAL.R.2} Common actions, such as opening an application or 
  navigating files, shall complete within 3 seconds on standard hardware.
\end{itemize}

\subsection{Safety-Critical Requirements}

\begin{itemize}
  \item \textbf{SAF.R.1} Any action that changes files or system settings shall 
  require user confirmation before execution.
  
  \item \textbf{SAF.R.2} The system shall allow users to undo or roll back 
  critical actions in a single step to prevent accidental changes.
\end{itemize}

\subsection{Precision or Accuracy Requirements}

\begin{itemize}
  \item \textbf{POA.R.1} The system shall correctly recognize at least 90\% of 
  supported voice commands in a quiet environment.
  
  \item \textbf{POA.R.2} In a moderately noisy environment, the system shall 
  maintain at least 80\% command recognition accuracy.
\end{itemize}

\subsection{Robustness or Fault-Tolerance Requirements}

\begin{itemize}
  \item \textbf{ROFT.R.1} The system shall log and report input errors without 
  crashing or losing user data.
  
  \item \textbf{ROFT.R.2} The system shall continue running even if it receives 
  an invalid or incomplete command.
\end{itemize}

\subsection{Capacity Requirements}

\begin{itemize}
  \item \textbf{CAP.R.1} The system shall support continuous operation for at 
  least 4 hours without performance degradation.
  
  \item \textbf{CAP.R.2} CPU usage shall remain under 30\% during normal 
  operation on standard hardware.
\end{itemize}

\subsection{Scalability or Extensibility Requirements}

\begin{itemize}
  \item \textbf{SOE.R.1} The system shall allow new features or modules to be 
  added without major changes to existing functionality.
  
  \item \textbf{SOE.R.2} Any new feature or skill shall load and become 
  available for use within 200 milliseconds.
\end{itemize}

\subsection{Longevity Requirements}

\begin{itemize}
  \item \textbf{LON.R.1} The system shall retain all user settings and 
  personalization data after updates.
  
  \item \textbf{LON.R.2} The system shall allow rollback to a previous version 
  within 60 seconds if an update fails.
\end{itemize}


\section{Operational and Environmental Requirements}
\subsection{Expected Physical Environment}
\lips
\subsection{Wider Environment Requirements}
\lips
\subsection{Requirements for Interfacing with Adjacent Systems}
\lips
\subsection{Productization Requirements}
\lips
\subsection{Release Requirements}
\lips

\section{Maintainability and Support Requirements}
\subsection{Maintenance Requirements}
\lips
\subsection{Supportability Requirements}
\lips
\subsection{Adaptability Requirements}
\lips

\section{Security Requirements}
\subsection{Access Requirements}
\lips
\subsection{Integrity Requirements}
\lips
\subsection{Privacy Requirements}
\lips
\subsection{Audit Requirements}
\lips
\subsection{Immunity Requirements}
\lips

\section{Cultural Requirements}
\subsection{Cultural Requirements}
\lips

\section{Compliance Requirements}
\subsection{Legal Requirements}
\lips
\subsection{Standards Compliance Requirements}
\lips

\section{Open Issues}

\begin{itemize}
  \item \textbf{OI.1} The integration of the MCP (Model Context Protocol) for 
  controlling system-level functions is still in progress, and further testing is 
  needed to ensure reliable and safe execution of commands across different 
  applications.

  \item \textbf{OI.2} The system’s compatibility and consistent performance 
  across different operating systems (such as Windows, macOS, and Linux) 
  have not yet been fully tested or confirmed.

  \item \textbf{OI.3} Ensuring that user commands do not unintentionally trigger 
  harmful or unauthorized actions on the device is still under investigation 
  and requires additional safety checks and permission handling.
\end{itemize}


\section{Off-the-Shelf Solutions}
\subsection{Ready-Made Products}
\lips
\subsection{Reusable Components}
\lips
\subsection{Products That Can Be Copied}
\lips

\section{New Problems}
\subsection{Effects on the Current Environment}
\lips
\subsection{Effects on the Installed Systems}
\lips
\subsection{Potential User Problems}
\lips
\subsection{Limitations in the Anticipated Implementation Environment That May
Inhibit the New Product}
\lips
\subsection{Follow-Up Problems}
\lips

\section{Tasks}
\subsection{Project Planning}
\lips
\subsection{Planning of the Development Phases}
\lips

\section{Migration to the New Product}
\subsection{Requirements for Migration to the New Product}
\lips
\subsection{Data That Has to be Modified or Translated for the New System}
\lips

\section{Costs}
\lips
\section{User Documentation and Training}
\subsection{User Documentation Requirements}
\lips
\subsection{Training Requirements}
\lips

\section{Waiting Room}
\lips

\section{Ideas for Solution}
\lips

\newpage{}
\section*{Appendix --- Reflection}

\input{../Reflection.tex}

\begin{enumerate}
  \item What went well while writing this deliverable? 
  \item What pain points did you experience during this deliverable, and how did
  you resolve them?
  \item How many of your requirements were inspired by speaking to your
  client(s) or their proxies (e.g. your peers, stakeholders, potential users)?
  \item Which of the courses you have taken, or are currently taking, will help
  your team to be successful with your capstone project.
  \item What knowledge and skills will the team collectively need to acquire to
  successfully complete this capstone project?  Examples of possible knowledge
  to acquire include domain specific knowledge from the domain of your
  application, or software engineering knowledge, mechatronics knowledge or
  computer science knowledge.  Skills may be related to technology, or writing,
  or presentation, or team management, etc.  You should look to identify at
  least one item for each team member.
  \item For each of the knowledge areas and skills identified in the previous
  question, what are at least two approaches to acquiring the knowledge or
  mastering the skill?  Of the identified approaches, which will each team
  member pursue, and why did they make this choice?
\end{enumerate}

\section{Reflections}

\subsection{Amanbeer Minhas Reflection}

\begin{enumerate}
  \item \textbf{What went well while writing this deliverable?} \\
  One thing that went well while writing this deliverable was that I had a 
  much clearer understanding of what was expected compared to our previous 
  submissions. The Volere template helped a lot because it gave a clear 
  structure and description of what needed to be included in each section, 
  so I was able to focus on writing instead of figuring out the format. 
  As a team, we also communicated better and stayed consistent in how we 
  wrote different sections, which made the final document more cohesive.

  \item \textbf{What pain points did you experience during this deliverable, 
  and how did you resolve them?} \\
  The biggest challenge we faced was dividing up the work evenly across the 
  team. At first, some team members started earlier and took on more tasks, 
  while others joined later, leading to an imbalance in workload. This caused 
  some sections to be rushed near the end. To resolve this, we agreed to plan 
  task assignments ahead of time for future deliverables so that everyone has 
  clear responsibilities and the work is distributed more fairly. This will 
  help us stay more organized and avoid last-minute issues.

  \item \textbf{How many of your requirements were inspired by speaking to 
  your client(s) or their proxies?} \\
  It is hard to put an exact number on how many requirements were inspired 
  by client conversations, but several key ones were shaped by those 
  discussions. Our focus on accessibility and ease of use came directly from 
  client feedback about making the system usable for people with motor or 
  visual impairments. Additionally, requirements around safety, such as 
  confirming system-level actions before execution, were influenced by 
  stakeholder input about avoiding accidental or harmful commands.

  \item \textbf{Which of the courses you have taken, or are currently taking, 
  will help your team to be successful with your capstone project?} \\
  Several courses I have taken will directly support our work on this project. 
  Software Requirements (SFWRENG 3RA3) taught me how to gather and define 
  clear requirements, which is essential for this deliverable. Software 
  Architecture and Design courses will help with structuring the system and 
  designing modular components. Human-Computer Interfaces will be useful 
  for designing an accessible and user-friendly interface. Finally, our 
  programming and systems courses like Object-Oriented Programming and 
  Concurrent Systems Design provide the technical foundation for building 
  and connecting different parts of the system.

  \item \textbf{What knowledge and skills will the team collectively need to 
  acquire to successfully complete this capstone project?} \\
  As a team, we will need to gain deeper knowledge of how to integrate the 
  MCP (Model Context Protocol) effectively to control system-level actions. 
  We also need to improve our understanding of operating system differences 
  to make sure the solution works across Windows, macOS, and Linux. On the 
  skills side, we will need to strengthen our testing practices, improve our 
  technical writing for documentation, and enhance our team coordination and 
  time management skills. Individually, I want to focus on improving my 
  testing and integration skills, as that will be important for the reliability 
  of our system.

  \item \textbf{For each of the knowledge areas and skills identified in the 
  previous question, what are at least two approaches to acquiring the knowledge 
  or mastering the skill? Which will each team member pursue, and why did they 
  make this choice?} \\
  To improve our knowledge of MCP and system-level integration, we can study 
  official documentation and build small prototype projects to experiment with 
  its capabilities. We can also seek help from online forums and developer 
  communities to learn from real-world use cases. For cross-platform knowledge, 
  we can test our software on multiple operating systems throughout development 
  and review OS-specific guidelines. To improve testing skills, we can practice 
  writing unit and integration tests for smaller modules early in the project 
  and use tools like automated testing frameworks. For writing and documentation, 
  we plan to review examples of high-quality documentation and get feedback from 
  instructors and TAs. I will personally focus on building prototypes to learn 
  MCP integration and writing more automated tests, as these are most relevant 
  to the parts of the project I am contributing to.
\end{enumerate}

\subsection{Gourob Podder Reflection}

\begin{enumerate}
  \item \textbf{What went well while writing this deliverable?} \\
  While writing this deliverable, we were able to clearly define the functional 
  and operational requirements of our AI-powered assistive technology platform. 
  Structuring requirements as black-box and verifiable statements ensured clarity 
  and made it easier to anticipate testing criteria. Breaking down the system into 
  modular components such as input handling, agent planning, and feedback mechanisms 
  allowed us to comprehensively cover potential user interactions and system behaviors. 
  Additionally, using a structured LaTeX format helped maintain consistency across sections.

  \item \textbf{What pain points did you experience during this deliverable, and how did you resolve them?} \\
  A key challenge was ensuring that all requirements were verifiable and free of design 
  decisions. Initially, some requirements were too abstract (e.g., “the system shall 
  be user-friendly”), which was not testable. We resolved this by reframing each 
  requirement to include measurable criteria, such as task completion rates, recognition 
  accuracy, or response times. Another pain point was anticipating all potential user 
  problems and environmental constraints without making assumptions. We addressed this 
  by reviewing accessibility guidelines, examining similar assistive technologies, 
  and considering diverse user scenarios, including elderly users and people with disabilities.

  \item \textbf{How many of your requirements were inspired by speaking to your client(s) or their proxies?} \\
  We don't have an exact measure of client impact, but the core requirements were either 
  constructed or revised based on talking to potential clients. Discussions with accessibility-focused 
  community members provided insight into what tasks are most important, potential user errors, 
  and expectations for feedback mechanisms. This input was crucial for shaping functional requirements 
  such as conversational memory and clarification prompt features.

  \item \textbf{Which of the courses you have taken, or are currently taking, will help your team to be successful with your capstone project?} \\
  Several courses will directly support our work on this project: 
  \begin{itemize}
      \item \textbf{Human-Computer Interaction (4HC3):} Understanding usability principles and accessibility guidelines.  
      \item \textbf{Software Engineering Requirements (3RA3):} Writing SRS, functional and non-functional requirements, and system design.  
      \item \textbf{Operating Systems (3SH3):} Managing agent execution, concurrency, and integration with host systems.  
      \item \textbf{Databases and Networking (4C03):} Handling logging, cloud-based APIs, and data persistence for the system.  
  \end{itemize}

  \item \textbf{What knowledge and skills will the team collectively need to acquire to successfully complete this capstone project?} \\
  To successfully complete this project, the team will need:
  \begin{itemize}
      \item \textbf{Natural Language Processing (NLP) knowledge:} Understanding speech-to-text, intent recognition, and dialogue management.  
      \item \textbf{Agent-based system design:} Planning, execution, and tool integration for autonomous agents.  
      \item \textbf{Accessibility and usability skills:} Designing for diverse users, including those with disabilities or low technical proficiency.   
  \end{itemize}

  \item \textbf{For each of the knowledge areas and skills identified in the previous question, what are at least two approaches to acquiring the knowledge or mastering the skill? Which will each team member pursue, and why did they make this choice?} \\
  To acquire these skills:
  \begin{itemize}
      \item \textbf{NLP knowledge:}  
      \begin{enumerate}
          \item Complete online tutorials and courses on speech recognition and intent detection.  
          \item Implement small prototypes of text and speech processing modules for hands-on experience.  
      \end{enumerate}
      \item \textbf{Agent-based system design:}  
      \begin{enumerate}
          \item Study academic papers and open-source projects on multi-agent systems.  
          \item Develop proof-of-concept agents in a controlled environment to understand task planning and execution.  
      \end{enumerate}
      \item \textbf{Accessibility and usability skills:}  
      \begin{enumerate}
          \item Review accessibility guidelines such as WCAG 2.1 and analyze existing assistive software.  
          \item Conduct user testing sessions with peers or volunteers to identify usability issues.  
      \end{enumerate}
  \end{itemize}
  Each team member will focus on specific areas while gaining a breadth-first understanding of other areas. 
  In my case, I have prior knowledge with NLP and basic agentic systems, so I will focus on reviewing accessibility studies 
  related to software design.
\end{enumerate}

\subsection{Ajay Grewal Reflection}

\begin{enumerate}
  \item \textbf{What went well while writing this deliverable?} \\
  We agreed early on what Proxi should do, which is to turn prompts from a user into safe actions. 
  That made the writing easier and overall requirements easier to understand and write down. This 
  resulted in the sections coming together quickly. We also had good team communication, with everyone contributing ideas.
  \item \textbf{What pain points did you experience during this deliverable, and how did you resolve them?} \\
  We kept slipping into how this would be built instead of what Proxi would do. We fixed this by asking ourselves if this 
  can be ideated properly without going too deep into the technology side. We also had many LaTeX issues when writing this 
  regarding formatting. This was resolved through heavy trial and error.
  \item \textbf{How many of your requirements were inspired by speaking to your client(s) or their proxies?} \\
  It is difficult to put an exact number on this, but many important ones were from those discussions. Making Proxi accessible 
  and simple came from obvious client needs. Safety requirements, such as approving system level actions, were also heavily 
  influenced by stakeholders.
  \item \textbf{Which of the courses you have taken, or are currently taking, will help your team to be successful with your 
  capstone project?} \\
  Data structures and algorithms will help us in having efficient code and data processing. Software requirements will help in 
  defining structured requirements. Human computer interactions will help in making an effective and simple to use UI design for 
  Proxi.
  \item \textbf{What knowledge and skills will the team collectively need to acquire to successfully complete this capstone 
  project?} \\
  Better understanding of MCP tools, how to do OS automation, and how to build reliable tests/logs for our software. 
  Personally, I really want to get better at MCP and OS automation as this will define how well Proxi works. We will also need to 
  get better at writing tests and logs to ensure Proxi is reliable and safe. We also need to better work as a group to ensure the 
  code work is separated well and everyone is contributing equally.
  \item \textbf{For each of the knowledge areas and skills identified in the 
  previous question, what are at least two approaches to acquiring the knowledge 
  or mastering the skill? Which will each team member pursue, and why did they 
  make this choice?} \\
  Firstly, for this to work, a lot of trial and error will need to happen. Multiple prototypes will have to be built 
  to see what works best, furthering our understanding of how MCP, and OS automation works, as well as writing effective test 
  cases. I and each team member will also watch and read tutorials on MCP and OS automation to further understanding of these 
  topics. This way we can all equally contribute to the technical design of Proxi. 
\end{enumerate}

\subsection{Savinay Chhabra Reflection}
\begin{enumerate}
  \item \textbf{What went well while writing this deliverable?} \\
  The templates helped a lot as these explained what was expected of us. The 
  weekly team meetings also helped us with coordinate with other team members 
  for inspiration and ideas.

  \item \textbf{What pain points did you experience during this deliverable, 
  and how did you resolve them?} \\
  Dividing the work proved challenging once again. The previous deliverable
  also suffered from an imbalanced workload. The problem is that we don't know
  the time commitment each section requires until we actually do that section. 
  We split out the workload based on what we thougth were our individual 
  strengths however sections are not necessarily even in their workload. Another
  challenge was staying solution neutral while coming up with requirements.
  

  \item \textbf{How many of your requirements were inspired by speaking to 
  your client(s) or their proxies?} \\
  I don't recall excatly how many requirements were inspired by the clients, 
  but the core requirements were made after understanding the needs of older
  adults. Other requirements were inferred from general best practices and what
  we thought would make sense.

  \item \textbf{Which of the courses you have taken, or are currently taking, 
  will help your team to be successful with your capstone project?} \\
  SFWRENG 2AA4 and SFWRENG 3RA3 are two course in particular that will help our
  team to be successful. These courses include requirements and software 
  delivery which are particularly important for this course.
  

  \item \textbf{What knowledge and skills will the team collectively need to 
  acquire to successfully complete this capstone project?} \\
  As a team, we will need to learn more about MCP and how to use it to control
  the user's system. We will also need to learn about LLM integration and 
  voice to text features. We will also need to upskill in the testing domain 
  as none of us have exhaustive knowledge on end-to-end system testing.
 

  \item \textbf{For each of the knowledge areas and skills identified in the 
  previous question, what are at least two approaches to acquiring the knowledge 
  or mastering the skill? Which will each team member pursue, and why did they 
  make this choice?} \\
  For MCP, documentation and smaller projects are good approaches to familiarize
  ourselves with the technology. For LLM and Voice to speech integrations, 
  online courses and exmaple projects on github are good real world examples we 
  can learn from. Projects are always a good learning method as they provide 
  hands on learning.

\end{enumerate}

\end{document}