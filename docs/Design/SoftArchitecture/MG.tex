\documentclass[12pt, titlepage]{article}

\usepackage{fullpage}
\usepackage[round]{natbib}
\usepackage{multirow}
\usepackage{booktabs}
\usepackage{tabularx}
\usepackage{graphicx}
\usepackage{float}
\usepackage{hyperref}
\hypersetup{
    colorlinks,
    citecolor=blue,
    filecolor=black,
    linkcolor=red,
    urlcolor=blue
}

\input{../../Comments}
%% Common Parts

\newcommand{\progname}{Software Engineering} % PUT YOUR PROGRAM NAME HERE
\newcommand{\authname}{Team \#23, Project Proxi
\\ Savinay Chhabra
\\ Amanbeer Singh Minhas
\\ Gourob Podder
\\ Ajay Grewal} % AUTHOR NAMES                  

\usepackage{hyperref}
    \hypersetup{colorlinks=true, linkcolor=blue, citecolor=blue, filecolor=blue,
                urlcolor=blue, unicode=false}
    \urlstyle{same}
                                


\newcounter{acnum}
\newcommand{\actheacnum}{AC\theacnum}
\newcommand{\acref}[1]{AC\ref{#1}}

\newcounter{ucnum}
\newcommand{\uctheucnum}{UC\theucnum}
\newcommand{\uref}[1]{UC\ref{#1}}

\newcounter{mnum}
\newcommand{\mthemnum}{M\themnum}
\newcommand{\mref}[1]{M\ref{#1}}

\begin{document}

\title{Module Guide for \progname{}} 
\author{\authname}
\date{\today}

\maketitle

\pagenumbering{roman}

\section{Revision History}

\begin{tabularx}{\textwidth}{p{3cm}p{2cm}X}
\toprule {\bf Date} & {\bf Version} & {\bf Notes}\\
\midrule
Date 1 & 1.0 & Notes\\
Date 2 & 1.1 & Notes\\
\bottomrule
\end{tabularx}

\newpage

\section{Reference Material}

This section records information for easy reference.

\subsection{Abbreviations and Acronyms}

\renewcommand{\arraystretch}{1.2}
\begin{tabular}{l l} 
  \toprule		
  \textbf{symbol} & \textbf{description}\\
  \midrule 
  AC & Anticipated Change\\
  DAG & Directed Acyclic Graph \\
  M & Module \\
  MG & Module Guide \\
  OS & Operating System \\
  R & Requirement\\
  SC & Scientific Computing \\
  SRS & Software Requirements Specification\\
  \progname & Explanation of program name\\
  UC & Unlikely Change \\
  \wss{etc.} & \wss{...}\\
  \bottomrule
\end{tabular}\\

\newpage

\tableofcontents

\listoftables

\listoffigures

\newpage

\pagenumbering{arabic}

\section{Introduction}

Decomposing a system into modules is a commonly accepted approach to developing
software.  A module is a work assignment for a programmer or programming
team~\citep{ParnasEtAl1984}.  We advocate a decomposition
based on the principle of information hiding~\citep{Parnas1972a}.  This
principle supports design for change, because the ``secrets'' that each module
hides represent likely future changes.  Design for change is valuable in SC,
where modifications are frequent, especially during initial development as the
solution space is explored.  

Our design follows the rules layed out by \citet{ParnasEtAl1984}, as follows:
\begin{itemize}
\item System details that are likely to change independently should be the
  secrets of separate modules.
\item Each data structure is implemented in only one module.
\item Any other program that requires information stored in a module's data
  structures must obtain it by calling access programs belonging to that module.
\end{itemize}

After completing the first stage of the design, the Software Requirements
Specification (SRS), the Module Guide (MG) is developed~\citep{ParnasEtAl1984}. The MG
specifies the modular structure of the system and is intended to allow both
designers and maintainers to easily identify the parts of the software.  The
potential readers of this document are as follows:

\begin{itemize}
\item New project members: This document can be a guide for a new project member
  to easily understand the overall structure and quickly find the
  relevant modules they are searching for.
\item Maintainers: The hierarchical structure of the module guide improves the
  maintainers' understanding when they need to make changes to the system. It is
  important for a maintainer to update the relevant sections of the document
  after changes have been made.
\item Designers: Once the module guide has been written, it can be used to
  check for consistency, feasibility, and flexibility. Designers can verify the
  system in various ways, such as consistency among modules, feasibility of the
  decomposition, and flexibility of the design.
\end{itemize}

The rest of the document is organized as follows. Section
\ref{SecChange} lists the anticipated and unlikely changes of the software
requirements. Section \ref{SecMH} summarizes the module decomposition that
was constructed according to the likely changes. Section \ref{SecConnection}
specifies the connections between the software requirements and the
modules. Section \ref{SecMD} gives a detailed description of the
modules. Section \ref{SecTM} includes two traceability matrices. One checks
the completeness of the design against the requirements provided in the SRS. The
other shows the relation between anticipated changes and the modules. Section
\ref{SecUse} describes the use relation between modules.

\section{Anticipated and Unlikely Changes} \label{SecChange}

This section lists possible changes to the system. According to the likeliness
of the change, the possible changes are classified into two
categories. Anticipated changes are listed in Section \ref{SecAchange}, and
unlikely changes are listed in Section \ref{SecUchange}.

\subsection{Anticipated Changes} \label{SecAchange}

Anticipated changes are the source of the information that is to be hidden
inside the modules. Ideally, changing one of the anticipated changes will only
require changing the one module that hides the associated decision. The approach
adapted here is called design for
change.

\begin{description}
\item[\refstepcounter{acnum} \actheacnum \label{acHardware}:] The specific
  hardware on which the software is running.
\item[\refstepcounter{acnum} \actheacnum \label{acInput}:] The format of the
  initial input data.
\item ...
\end{description}

\wss{Anticipated changes relate to changes that would be made in requirements,
design or implementation choices.  They are not related to changes that are made
at run-time, like the values of parameters.}

\subsection{Unlikely Changes} \label{SecUchange}

The module design should be as general as possible. However, a general system is
more complex. Sometimes this complexity is not necessary. Fixing some design
decisions at the system architecture stage can simplify the software design. If
these decision should later need to be changed, then many parts of the design
will potentially need to be modified. Hence, it is not intended that these
decisions will be changed.

\begin{description}
\item[\refstepcounter{ucnum} \uctheucnum \label{ucIO}:] Input/Output devices
  (Input: File and/or Keyboard, Output: File, Memory, and/or Screen).
\item ...
\end{description}

\section{Module Hierarchy} \label{SecMH}

This section provides an overview of the module design. Modules are summarized
in a hierarchy decomposed by secrets in Table \ref{TblMH}. The modules listed
below, which are leaves in the hierarchy tree, are the modules that will
actually be implemented.

\begin{description}

  % HH
  \item [\refstepcounter{mnum} \mthemnum \label{mHHIO}:] 
  \textbf{HH-IO (Audio Adapter)} – manages microphone input and audio 
  output across platforms.
  \item [\refstepcounter{mnum} \mthemnum \label{mHHAuto}:] 
  \textbf{HH-Auto (System Control)} – performs desktop automation 
  tasks such as typing, clicking, and launching applications.

  % BH
  \item [\refstepcounter{mnum} \mthemnum \label{mBHInput}:] 
  \textbf{BH-Input (Voice \& Text Manager)} – captures user speech, 
  uses Whisper STT, and normalizes input.
  \item [\refstepcounter{mnum} \mthemnum \label{mBHNLU}:] 
  \textbf{BH-NLU (Intent Parser)} – interprets text into structured 
  intents based on command patterns.
  \item [\refstepcounter{mnum} \mthemnum \label{mBHPlan}:] 
  \textbf{BH-Plan (Task Executor)} – selects MCP tools and executes 
  user intents.
  \item [\refstepcounter{mnum} \mthemnum \label{mBHSafety}:] 
  \textbf{BH-Safety (Confirmation Gate)} – validates risky actions and 
  requests confirmation as needed.
  \item [\refstepcounter{mnum} \mthemnum \label{mBHSession}:] 
  \textbf{BH-Session (Context Manager)} – stores history, session 
  context, and undo information.
  \item [\refstepcounter{mnum} \mthemnum \label{mBHFeedback}:] 
  \textbf{BH-Feedback (Response Manager)} – outputs responses using 
  TTS or visual text.
  \item [\refstepcounter{mnum} \mthemnum \label{mBHUI}:] 
  \textbf{BH-UI (Proxi Interface)} – displays status, confirmation 
  prompts, and results.

  % SD
  \item [\refstepcounter{mnum} \mthemnum \label{mSDTypes}:] 
  \textbf{SD-Types (Core Structures)} – defines ADTs for Command, 
  Intent, Plan, and related structures.
  \item [\refstepcounter{mnum} \mthemnum \label{mSDRegistry}:] 
  \textbf{SD-ToolRegistry (Action Map)} – maps intents to MCP 
  tools/actions.
  \item [\refstepcounter{mnum} \mthemnum \label{mSDStore}:] 
  \textbf{SD-Store (Local Storage)} – manages persistent data such as 
  preferences, session history, and logs.
  \item [\refstepcounter{mnum} \mthemnum \label{mSDConfig}:] 
  \textbf{SD-STT/TTS Config} – defines Whisper + TTS model configuration 
  and language settings.
  \item [\refstepcounter{mnum} \mthemnum \label{mSDLog}:] 
  \textbf{SD-Log (Event Logger)} – records events for debugging and 
  V\&V traceability.

\end{description}


\section{Connection Between Requirements and Design} \label{SecConnection}

The design of the system is intended to satisfy the requirements developed in
the SRS. In this stage, the system is decomposed into modules. The connection
between requirements and modules is listed in Table~\ref{TblRT}.

\wss{The intention of this section is to document decisions that are made
  ``between'' the requirements and the design.  To satisfy some requirements,
  design decisions need to be made.  Rather than make these decisions implicit,
  they are explicitly recorded here.  For instance, if a program has security
  requirements, a specific design decision may be made to satisfy those
  requirements with a password.}

\section{Module Decomposition} \label{SecMD}

Modules are decomposed according to the principle of ``information hiding''
proposed by \citet{ParnasEtAl1984}. The \emph{Secrets} field in a module
decomposition is a brief statement of the design decision hidden by the
module. The \emph{Services} field specifies \emph{what} the module will do
without documenting \emph{how} to do it. For each module, a suggestion for the
implementing software is given under the \emph{Implemented By} title. If the
entry is \emph{OS}, this means that the module is provided by the operating
system or by standard programming language libraries.  \emph{\progname{}} means the
module will be implemented by the \progname{} software.

Only the leaf modules in the hierarchy have to be implemented. If a dash
(\emph{--}) is shown, this means that the module is not a leaf and will not have
to be implemented.

\subsection{Hardware Hiding Modules (\mref{mHHIO})-(\mref{mHHAuto})}

\subsubsection{HH-IO: Audio Adapter (\mref{mHHIO})}
\begin{description}
\item[Secrets:]The data structure and algorithm used to interface with 
microphone and speaker hardware.
\item[Services:]Provides audio hardware I/O operations including operning and 
closing devices, recording audio and playing audio.
\item[Implemented By:] OpenAI Whisper
\item[Type Of Module:] Library 
\end{description}


\subsubsection{HH-Auto: System Control (\mref{mHHAuto})}
\begin{description}
\item[Secrets:]OS specific automations (mouse, keyboard control) implementation.
Hides differences between Windows, MacOS and Linux.
\item[Services:]Move Cursor, click, type, open and close applications.
\item[Implemented By:] PyAutoGUI
\item[Type Of Module:] Library 
\end{description}

\subsection{Behaviour-Hiding Modules (\mref{mBHInput})-(\mref{mBHUI})}

\subsubsection{BH-Input: Voice and Text Manager (\mref{mBHInput})}
\begin{description}
\item[Secrets:] Speech To Text model configuration, text normalization rules and
buffering strategies.
\item[Services:]Handles speech variation, different speech patterns, varied 
accents and filters background noise.
\item[Implemented By:] Project Proxi
\end{description}

\subsubsection{BH-NLU: Intent Parser (\mref{mBHNLU})}
\begin{description}
\item[Secrets:] Rules and internal parameters for interpretation of noisy or 
imperfect text into structured intent objects.
\item[Services:]Coverts normalized text into intents with metadata fields.
\item[Implemented By:] Project Proxi
\end{description}

\subsubsection{BH-Plan: Task Executor (\mref{mBHPlan})}
\begin{description}
\item[Secrets:] Decision-making logic for selecting correct MCP agent based on 
intent. Execution plan and task table. Error handlers and request handlers.
\item[Services:] Create an execution plan based on intent. Execute plan using 
MCP agent and track task to completion.
\item[Implemented By:] Project Proxi
\end{description}

\subsubsection{BH-Safety: Confirmation Gate (\mref{mBHSafety})}
\begin{description}
\item[Secrets:] Risk Analysis PolicyTable, Irreversible Detection algorithm,
timeout actions. 
\item[Services:] Implements accidental command protections. Classifies action 
risks, decides appropriate policy and prompts user for confirmation when 
action is deemed high risk.
\item[Implemented By:]Project Proxi
\end{description}


\subsubsection{BH-Session: Context Manager (\mref{mBHSession})}
\begin{description}
\item[Secrets:]Session History Table, short-term conversational context for AI 
model, previous action tree.
\item[Services:]Tracks all previous sessions and previous actions completed for
each session. Maintains continuity across user requests.
\item[Implemented By:] Project Proxi
\end{description}


\subsubsection{BBH-Feedback: Response Manager (\mref{mBHFeedback})}
\begin{description}
\item[Secrets:]Output action logic, Text-to-speech configuration, message delivery
monitor.
\item[Services:]Manages applications to open and actions to do based on user 
input. Maintains messages and prompts to be shown to the user. Converts messages
to speech and outputs through speakers if required.
\item[Implemented By:]Project Proxi
\end{description}


\subsubsection{BH-UI: Proxi Interface (\mref{mBHUI})}
\begin{description}
\item[Secrets:]View Groups, Information Presentation Rules and prompt timing 
rules.
\item[Services:]Updates UI State, displays messages, presents user prompts and 
show voice feedback prompts.
\item[Implemented By:]Project Proxi
\end{description}


\subsection{Software Decision Module (\mref{mSDTypes})-(\mref{mSDLog})}

\subsubsection{SD-Types: Core Structures (\mref{mSDTypes})}
\begin{description}
\item[Secrets:] Internal representaiton of commands, actions, risks, policies 
and tool metadata.
\item[Services:] Defines shared Classes and Objects used system-wide.
\item[Implemented By:] Project Proxi
\end{description}

\subsubsection{SD-ToolRegistry: Action Map (\mref{mSDRegistry})}
\begin{description}
\item[Secrets:]Maps to link action intents to MCP tools, MCP Agents and system 
automation routines.
\item[Services:]Provides appropriate MCP agent or tool for a given intent.
\item[Implemented By:] Project Proxi
\end{description}

\subsubsection{SD-Store: Local Storage (\mref{mSDStore})}
\begin{description}
\item[Secrets:]User Settings, Accessibility Settings, Session History, Action 
Tree, AI model context.
\item[Services:]Saves and loads user settings and previously saved states from 
memory. 
\item[Implemented By:] Project Proxi
\end{description}

\subsubsection{SD-STT/TTS Config (\mref{mSDConfig})}
\begin{description}
\item[Secrets:] Stores API keys, configurations and fallback options for OpenAI 
Whisper model.
\item[Services:]Passes Whisper/TTS parameters to BH-Input and parses API output 
to BH-Feedback.
\item[Implemented By:] Project Proxi
\end{description}

\subsubsection{SD-Log: Event Logger (\mref{mSDLog})}
\begin{description}
\item[Secrets:]Diagnostic log formatter, parser and storage algorithm.
\item[Services:]Stores and sends diagnostic logs in the event of failure.
\item[Implemented By:] Project Proxi
\end{description}


\section{Traceability Matrix} \label{SecTM}

This section shows two traceability matrices: between the modules and the
requirements, and between the modules and the anticipated changes. Functional
requirements are referenced using their identifiers from the SRS
(e.g.\ FUNC.R.1--FUNC.R.9).

\begin{table}[H]
\centering
\renewcommand{\arraystretch}{1.1}
\begin{tabular}{p{0.32\textwidth} p{0.58\textwidth}}
\toprule
\textbf{Requirement (SRS)} & \textbf{Modules}\\
\midrule
FUNC.R.1 -- Accept input via speech and text
  & \mref{mBHInput}, \mref{mHHIO}, \mref{mBHUI}, \mref{mSDConfig} \\
FUNC.R.2 -- Convert speech to text (STT)
  & \mref{mBHInput}, \mref{mHHIO}, \mref{mSDConfig}, \mref{mSDLog} \\
FUNC.R.3 -- Interpret user intent from NL input
  & \mref{mBHNLU}, \mref{mBHInput}, \mref{mSDTypes}, \mref{mSDLog} \\
FUNC.R.4 -- Plan and execute tasks via agents/tools
  & \mref{mBHPlan}, \mref{mSDRegistry}, \mref{mSDTypes},
    \mref{mHHAuto}, \mref{mBHSession}, \mref{mBHSafety}, \mref{mSDLog} \\
FUNC.R.5 -- Provide textual / spoken feedback
  & \mref{mBHFeedback}, \mref{mBHUI}, \mref{mSDConfig}, \mref{mSDLog} \\
FUNC.R.6 -- Maintain short-term conversational memory
  & \mref{mBHSession}, \mref{mSDStore}, \mref{mBHInput},
    \mref{mBHPlan}, \mref{mSDLog} \\
FUNC.R.7 -- Log user interactions and task results
  & \mref{mSDLog}, \mref{mSDStore}, \mref{mBHSession},
    \mref{mBHInput}, \mref{mBHPlan}, \mref{mBHFeedback},
    \mref{mBHUI}, \mref{mBHSafety}, \mref{mHHIO}, \mref{mHHAuto} \\
FUNC.R.8 -- High-level interaction (no low-level OS details)
  & \mref{mBHUI}, \mref{mBHInput}, \mref{mBHPlan},
    \mref{mBHFeedback}, \mref{mHHAuto}, \mref{mSDTypes} \\
FUNC.R.9 -- Confirm privileged / system-level actions
  & \mref{mBHSafety}, \mref{mBHUI}, \mref{mBHPlan},
    \mref{mSDTypes}, \mref{mSDLog}, \mref{mSDStore} \\
\bottomrule
\end{tabular}
\caption{Traceability between SRS functional requirements and modules}
\label{TblRT}
\end{table}

The following matrix links anticipated changes (ACs) from
Section~\ref{SecChange} to the modules that hide the corresponding design
decisions.

\begin{table}[H]
\centering
\renewcommand{\arraystretch}{1.1}
\begin{tabular}{p{0.32\textwidth} p{0.58\textwidth}}
\toprule
\textbf{Anticipated Change} & \textbf{Modules}\\
\midrule
\acref{acHardware} & \mref{mHH}\\
\acref{acInput} & \mref{mInput}\\
\acref{acParams} & \mref{mParams}\\
\acref{acVerify} & \mref{mVerify}\\
\acref{acOutput} & \mref{mOutput}\\
\acref{acVerifyOut} & \mref{mVerifyOut}\\
\acref{acODEs} & \mref{mODEs}\\
\acref{acEnergy} & \mref{mEnergy}\\
\acref{acControl} & \mref{mControl}\\
\acref{acSeqDS} & \mref{mSeqDS}\\
\acref{acSolver} & \mref{mSolver}\\
\acref{acPlot} & \mref{mPlot}\\
\bottomrule
\end{tabular}
\caption{Traceability between anticipated changes and modules}
\label{TblACT}
\end{table}


\section{Use Hierarchy Between Modules} \label{SecUse}

In this section, the uses hierarchy between modules is
provided. \citet{Parnas1978} said of two programs A and B that A {\em uses} B if
correct execution of B may be necessary for A to complete the task described in
its specification. That is, A {\em uses} B if there exist situations in which
the correct functioning of A depends upon the availability of a correct
implementation of B.  Figure \ref{FigUH} illustrates the use relation between
the modules. It can be seen that the graph is a directed acyclic graph
(DAG). Each level of the hierarchy offers a testable and usable subset of the
system, and modules in the higher level of the hierarchy are essentially simpler
because they use modules from the lower levels.

\wss{The uses relation is not a data flow diagram.  In the code there will often
be an import statement in module A when it directly uses module B.  Module B
provides the services that module A needs.  The code for module A needs to be
able to see these services (hence the import statement).  Since the uses
relation is transitive, there is a use relation without an import, but the
arrows in the diagram typically correspond to the presence of import statement.}

\wss{If module A uses module B, the arrow is directed from A to B.}

\begin{figure}[H]
\centering
%\includegraphics[width=0.7\textwidth]{UsesHierarchy.png}
\caption{Use hierarchy among modules}
\label{FigUH}
\end{figure}

%\section*{References}

\section{User Interfaces}

\wss{Design of user interface for software and hardware.  Attach an appendix if
needed. Drawings, Sketches, Figma}

\section{Design of Communication Protocols}

\wss{If appropriate}

\section{Timeline}

\wss{Schedule of tasks and who is responsible}

\wss{You can point to GitHub if this information is included there}

\bibliographystyle {plainnat}
\bibliography{../../../refs/References}

\newpage{}

\end{document}