\documentclass[12pt, titlepage]{article}

\usepackage{amsmath, mathtools}

\usepackage[round]{natbib}
\usepackage{amsfonts}
\usepackage{amssymb}
\usepackage{graphicx}
\usepackage{colortbl}
\usepackage{xr}
\usepackage{hyperref}
\usepackage{longtable}
\usepackage{xfrac}
\usepackage{tabularx}
\usepackage{float}
\usepackage{siunitx}
\usepackage{booktabs}
\usepackage{multirow}
\usepackage[section]{placeins}
\usepackage{caption}
\usepackage{fullpage}
\usepackage[utf8]{inputenc}

\hypersetup{
bookmarks=true,
colorlinks=true,
linkcolor=red,
citecolor=blue,
filecolor=magenta,
urlcolor=cyan
}

\usepackage{array}

\externaldocument{../../SRS/SRS}

\input{../../Comments}
%% Common Parts

\newcommand{\progname}{Software Engineering} % PUT YOUR PROGRAM NAME HERE
\newcommand{\authname}{Team \#23, Project Proxi
\\ Savinay Chhabra
\\ Amanbeer Singh Minhas
\\ Gourob Podder
\\ Ajay Grewal} % AUTHOR NAMES                  

\usepackage{hyperref}
    \hypersetup{colorlinks=true, linkcolor=blue, citecolor=blue, filecolor=blue,
                urlcolor=blue, unicode=false}
    \urlstyle{same}
                                


\begin{document}

\title{Module Interface Specification for \progname{}}

\author{\authname}

\date{\today}

\maketitle

\pagenumbering{roman}

\section{Revision History}

\begin{tabularx}{\textwidth}{p{3cm}p{2cm}X}
\toprule
Date & Version & Notes \\
\midrule
2025-11-02 & 1.0 & Initial draft created. Included module hierarchy and
core Proxi design overview. \\
2025-11-03 & 1.1 & Added detailed MIS for HH-IO, HH-Auto, and BH-Input
modules. \\
2025-11-04 & 1.2 & Completed BH-NLU, BH-Plan, BH-Safety, BH-Feedback,
and BH-UI sections with semantics and tables. \\
2025-11-07 & 1.3 & Added refined symbols and abbreviations and reflection
section \\
2025-11-11 & 1.4 & Final formatting, line-width cleanup, and rubric
alignment for submission. \\
2025-01-21 & 1.4 & Rev 0, implemented TA Feedback \\
\bottomrule
\end{tabularx}

~\newpage

\section{Symbols, Abbreviations and Acronyms}

See SRS documentation at
\url{https://github.com/gkpodder/Capstone/blob/design/docs/SRS-Volere/%
SRS.pdf}

\vspace{0.5cm}

\noindent Additional symbols, abbreviations, and acronyms specific to this
document are listed below.

\begin{center}
\renewcommand{\arraystretch}{1.2}
\begin{tabular}{p{3cm} p{11cm}}
\toprule
Symbol / Term & Definition \\
\midrule
:= & Assignment operator used for state transitions.\\
$\geq, \leq$ & Greater than or equal to, less than or equal to.\\
$\rightarrow$ & Function mapping or transition arrow.\\
Action & Atomic executable system behaviour.\\
BH & Behaviour-Hiding module (e.g., BH-Input, BH-Plan).\\
ExecStatus & Execution result (Pending, Success, or Failed).\\
FUNC.R.\# & Functional requirement number from the SRS.\\
Hazard ID & Identifier from Hazard Analysis (e.g., H1, H2).\\
HH & Hardware-Hiding module (e.g., HH-IO, HH-Auto).\\
InputMode & Input type (VoiceOnly, TextOnly, Mixed).\\
Intent & Structured interpretation of user command text.\\
MCP & Modular Command Protocol agent interface.\\
OutputMode & Output type (VoiceOnly, TextOnly, Both).\\
Plan & Structured action sequence from BH-Plan.\\
QA & Quality Assurance (software testing process).\\
RiskLevel & Safety classification for user actions.\\
SD & Software-Decision module (e.g., SD-Types, SD-Log).\\
SRS & System Requirements Specification.\\
STT & Speech-to-Text (audio input converted to text).\\
TTS & Text-to-Speech (text output converted to speech).\\
UI & User Interface.\\
V\&V & Verification and Validation Plan.\\
\bottomrule
\end{tabular}
\end{center}

\newpage

\tableofcontents

\newpage

\pagenumbering{arabic}

\section{Introduction}
\noindent This document presents the Module Interface Specification (MIS) for
Proxi, an intelligent voice first assistant developed as part of the
SFWRENG 4G06A Capstone Design Project at McMaster University.
Proxi enables users to interact with their computer using speech or
text commands and executes actions through a modular command protocol
(MCP) agent system. The goal is to improve accessibility and user
productivity by providing hands free, context aware computer control.

\vspace{0.5cm}
\noindent The MIS defines the interface and behaviour of each module described
in the Module Guide. Each module encapsulates a single responsibility and
follows the principles of information hiding and low coupling. Together these
modules define how Proxi captures user input, interprets intent, plans actions,
confirms operations, and provides feedback.

\vspace{0.5cm}
\noindent Complementary documents include:
\begin{itemize}
  \item System Requirements Specification (SRS):
  \url{https://github.com/gkpodder/Capstone/blob/design/docs/SRS-Volere/%
SRS.pdf}
  \item Verification and Validation Plan (V\&V):
  \url{https://github.com/gkpodder/Capstone/blob/design/docs/VnVPlan/%
VnVPlan.pdf}
  \item Hazard Analysis Report:
  \url{https://github.com/gkpodder/Capstone/blob/design/docs/HazardAnalysis/%
HazardAnalysis.pdf}
\end{itemize}

\noindent All documents are stored in the team’s public repository at
\url{https://github.com/gkpodder/Capstone/}.
The SRS and V\&V Plan define the measurable requirements and testing
criteria that this MIS traces to. The Hazard Analysis identifies
potential risks, such as unintended system actions, which are addressed
by the BH-Safety module.

\noindent This MIS uses the structure and conventions from
\citet{HoffmanAndStrooper1995} and \citet{GhezziEtAl2003}. Each module is
presented with clear syntax, semantics, and local function definitions,
ensuring traceability between requirements, design, and testing. The notation
style for states, transitions, and functions follows the mathematical
conventions outlined in Section \ref{Notation}. Overall, this document provides
a formal yet readable description of Proxi’s modular design, supporting future
implementation, testing, and maintenance.

\section{Notation}
\label{Notation}

The structure of the MIS for modules comes from \citet{HoffmanAndStrooper1995},
with the addition that template modules have been adapted from
\cite{GhezziEtAl2003}. The mathematical notation comes from Chapter 3 of
\citet{HoffmanAndStrooper1995}. For instance, the symbol := is used for a
multiple assignment statement and conditional rules follow the form
$(c_1 \Rightarrow r_1 | c_2 \Rightarrow r_2 | ... | c_n \Rightarrow r_n )$.

\noindent The following table summarizes the basic and derived data types used
by Proxi modules.

\begin{center}
\renewcommand{\arraystretch}{1.2}
\noindent
\begin{tabular}{p{3cm} p{10.5cm}}
\toprule
Data Type & Description \\
\midrule
char & A single character or digit.\\
integer & Whole number in the range $(-\infty, \infty)$.\\
real & Number with fractional part, used for timing or duration.\\
boolean & Logical value in \{true, false\}.\\
string & Sequence of characters.\\
sequence(T) & Ordered list of elements of type T.\\
tuple(T$_1$,T$_2$, \ldots) & Finite ordered group of typed values.\\
AudioStream & Representation of sampled voice input.\\
Intent & Structured record containing type and parameters.\\
Plan & Record defining tool, parameters, and execution time.\\
Action & Atomic operation triggered by a plan or intent.\\
RiskLevel & Enum \{Low, Medium, High\} for safety checks.\\
ExecStatus & Enum \{Pending, Success, Failed\} for task outcomes.\\
\bottomrule
\end{tabular}
\end{center}

\noindent The specification of \progname\ uses some derived data types:
sequences, strings, and tuples. Sequences are lists filled with elements of the
same data type. Strings are sequences of characters. Tuples contain a list of
values, potentially of different types. In addition, \progname\ uses functions,
which are defined by the data types of their inputs and outputs.

\noindent Derived functions and transitions are written in the form
$f : A \rightarrow B$ to indicate a function mapping from type A to type B.
Local functions such as \textit{nextState} or \textit{policy} are used to
describe the intended mathematical behaviour of modules but may not exist as
explicit code implementations.

\section{Module Decomposition}

The Proxi is decomposed into a hierarchy of modules following the design
principles of information hiding and separation of concerns. Each module
corresponds to a well-defined secret and is independently assignable to a
developer. The decomposition balances hardware hiding, behaviour hiding, and
software decision modules.

\begin{table}[H]
\centering
\caption{Module Hierarchy for Proxi Voice Assistant}
\label{TblMH}
\renewcommand{\arraystretch}{1.2}
\begin{tabular}{p{0.3\textwidth} p{0.65\textwidth}}
\toprule
\textbf{Level 1} & \textbf{Level 2 Modules (Secrets / Responsibilities)}\\
\midrule

\multirow{2}{*}{\textbf{Hardware-Hiding}} &
\textbf{HH-IO (Audio I/O Adapter)} manages microphone input and audio
output across different platforms.\\
& \textbf{HH-Auto (Desktop Automation)} performs basic desktop actions
such as typing, clicking, or launching applications.\\
\midrule

\multirow{7}{*}{\textbf{Behaviour-Hiding}} &
\textbf{BH-Input (Voice \& Text Manager)} captures user input,
converts speech to text, and normalizes text commands. Implements
\textit{FUNC.R.1–R.2}.\\
& \textbf{BH-NLU (Intent Parser)} interprets text into structured
intents and parameters based on defined command patterns. Implements
\textit{FUNC.R.3}.\\
& \textbf{BH-Plan (Task Executor)} determines which agent or tool
should handle a command and coordinates its execution. Implements
\textit{FUNC.R.4}.\\
& \textbf{BH-Safety (Confirmation Gate)} validates actions that may
affect files or system settings and requests confirmation. Implements
\textit{FUNC.R.9}.\\
& \textbf{BH-Session (Context Manager)} maintains user session data,
history, and undo information for continuity. Supports \textit{FUNC.R.4}.\\
& \textbf{BH-Feedback (Response Manager)} converts textual responses
into spoken or visual feedback for the user. Implements
\textit{FUNC.R.5–R.6}.\\
& \textbf{BH-UI (Proxi Interface)} presents status updates,
confirmations, and results; supports full voice-only interaction.
Implements \textit{FUNC.R.8}.\\
\midrule

\multirow{5}{*}{\textbf{Software-Decision}} &
\textbf{SD-Types (Core Structures)} defines abstract data types for
Command, Intent, and Plan.\\
& \textbf{SD-ToolRegistry (Action Map)} maintains the mapping between
recognized intents and available system actions.\\
& \textbf{SD-Store (Local Storage)} handles persistent storage for user
preferences, session history, and logs.\\
& \textbf{SD-AIClient (AI Service Client)} provides configuration and
communication for STT and text synthesis services.\\
& \textbf{SD-Log (Event Logger)} records system actions and feedback
events for debugging and validation.\\
\bottomrule
\end{tabular}
\end{table}

\textbf{Likely Changes:}
\begin{itemize}
  \item The choice of speech recognition or text-to-speech library
  (for example, switching from Whisper API to a local model).
  \item Adjustments to the user interface layout or how voice commands
  trigger visible feedback or audio playback.
  \item Fine-tuning thresholds for speech detection and timing between
  input and response based on user testing.
  \item Updating supported voice commands or adding new MCP tools as
  features are expanded.
\end{itemize}

\textbf{Unlikely Changes:}
\begin{itemize}
  \item The main processing loop of Input $\rightarrow$ Interpret
  $\rightarrow$ Plan $\rightarrow$ Execute $\rightarrow$ Feedback.
  \item The core data structures used for storing Commands, Intents, and
  Action Plans.
  \item A local, reliable, request-response communication style between
  modules via the MCP agent interface.
\end{itemize}

\textbf{Traceability to SRS:}
\begin{itemize}
  \item \textbf{BH-Input} fulfills \textit{FUNC.R.1–R.2}: speech and text
  input handling with accuracy $\geq 90\%$.
  \item \textbf{BH-NLU} fulfills \textit{FUNC.R.3}: intent recognition
  accuracy $\geq 90\%$.
  \item \textbf{BH-Plan} fulfills \textit{FUNC.R.4}: agent planning and
  execution success rate $\geq 85\%$.
  \item \textbf{BH-Feedback} fulfills \textit{FUNC.R.5–R.6}: provides
  feedback and spoken confirmation within response time $\leq 2\,\text{s}$.
  \item \textbf{BH-UI} fulfills \textit{FUNC.R.8}: supports full hands-free
  operation for accessibility.
  \item \textbf{BH-Safety} fulfills \textit{FUNC.R.9}: requests confirmation
  before executing high-risk or destructive actions.
  \item \textbf{Support modules (SD, HH)} enable non-functional goals on
  latency, reliability, and auditability through structured logging.
\end{itemize}

\newpage

\section{MIS of HH-IO (Audio I/O Adapter)} \label{mod:HHIO}

\subsection{Module}

HH-IO (Audio I/O Adapter)

\subsection{Type}

Abstract object.

\subsection{Uses}

System audio interface

\subsection{Syntax}

\subsubsection{Exported Constants}

None.

\subsubsection{Exported Access Programs}

\begin{center}
\begin{tabular}{p{2.7cm} p{4cm} p{4cm} p{2.5cm}}
\hline
\textbf{Name} & \textbf{Input} & \textbf{Output} & \textbf{Errors} \\
\hline
initAudio & N/A & N/A & AudioInitFailed \\
openMic & N/A & N/A & MicNotFound \\
closeMic & N/A & N/A & CloseFailed \\
recordAudio & seconds: real & sound: AudioStream & RecordFailed \\
playAudio & sound: AudioStream & N/A & PlaybackFailed \\
\hline
\end{tabular}
\end{center}

\subsection{Semantics}

\subsubsection{State Variables}

\begin{itemize}
  \item micOpen : boolean
\end{itemize}

\subsubsection{Environment Variables}

\begin{itemize}
  \item micDevice : physical microphone
  \item speakerDevice : physical speaker or headset
\end{itemize}

\subsubsection{Assumptions}

\begin{itemize}
  \item At least one working microphone and speaker device exists.
  \item Only one module controls the microphone at a time.
  \item \textbf{initAudio} is called before any other access routine.
\end{itemize}

\subsubsection{Access Routine Semantics}

\noindent
\textbf{initAudio}():
\begin{itemize}
  \item transition: $\text{micOpen} := \text{false}$ if audio devices exist.
  \item exception: AudioInitFailed if audio devices cannot be initialized.
\end{itemize}

\noindent
\textbf{openMic}():
\begin{itemize}
  \item transition: $\text{micOpen} := \text{true}$ if micDevice is available.
  \item exception: MicNotFound if micDevice is missing or busy.
\end{itemize}

\noindent
\textbf{closeMic}():
\begin{itemize}
  \item transition: $\text{micOpen} := \text{false}$ if open.
  \item exception: CloseFailed if device cannot close.
\end{itemize}

\noindent
\textbf{recordAudio}(seconds):
\begin{itemize}
  \item output: returns an AudioStream captured from micDevice for
  \textit{seconds}.
  \item exception: RecordFailed if capture fails or $\text{micOpen} = false$.
\end{itemize}

\noindent
\textbf{playAudio}(sound):
\begin{itemize}
  \item transition: speakerDevice plays \textit{sound}.
  \item exception: PlaybackFailed if playback fails.
\end{itemize}

\subsubsection{Local Functions}

None.

\section{MIS of HH-Auto (Desktop Automation)} \label{mod:HHAuto}

\subsection{Module}

HH-Auto (Desktop Automation)

\subsection{Type}

Abstract object.

\subsection{Uses}

Operating system automation interface

\subsection{Syntax}

\subsubsection{Exported Constants}

None.

\subsubsection{Exported Access Programs}

\begin{center}
\begin{tabular}{p{2.7cm} p{4cm} p{4cm} p{2.5cm}}
\hline
\textbf{Name} & \textbf{Input} & \textbf{Output} & \textbf{Errors} \\
\hline
initAuto & N/A & N/A & AutoInitFailed \\
moveCursor & p: ScreenPos & N/A & ActionError \\
leftClick & N/A & N/A & ActionError \\
typeText & t: String & N/A & ActionError \\
openApp & id: AppId & N/A & ActionError \\
\hline
\end{tabular}
\end{center}

\subsection{Semantics}

\subsubsection{State Variables}

None.

\subsubsection{Environment Variables}

\begin{itemize}
  \item desktopEnv : user desktop environment
  \item keyboardDevice : keyboard input channel
  \item pointingDevice : mouse or trackpad input channel
  \item pointerPos : current cursor position on the active display
\end{itemize}

\subsubsection{Assumptions}

\begin{itemize}
  \item The user session allows simulated input events.
  \item Screen coordinates are valid for the active display.
  \item AppId refers to an installed and accessible application.
  \item \textbf{initAuto} is called before any other access routine.
\end{itemize}

\subsubsection{Access Routine Semantics}

\noindent
\textbf{initAuto}():
\begin{itemize}
  \item transition: automation channels to desktopEnv, keyboardDevice, and
  pointingDevice are established.
  \item exception: AutoInitFailed if automation cannot be initialized.
\end{itemize}

\noindent
\textbf{moveCursor}(p):
\begin{itemize}
  \item transition: $\text{pointerPos} := p$.
  \item exception: ActionError if cursor move fails.
\end{itemize}

\noindent
\textbf{leftClick}():
\begin{itemize}
  \item transition: a left-click event is emitted on pointingDevice at
  pointerPos.
  \item exception: ActionError if click event fails.
\end{itemize}

\noindent
\textbf{typeText}(t):
\begin{itemize}
  \item transition: keystroke events for t are emitted on keyboardDevice.
  \item exception: ActionError if key input fails.
\end{itemize}

\noindent
\textbf{openApp}(id):
\begin{itemize}
  \item transition: desktopEnv launches the application identified by id.
  \item exception: ActionError if app launch fails.
\end{itemize}

\subsubsection{Local Functions}

None.

\section{MIS of BH-Input (Voice \& Text Manager)} \label{mod:BHInput}

\subsection{Module}

BH-Input (Voice \& Text Manager)

\subsection{Type}

Abstract object.

\subsection{Uses}

HH-IO, SD-AIClient, SD-Types, SD-Log

\subsection{Syntax}

\subsubsection{Exported Constants}

None.

\subsubsection{Exported Access Programs}

\begin{center}
\begin{tabular}{p{2.7cm} p{4cm} p{4cm} p{2.5cm}}
\hline
\textbf{Name} & \textbf{Input} & \textbf{Output} & \textbf{Errors} \\
\hline
initInput & N/A & N/A & InputInitFailed \\
startCapture & mode: InputMode & N/A &
MicUnavailable, AlreadyCapturing \\
stopCapture & N/A & N/A & NotCapturing \\
getLastText & N/A & text: String & NoInputAvailable \\
getStatus & N/A & s: InputStatus & N/A \\
\hline
\end{tabular}
\end{center}

\subsection{Semantics}

\subsubsection{State Variables}

\begin{itemize}
  \item inputState : InputState
  \item currentMode : InputMode
  \item lastText : String
  \item partialText : String
  \item lastError : InputError or null
\end{itemize}

InputState = \{\text{Idle}, \text{Listening}, \text{Processing}\}

InputMode = \{\text{VoiceOnly}, \text{TextOnly}, \text{Mixed}\}

InputStatus is a record:
\begin{itemize}
  \item state : InputState
  \item hasText : boolean
  \item hasError : boolean
\end{itemize}

\subsubsection{Environment Variables}

\begin{itemize}
  \item micStream : handled by HH-IO for live audio
  \item sttService : speech-to-text service client (through SD-AIClient)
  \item now : system clock for timing
\end{itemize}

\subsubsection{Assumptions}

\begin{itemize}
  \item The microphone and STT component are available when started.
  \item Only one capture session runs at a time.
  \item Calling modules handle all exceptions raised.
  \item \textbf{initInput} is called before any other access routine.
\end{itemize}

\subsubsection{Access Routine Semantics}

\noindent
\textbf{initInput}():
\begin{itemize}
  \item transition:
  \[
    \text{inputState}, \text{currentMode}, \text{lastText}, \text{partialText}
    := \text{Idle}, \text{TextOnly}, "", ""
  \]
  \item exception: InputInitFailed if dependencies cannot be initialized.
\end{itemize}

\noindent
\textbf{startCapture}(mode):
\begin{itemize}
  \item transition: if $\text{inputState} = \text{Idle}$ and micStream ready then
  \[
    \text{inputState}, \text{currentMode}, \text{partialText}
    := \text{Listening}, \text{mode}, ""
  \]
  \item exception: MicUnavailable if device fails, AlreadyCapturing if
  $\text{inputState} \neq \text{Idle}$.
\end{itemize}

\noindent
\textbf{stopCapture}():
\begin{itemize}
  \item transition: if $\text{inputState} \neq \text{Idle}$ then
  $\text{inputState} := \text{Idle}$.
  \item exception: NotCapturing if $\text{inputState} = \text{Idle}$.
\end{itemize}

\noindent
\textbf{getLastText}():
\begin{itemize}
  \item output: returns lastText if not empty.
  \item exception: NoInputAvailable if lastText is empty.
\end{itemize}

\noindent
\textbf{getStatus}():
\begin{itemize}
  \item output: returns a record s with
  $s.\text{state} = \text{inputState}$,
  $s.\text{hasText} = (\text{lastText} \neq "")$,
  $s.\text{hasError} = (\text{lastError} \neq null)$.
\end{itemize}

\subsubsection{Local Functions}

Let Event = \{\text{StartCmd}, \text{StopCmd}, \text{Chunk}, \text{Error},
\text{Timeout}\}

\noindent
$\text{nextState} : \text{InputState} \times \text{Event} \rightarrow
\text{InputState}$

\[
  \text{nextState}(s, e) =
  \begin{cases}
    \text{Listening} & \text{if } s = \text{Idle} \land e = \text{StartCmd}\\
    \text{Idle} & \text{if } s = \text{Listening} \land e = \text{StopCmd}\\
    \text{Processing} & \text{if } s = \text{Listening} \land e = \text{Chunk}\\
    \text{Processing} & \text{if } s = \text{Processing} \land e = \text{Chunk}\\
    \text{Idle} & \text{if } e = \text{Error} \lor e = \text{Timeout}\\
    s & \text{otherwise}
  \end{cases}
\]

During execution BH-Input updates
\[
  \text{inputState} := \text{nextState}(\text{inputState}, e)
\]
for each event e. When transcription ends, partialText moves into lastText.

\section{MIS of BH-NLU (Intent Parser)} \label{mod:BHNlu}

\subsection{Module}

BH-NLU (Intent Parser)

\subsection{Type}

Library.

\subsection{Uses}

SD-Types, SD-Log

\subsection{Syntax}

\subsubsection{Exported Access Programs}

\begin{center}
\begin{tabular}{p{2.7cm} p{4cm} p{4cm} p{2.5cm}}
\hline
\textbf{Name} & \textbf{Input} & \textbf{Output} & \textbf{Errors} \\
\hline
parseText & text: String & i: Intent & ParseError \\
\hline
\end{tabular}
\end{center}

\subsection{Semantics}

\subsubsection{State Variables}

None.

\subsubsection{Environment Variables}

None.

\subsubsection{Assumptions}

\begin{itemize}
  \item Input text may be noisy, incomplete, or ungrammatical.
  \item Command patterns and parameter schemas are defined in SD-Types.
\end{itemize}

\subsubsection{Access Routine Semantics}

\noindent
\textbf{parseText}(text):
\begin{itemize}
  \item output: produces an Intent record with fields
  \[
    i.\text{type} = \text{detectCommand}(\text{text}), \quad
    i.\text{params} = \text{extractParams}(\text{text})
  \]
  \item exception: ParseError if text cannot be matched to any pattern.
\end{itemize}

\subsubsection{Local Functions}

\[
  \text{detectCommand} : \text{String} \rightarrow \text{IntentType}
\]
\[
  \text{extractParams} : \text{String} \rightarrow \text{ParamSet}
\]

\noindent
\text{detectCommand} matches the normalized input against a finite set of
command patterns (keywords, regular expressions, and templates) defined in
SD-Types. \text{extractParams} parses any parameters required by the selected
pattern (for example, filenames, application identifiers, or search text) using
the parameter schemas from SD-Types. Both routines are deterministic and do not
require external network calls.

\section{MIS of BH-Plan (Task Executor)} \label{mod:BHPlan}

\subsection{Module}

BH-Plan (Task Executor)

\subsection{Type}

Abstract object.

\subsection{Uses}

BH-NLU, SD-ToolRegistry, SD-Types, SD-Log, HH-Auto

\subsection{Syntax}

\subsubsection{Exported Constants}

ExecStatus = \{\text{Pending}, \text{Success}, \text{Failed}\}

\subsubsection{Exported Access Programs}

\begin{center}
\begin{tabular}{p{2.7cm} p{4cm} p{4cm} p{2.5cm}}
\hline
\textbf{Name} & \textbf{Input} & \textbf{Output} & \textbf{Errors} \\
\hline
initPlan & N/A & N/A & PlanInitFailed \\
planAction & i: Intent & p: Plan & NoToolFound \\
executePlan & p: Plan & s: ExecStatus & ExecError \\
cancelPlan & N/A & N/A & NoPendingPlan \\
getLastStatus & N/A & s: ExecStatus & N/A \\
\hline
\end{tabular}
\end{center}

\subsection{Semantics}

\subsubsection{State Variables}

\begin{itemize}
  \item currentPlan : Plan or null
  \item lastStatus : ExecStatus
\end{itemize}

\subsubsection{Environment Variables}

\begin{itemize}
  \item toolSet : accessible system tools or MCP agents
  \item now : system clock for execution timing
\end{itemize}

\subsubsection{Assumptions}

\begin{itemize}
  \item The input intent has been validated by BH-NLU.
  \item Each available tool in SD-ToolRegistry exposes a run() routine.
  \item MCP agents or automation tools are reachable when requested.
  \item \textbf{initPlan} is called before any other access routine.
\end{itemize}

\subsubsection{Access Routine Semantics}

\noindent
\textbf{initPlan}():
\begin{itemize}
  \item transition:
  \[
    \text{currentPlan}, \text{lastStatus} := null, \text{Pending}
  \]
  \item exception: PlanInitFailed if tool registry cannot be accessed.
\end{itemize}

\noindent
\textbf{planAction}(i):
\begin{itemize}
  \item output: generates a Plan record p with:
  \[
    p.\text{tool} = \text{matchTool}(i.\text{type}), \quad
    p.\text{parameters} = i.\text{params}, \quad
    p.\text{time} = \text{now}
  \]
  \item transition: $\text{currentPlan} := p$.
  \item exception: NoToolFound if matchTool fails.
\end{itemize}

\noindent
\textbf{executePlan}(p):
\begin{itemize}
  \item transition:
  \[
    \text{currentPlan}, \text{lastStatus}
    := p, \text{runTool}(p.\text{tool}, p.\text{parameters})
  \]
  \item output: returns lastStatus.
  \item exception: ExecError if runTool fails unexpectedly.
\end{itemize}

\noindent
\textbf{cancelPlan}():
\begin{itemize}
  \item transition: $\text{currentPlan}, \text{lastStatus} := null, \text{Failed}$.
  \item exception: NoPendingPlan if currentPlan = null.
\end{itemize}

\noindent
\textbf{getLastStatus}():
\begin{itemize}
  \item output: returns lastStatus.
\end{itemize}

\subsubsection{Planning Logic}

Define:
\[
  \text{matchTool} : \text{IntentType} \rightarrow \text{ToolId}
\]
\[
  \text{runTool} : \text{ToolId} \times \text{ParamSet} \rightarrow \text{ExecStatus}
\]

\noindent
The planning decision can be expressed as:

\[
  \text{planAction}(i) =
  \begin{cases}
    p = (\text{matchTool}(i.\text{type}), i.\text{params}, \text{now}) &
    \text{if a tool exists for } i.\text{type} \\
    \text{NoToolFound error} & \text{otherwise}
  \end{cases}
\]

\noindent
Execution behaviour follows:
\[
  executePlan(p) =
  \begin{cases}
    Success & \text{if runTool(p.tool, p.parameters) = true}\\
    Failed & \text{otherwise}
  \end{cases}
\]

\subsubsection{Local Functions}

\begin{itemize}
  \item \textbf{matchTool}(t): searches SD-ToolRegistry for a matching tool.
  \item \textbf{runTool}(id, params): calls the linked MCP or system command.
\end{itemize}

\section{MIS of BH-Safety (Confirmation Gate)} \label{mod:BHSafety}

\subsection{Module}

BH-Safety (Confirmation Gate)

\subsection{Type}

Abstract object.

\subsection{Uses}

BH-UI, SD-Types, SD-Log

\subsection{Syntax}

\subsubsection{Exported Constants}

RiskLevel = \{\text{Low}, \text{Medium}, \text{High}\}

\noindent SafetyDecision = \{\text{AutoAllow}, \text{AskUser}, \text{Block}\}

\noindent ApprovalResult = \{\text{Approved}, \text{Denied}, \text{Cancelled}\}

\subsubsection{Exported Access Programs}

\begin{center}
\begin{tabular}{p{2.7cm} p{4cm} p{4cm} p{2.5cm}}
\hline
\textbf{Name} & \textbf{Input} & \textbf{Output} & \textbf{Errors} \\
\hline
classifyAction & a: Action & r: RiskLevel & N/A \\
decidePolicy & a: Action & d: SafetyDecision & N/A \\
confirmAction & a: Action & res: ApprovalResult & UserTimeout \\
\hline
\end{tabular}
\end{center}

\subsection{Semantics}

\subsubsection{State Variables}

\begin{itemize}
  \item pendingAction : Action or null
  \item lastDecision : SafetyDecision or null
\end{itemize}

\subsubsection{Environment Variables}

\begin{itemize}
  \item uiChannel : connection to BH-UI for user prompts
  \item now : system clock for time limits
\end{itemize}

\subsubsection{Assumptions}

\begin{itemize}
  \item BH-UI can show a yes/no prompt and return a user response.
  \item Every Action record includes a defined riskLevel field.
  \item The system clock is monotonic for timeout checks.
\end{itemize}

\subsubsection{Access Routine Semantics}

\noindent
\textbf{classifyAction}(a):

\noindent Output:
\[
  r := a.\text{riskLevel}
\]
\noindent Transition: none.

\vspace{0.2cm}

\noindent
\textbf{decidePolicy}(a):
\begin{itemize}
  \item output: returns d of type SafetyDecision, where
  \[
    d =
    \begin{cases}
      AutoAllow & \text{if } a.riskLevel = Low\\
      AskUser & \text{if } a.riskLevel = Medium\\
      Block & \text{if } a.riskLevel = High \land a.isIrreversible\\
      AskUser & \text{if } a.riskLevel = High \land \neg a.isIrreversible
    \end{cases}
  \]
  \item transition: lastDecision := d.
\end{itemize}

\noindent
\textbf{confirmAction}(a):
\begin{itemize}
  \item transition: $\text{pendingAction} := a$.
  \item output:
  \begin{itemize}
    \item If decidePolicy(a) = AutoAllow then res = Approved.
    \item If decidePolicy(a) = Block then res = Denied.
    \item If decidePolicy(a) = AskUser then BH-UI prompts user.
  \end{itemize}
  \item exception: UserTimeout if no answer before time limit.
\end{itemize}

\subsubsection{Risk Policy Table}

\begin{center}
\begin{tabular}{p{2cm} p{3cm} p{3cm} p{3cm}}
\toprule
\textbf{Risk} & \textbf{Irreversible?} & \textbf{Decision} & \textbf{Example} \\
\midrule
Low & N/A & AutoAllow & Open folder, read file \\
Medium & N/A & AskUser & Rename or move file \\
High & false & AskUser & Delete to recycle bin \\
High & true & Block & Permanently delete data \\
\bottomrule
\end{tabular}
\end{center}

\subsubsection{Local Functions}

\[
  \text{policy} : \text{RiskLevel} \times \text{boolean} \rightarrow
  \text{SafetyDecision}
\]

\[
  \text{policy}(r, irr) =
  \begin{cases}
    \text{AutoAllow} & \text{if } r = \text{Low}\\
    \text{AskUser} & \text{if } r = \text{Medium}\\
    \text{Block} & \text{if } r = \text{High} \land irr = \text{true}\\
    \text{AskUser} & \text{if } r = \text{High} \land irr = \text{false}
  \end{cases}
\]

\noindent
This module fulfills \textit{FUNC.R.9} by ensuring confirmation or blocking of
high-risk actions, reducing hazards identified in the safety analysis.

\section{MIS of BH-Feedback (Response Manager)} \label{mod:BHFeedback}

\subsection{Module}

BH-Feedback (Response Manager)

\subsection{Type}

Abstract object.

\subsection{Uses}

HH-IO, SD-AIClient, SD-Types, SD-Log

\subsection{Syntax}

\subsubsection{Exported Constants}

OutputMode = \{\text{VoiceOnly}, \text{TextOnly}, \text{Both}\}

\noindent FeedbackStatus = \{\text{Idle}, \text{Speaking}, \text{Completed},
\text{Failed}\}

\subsubsection{Exported Access Programs}

\begin{center}
\begin{tabular}{p{2.7cm} p{4cm} p{4cm} p{2.5cm}}
\hline
\textbf{Name} & \textbf{Input} & \textbf{Output} & \textbf{Errors} \\
\hline
initFeedback & N/A & N/A & FeedbackInitFailed \\
queueMessage & msg: String, m: OutputMode & N/A & QueueFull \\
speakNow & msg: String, m: OutputMode & s: FeedbackStatus & TtsError \\
getLastStatus & N/A & s: FeedbackStatus & N/A \\
cancelAll & N/A & N/A & N/A \\
\hline
\end{tabular}
\end{center}

\subsection{Semantics}

\subsubsection{State Variables}

\begin{itemize}
  \item outputQueue : sequence of (String, OutputMode)
  \item lastStatus : FeedbackStatus
  \item isSpeaking : boolean
\end{itemize}

\subsubsection{Environment Variables}

\begin{itemize}
  \item audioOut : speaker connection through HH-IO
  \item ttsService : text-to-speech component (through SD-AIClient)
  \item uiChannel : text output channel to BH-UI
\end{itemize}

\subsubsection{Assumptions}

\begin{itemize}
  \item ttsService can turn any short message into speech in less than the
  required response time from the SRS.
  \item HH-IO can play audio without blocking the whole system.
  \item The output queue has a fixed maximum size.
  \item \textbf{initFeedback} is called before any other access routine.
\end{itemize}

\subsubsection{Access Routine Semantics}

\noindent
\textbf{initFeedback}():
\begin{itemize}
  \item transition:
  \[
    \text{outputQueue}, \text{lastStatus}, \text{isSpeaking}
    := [], \text{Idle}, \text{false}
  \]
  \item exception: FeedbackInitFailed if dependencies cannot be initialized.
\end{itemize}

\noindent
\textbf{queueMessage}(msg, m):
\begin{itemize}
  \item transition: if the queue is not full then append (msg, m) to outputQueue.
  \item exception: QueueFull if appending would exceed the limit.
\end{itemize}

\noindent
\textbf{speakNow}(msg, m):
\begin{itemize}
  \item transition: $\text{isSpeaking}, \text{lastStatus} := \text{true},
  \text{Speaking}$.
  \item transition: if $m = \text{VoiceOnly}$ or $m = \text{Both}$ then
  ttsService produces speech for msg and HH-IO plays it on audioOut.
  \item transition: if $m = \text{TextOnly}$ or $m = \text{Both}$ then
  uiChannel displays msg.
  \item output:
  \begin{itemize}
    \item If both requested outputs succeed, then
    $\text{isSpeaking}, \text{lastStatus} := \text{false}, \text{Completed}$.
    \item Otherwise
    $\text{isSpeaking}, \text{lastStatus} := \text{false}, \text{Failed}$.
  \end{itemize}
  \item exception: TtsError if ttsService cannot produce speech.
\end{itemize}

\noindent
\textbf{getLastStatus}():
\begin{itemize}
  \item output: returns lastStatus.
\end{itemize}

\noindent
\textbf{cancelAll}():
\begin{itemize}
  \item transition:
  \[
    \text{outputQueue}, \text{isSpeaking}, \text{lastStatus}
    := [], \text{false}, \text{Idle}
  \]
\end{itemize}

\subsubsection{Local Functions}

\noindent
We model the processing of the queue with a helper function:

\[
  \text{nextMessage} :
  \text{sequence of (String, OutputMode)} \rightarrow
  (\text{String}, \text{OutputMode}) \cup \{\text{None}\}
\]

\[
  \text{nextMessage}(q) =
  \begin{cases}
    \text{first element of } q & \text{if } q \neq []\\
    \text{None} & \text{if } q = []
  \end{cases}
\]

\noindent
BH-Feedback periodically checks outputQueue. If nextMessage returns a
pair (msg, m), it behaves as in speakNow(msg, m) and then removes that
entry from the queue. If None, it leaves the state unchanged.

\noindent
This module fulfills \textit{FUNC.R.5–R.6} by providing timely spoken and visual
feedback to the user, and by reporting a clear status that can be logged or shown
in the interface.

\section{MIS of BH-UI (Proxi Interface)} \label{mod:BHUI}

\subsection{Module}

BH-UI (Proxi Interface)

\subsection{Type}

Abstract object.

\subsection{Uses}

BH-Feedback, BH-Safety, BH-Input, SD-Log, SD-Types

\subsection{Syntax}

\subsubsection{Exported Constants}

UIState = \{\text{Idle}, \text{Listening}, \text{Waiting},
\text{Displaying}, \text{Error}\}

\subsubsection{Exported Access Programs}

\begin{center}
\begin{tabular}{p{2.7cm} p{4cm} p{4cm} p{2.5cm}}
\hline
\textbf{Name} & \textbf{Input} & \textbf{Output} & \textbf{Errors} \\
\hline
initUI & N/A & N/A & UiInitFailed \\
updateView & s: UIState & N/A & RenderError \\
showMessage & msg: String & N/A & RenderError \\
promptUser & q: String & ans: Bool & Timeout \\
showStatus & st: FeedbackStatus & N/A & N/A \\
clearScreen & N/A & N/A & N/A \\
\hline
\end{tabular}
\end{center}

\subsection{Semantics}

\subsubsection{State Variables}

\begin{itemize}
  \item uiState : UIState
  \item lastMsg : String
  \item lastError : String or null
\end{itemize}

\subsubsection{Environment Variables}

\begin{itemize}
  \item display : visual interface (screen or console)
  \item micLED : visual cue showing listening status
  \item inputChannel : keyboard or voice response channel for prompts
\end{itemize}

\subsubsection{Assumptions}

\begin{itemize}
  \item Display device is available and writable.
  \item Voice cues or LEDs can toggle quickly without delay.
  \item Text is short enough to fit within screen limits.
  \item \textbf{initUI} is called before any other access routine.
\end{itemize}

\subsubsection{Access Routine Semantics}

\noindent
\textbf{initUI}():
\begin{itemize}
  \item transition:
  \[
    \text{uiState}, \text{lastMsg}, \text{lastError}
    := \text{Idle}, "", null
  \]
  \item exception: UiInitFailed if display cannot be initialized.
\end{itemize}

\noindent
\textbf{updateView}(s):
\begin{itemize}
  \item transition: $\text{uiState} := s$ and display updates status text and
  micLED according to s.
  \item exception: RenderError if update fails.
\end{itemize}

\noindent
\textbf{showMessage}(msg):
\begin{itemize}
  \item transition: $\text{lastMsg}, \text{uiState} := msg, \text{Displaying}$.
  \item transition: display renders msg (and BH-Feedback may speak it, depending
  on OutputMode chosen by the caller).
  \item exception: RenderError if display fails.
\end{itemize}

\noindent
\textbf{promptUser}(q):
\begin{itemize}
  \item transition: $\text{uiState} := \text{Waiting}$ and display renders q.
  \item output: returns ans as a yes/no value from inputChannel.
  \item exception: Timeout if no input received in time limit.
\end{itemize}

\noindent
\textbf{showStatus}(st):
\begin{itemize}
  \item transition: display shows latest FeedbackStatus st.
\end{itemize}

\noindent
\textbf{clearScreen}():
\begin{itemize}
  \item transition: $\text{uiState}, \text{lastMsg} := \text{Idle}, ""$ and
  display clears the visual area.
\end{itemize}

\subsubsection{Local Functions}

\noindent
We define a helper that maps states to display text:

\[
  \text{stateText} : \text{UIState} \rightarrow \text{String}
\]

\[
  \text{stateText}(s) =
  \begin{cases}
    \text{``Listening...''} & \text{if } s = \text{Listening}\\
    \text{``Waiting for input...''} & \text{if } s = \text{Waiting}\\
    \text{``Processing...''} & \text{if } s = \text{Displaying}\\
    \text{``Idle''} & \text{if } s = \text{Idle}\\
    \text{``Error''} & \text{if } s = \text{Error}
  \end{cases}
\]

\noindent
This mapping helps BH-Feedback and BH-Safety show consistent notifications
through both visual and voice channels. This module fulfills \textit{FUNC.R.8}
by ensuring full hands-free interaction and clear accessibility feedback for all
system states.

\newpage

\bibliographystyle{plainnat}
\bibliography{../../../refs/References}

\newpage

\section{Appendix} \label{Appendix}

\wss{Extra information if required}

\newpage{}

\section*{Appendix --- Reflection}

\wss{Not required for CAS 741 projects}

The information in this section will be used to evaluate the team members on the
graduate attribute of Problem Analysis and Design.

\input{../../Reflection.tex}

\begin{enumerate}
  \item What went well while writing this deliverable?
  \item What pain points did you experience during this deliverable, and how
    did you resolve them?
  \item Which of your design decisions stemmed from speaking to your client(s)
  or a proxy (e.g. your peers, stakeholders, potential users)? For those that
  were not, why, and where did they come from?
  \item While creating the design doc, what parts of your other documents (e.g.
  requirements, hazard analysis, etc), it any, needed to be changed, and why?
  \item What are the limitations of your solution? Put another way, given
  unlimited resources, what could you do to make the project better?
  (LO\_ProbSolutions)
  \item Give a brief overview of other design solutions you considered. What
  are the benefits and tradeoffs of those other designs compared with the chosen
  design? From all the potential options, why did you select the documented
  design? (LO\_Explores)
\end{enumerate}

% ---------------------- REFLECTIONS (UNCHANGED CONTENT) --------------------
\section*{Amanbeer Minhas Reflection}

\begin{enumerate}

\item \textbf{What went well while writing this deliverable?}

Once I had the SRS, VnV Plan, and Hazard Analysis in place, this
design doc felt much more manageable. Breaking the system into
BH-Input, BH-Plan, BH-Safety, BH-Feedback, and BH-UI made it easier
to see how everything fit together. Courses like COMPSCI/SFWRENG
3RA3 (Requirements) helped me think in terms of clear responsibilities
and traceable requirements. My QA co-op experience also helped because
I naturally thought about how each module would be tested while I was
specifying it.

\item \textbf{What pain points did you experience during this deliverable,
and how did you resolve them?}

The main pain point was finding the right level of abstraction. At
first, I wrote the MIS in terms of specific tools and APIs, which
made the design feel tied to one implementation. After revisiting
the lecture notes and examples, I rewrote the modules to focus on
behaviour and information hiding instead. Another challenge was
formal notation for state and transitions. I drew on ideas from
SFWRENG 2DM3 (Discret Maths) to think in terms of states, events,
and transitions, then simplified that down to the most important
pieces so the document would still be readable.

\item \textbf{Which of your design decisions stemmed from speaking to your
client(s) or a proxy (e.g. peers, stakeholders, users)?}

Most of our client feedback came from our team mates parents/relatives
trying early voice-based prototypes. Some people wanted fewer confirmations
so the system felt faster, while others were worried about accidental
destructive actions. This made us directly include the BH-Safety
module: we introduced risk levels and different behaviours for low,
medium, and high risk actions. Ideas around accessibility and clear
feedback were reinforced by discussions in ENGINEER 4A03, where we talked
about inclusivity, ethics, and duty of care. Where we did not have direct
user input, I leaned on course examples and the capstone lectures for
reference.

\item \textbf{While creating the design doc, what parts of your other
documents (e.g., requirements, hazard analysis, etc), if any, needed
to be changed, and why?}

While writing the MIS, I noticed that some requirements in the SRS
were too vague to map cleanly to modules. We will refine a few of them to
have clearer success conditions (for example, accuracy targets and
timing bounds), so they matched the behaviour of BH-Input and
BH-Feedback. The Hazard Analysis also needs change: some risks originally
assigned to ``the system'' were moved specifically to BH-Safety, since
that is where confirmation and blocking actually happen. The VnV
Plan needs change to add module-level tests for voice accuracy, plan
execution success, and safety prompts, reflecting what I learned
about test design in my QA co-op and SFWRENG 3S03 (Software testing).

\item \textbf{What are the limitations of your solution? Put another way,
given unlimited resources, what could you do to make the project
better? (LO\_ProbSolutions)}

Right now, the design assumes relatively simple commands and a
limited amount of context. The NLU is closer to pattern matching
than full natural language understanding. The system also depends on
reliable speech models and may not perform as well in noisy
environments or for all accents. With more time and resources, we
would explore more robust language models, better noise handling, and
wider accessibility testing with real users. We would also add more
automation around logging and replaying user sessions for regression
testing, drawing from ideas in my QA work and SFWRENG 3S03 and
STATS 3Y03 for analyzing failure patterns.

\item \textbf{Give a brief overview of other design solutions you
considered. What are the benefits and tradeoffs of those other
designs compared with the chosen design? From all the potential
options, why did you select the documented design? (LO\_Explores)}

We considered three main designs. The first was a single monolithic
voice assistant where input, planning, and safety were all handled in
one loop. This would have been easier to code quickly but very hard
to test or change without breaking things. The second option relied
mostly on an external cloud assistant, which might give better
speech accuracy but raises privacy and reliability concerns. The
third, which we chose, is the modular design in this document, with
separate BH modules and clear interfaces. It fits the ideas from
SFWRENG 2OP3, 2C03, 3RA3, and 3XB3 about modularity, testing, and
requirements traceability. We chose it because it balances what we can
realistically implement as a student with the level of structure and
safety we have learned to aim for in software engineering.
\end{enumerate}

\section*{Ajay Singh Grewal Reflection}

\begin{enumerate}

\item \textbf{What went well while writing this deliverable?}

What went well was that we were really able to define clear modules and map them
to the requirements. This made it straightforward to write the MIS for each
module.

\item \textbf{What pain points did you experience during this deliverable,
and how did you resolve them?}

A main pain point was having everyone be consistent with their ideas of how to
write the MIS. We resolved this by having team meetings to discuss the format
and style we wanted to use, and then sharing examples to ensure everyone was
on the same page. Also, breaking down the system into smaller modules helped
make this task more easy to handle.

\item \textbf{Which of your design decisions stemmed from speaking to your
client(s) or a proxy (e.g. peers, stakeholders, users)?}

From discussions with our peers and stakeholders, we chose a voice first design
with simple visual feedback to ensure accessibility. We also decided to include a
safety module to confirm high risk actions, as some users expressed concerns
with poteintial mistakes.

\item \textbf{While creating the design doc, what parts of your other
documents (e.g., requirements, hazard analysis, etc), if any, needed
to be changed, and why?}

While creating the design doc, we realized that some requirements in the SRS
were not descriptive enough to be effectively mapped to moduleds. We roughly
altered these requirements so they are more specific and measurable.

\item \textbf{What are the limitations of your solution? Put another way,
given unlimited resources, what could you do to make the project
better? (LO\_ProbSolutions)}

The limitations of our solution is that it is designed for a single desktop
enviroment and heavlily is reliant on STT and LLM. Hence, offline use cases
and strong privacy guarantees are not well supported. Given unlimited resources,
we would explore more defined language models, and richer accessibility features
to target a broader user base.

\item \textbf{Give a brief overview of other design solutions you
considered. What are the benefits and tradeoffs of those other
designs compared with the chosen design? From all the potential
options, why did you select the documented design? (LO\_Explores)}

We had also considered a command line interface design and browser extension.
However, these designs were not as user friendly and accessible as a voice
first design. Hence, we chose the current documented design as it balances
usability, accessibility, and modularity.

\end{enumerate}

\section*{Savinay Chhabra Reflection}

\begin{enumerate}

\item \textbf{What went well while writing this deliverable?}

This deliverable was a lot smoother than the previous deliverables. We finished
a good portion of the work before the TA meeting which allowed us to get good
feedback on the deliverable before submission. We did a much better job at
dividing the work this time and were more punctual in our delivery.

\item \textbf{What pain points did you experience during this deliverable,
and how did you resolve them?}

The pain point during this deliverable was being able to define the design
without being too solution oriented. It is very easily to use specific services
as we have done a good amount of work on the POC. This implementation might
change for the final project so it's important not to pick any specific
implmentation yet. I found iteratively adding content to the deliverable and
incrementally improving different parts of the design helps as you go over a
part multiple times and may catch any mistakes that may have slipped through.
Reviewing the lecture slides from our previous courses also helped.

\item \textbf{Which of your design decisions stemmed from speaking to your
client(s) or a proxy (e.g. peers, stakeholders, users)?}

We made most of our design decisions keeping our stakeholders in mind. One
feature we added explicitly because of stakeholder feedback was the Risk Policy
feature which asks for explicit user confirmation for tasks deemed to be high
risk. This was done after multiple peers brought up their concerns about
the application accidentally making mistakes while converting speech to text.

\item \textbf{While creating the design doc, what parts of your other
documents (e.g., requirements, hazard analysis, etc), if any, needed
to be changed, and why?}

I feel we did a fairly good job of coming up with hazards as none of the risks
that were brought up in early testing or meetings were something we hadn't
already considered. However, our SRS could use some refining as some
requirements can be interpreted in a few different ways. We will create issues
to address these requirements to make them more concise and simple to understand.

\item \textbf{What are the limitations of your solution? Put another way,
given unlimited resources, what could you do to make the project
better? (LO\_ProbSolutions)}

The current limitation of our solutions is using off-the-shelf existing AI models
for our project. Given unlimited time, we would aim to create our own model as
our use-case is relatively specific and niche. Making API calls for every single
action will quickly get very expensive and will make scaling the application
challenging over time.
Our error-handling strategy is also very rudimentary at the moment; it simply
asks the user to repeat or rephrase. Smarter and more automated recovery
mechanisms could be added, given additional resources.
Adding more test cases would certainly help with the robustness of the
application but testing requires significant resources, particularly on
different platforms as our application will be supported on multiple platforms.
Adding some local processing would also be beneficial as the application in it's
currently design will not work without an active internet connection.

\item \textbf{Give a brief overview of other design solutions you
considered. What are the benefits and tradeoffs of those other
designs compared with the chosen design? From all the potential
options, why did you select the documented design? (LO\_Explores)}

One of the alternatives we considered was building a rule-based system where
voice commands would be mapped to structured templates for specific tasks and if
a task could not fit one of the templates, it would not be attempted. This makes
the system very predictable and stable but maintaining it and extending it is
very cumbersome.

Another alternative we considered was a fully machine learning driven pipeline
where intent detection, safety handling and action were all handled by a single
model. This allowed the system to be very adaptable for a wide variety of
different uses but makes it very difficult to add safeguards. Testing for this
method is also quite difficult as it is only feasible to test a certain number
of cases; and quite a few different types of cases could go untested.

We considered a plugin system as well where the user could download and install
plugins for certain applications. This would make extending the system over time
easy but added a lot of complexity to dependency management, version management
and cross plugin communication. Not to mention it would require more steps from
users to setup as well. We ended up dropping the plugin idea completely due to
these reasons.

In the end we ended up picking a balance between structure and flexibility. It
keeps components sepearate enough for extensibility while allowing us to maintain
control over the application and testing.

\end{enumerate}

\section*{Gourob Podder Reflection}

\begin{enumerate}

\item \textbf{What went well while writing this deliverable?}

This deliverable was a lot less hectic in terms of planning. We did a large
chunk of the MIS before the TA meeting which allowed us to get good feedback on
the deliverable before submission and clear up misconceptions.

\item \textbf{What pain points did you experience during this deliverable,
and how did you resolve them?}

The main paint point during this deliverable was making sure we weren't being
too specific in how we wrote the modules like including specific services, APIs
or code implementations. I primarily used the good examples defined by the prof
to understand the expectations of this section and read the relevant slides.

\item \textbf{Which of your design decisions stemmed from speaking to your
client(s) or a proxy (e.g. peers, stakeholders, users)?}

The security gate module was a direct result of feedback from our peers and
stakeholders. Many users expressed concerns about accidental destructive actions
being performed by the application due to misinterpretation of voice commands.
To address this, we incorporated a safety module that classifies actions based
on their risk levels and prompts users for confirmation when necessary.

\item \textbf{While creating the design doc, what parts of your other
documents (e.g., requirements, hazard analysis, etc), if any, needed
to be changed, and why?}

For the most part our docs were fairly consistent. However we did notice some
requirements which could reworked to be more specific and measurable to be
effectively mapped to modules.

\item \textbf{What are the limitations of your solution? Put another way,
given unlimited resources, what could you do to make the project
better? (LO\_ProbSolutions)}

In an ideal scenario we would create our own model as there are a specific class
of llms called large action models (LAMs) which are specifically designed for
action planning tasks. These models are optimized to understand and execute
complex sequences of actions based on user commands. These models are typically
smaller and more efficient than general-purpose LLMs, making them suitable for
low latency and better user experience.

\item \textbf{Give a brief overview of other design solutions you
considered. What are the benefits and tradeoffs of those other
designs compared with the chosen design? From all the potential
options, why did you select the documented design? (LO\_Explores)}

One of the alternatives we considered was a complete local system. This would
have ensured better privacy and security for users as all data would be
processed locally. However, this would have required significant computational
resources and additional technical expertise in being able to train SOTA models.
In addition, the given timeline for the capstone we would have severely limited
the features we could implement and hurt the overall user experience.

We also considered a monolothic model design where we would use a very large
multimodal LLM to handle all aspects of the application from speech to text,
intent detection, action planning and execution. While this would have
simplified the architecture of the application, it would have made it very
difficult to add safeguards and test the system effectively. The api costs for
such models are also very high and would have made scaling the application
challenging.

In the end we chose a modular design which balanced focus for usability and
accessibility while also being able to get great results by using a mixture of
models approach through our modular architecture.
\end{enumerate}

\section*{Team Module Reflection}

\begin{enumerate}

\item \textbf{After you have implemented another team's module, which means this
isn't filled in until after the original deadline). What did you learn by
implementing another team's module? Were all the details you needed in the
documentation, or did you need to make assumptions, or ask the other team
questions?  If your team also had another team implement one of your modules,
what was this experience like?  Are there things in your documentation you
could have changed to make the process go more smoothly for when an
``outsider'' completes some of the implementation?}

By implementing another team's module, we learned how dependent a correct
implementation is on clear contracts and integration details, not just the
routine list. When the implementer is not the author, any ambiguity turns
into extra time spent reading, guessing intent, or coordinating with the
other team. This also made it clear that ``small'' missing details (types,
validation rules, and ownership of state) can change the architecture of
the implementation.

For the module we implemented (Text Buffer), the other team's MIS/MG were
helpful for understanding the intended responsibilities and the core API.
In particular, the documentation clearly stated the main operations
(create, modify, move, delete), the state concept (a mapping from TextID
to TextObject), and the exceptions expected (Invalid Location, Text Not
Found). However, not all the details needed for a clean, drop-in
implementation were present. We still had to make assumptions in order to
complete it. Examples include:
\begin{itemize}
  \item What exactly counts as a valid location (finite coordinates only,
        canvas bounds, snapping, negative values, etc.).
  \item The concrete structure of \texttt{location} and \texttt{formatting}
        in code, and whether formatting is stored inside TextObject or
        handled by another part of the system.
  \item ID strategy details: whether TextID must align with other object
        IDs, whether IDs must persist through undo/redo, and how uniqueness
        is guaranteed across sessions.
  \item How this module is expected to interact with global state, because
        the docs referenced usage of geometry state, but it was unclear
        whether TextBuffer directly mutates state or reports changes for a
        mutator layer to apply.
\end{itemize}
Because these details were not fully specified, we either chose the most
reasonable interpretation (based on the MIS and the surrounding system
design), or we would need to ask the other team to confirm expectations
before final integration. In short, the documentation provided the ``what''
but not always the ``how'' needed for consistent integration.

Our team also experienced the reverse situation when another team
implemented one of our modules, the Proxi UI interface. Their feedback
made the documentation gaps very obvious. They reported that there were
no API endpoint implementations and no clear guidance on UI and backend
integration. They also noted that there was no specific documentation for
UI flows and design guidance. Finally, they pointed out an important scope
issue: UI is not just a small module, it is a full feature that is often
critical and requires continuous integration and development, which can be
beyond the scope of implementing a single module from MIS/MG alone. Even
with these challenges, they still made a best-effort implementation to
align with our MIS, MG, and SRS, and they asked us to add clarifications
or requirements through the PR thread.

This experience was valuable because it showed us what an outsider needs
most: explicit integration contracts, concrete data formats, and examples.
If we could revise our documentation to make outsider implementation go
more smoothly, we would make the following changes:
\begin{itemize}
  \item Add a clear UI-to-backend integration section, including endpoint
        list, request/response schemas, error cases, and authentication or
        state assumptions.
  \item Document UI flows explicitly: key screens, user actions, expected
        state changes, and any navigation or validation rules.
  \item Provide a small set of acceptance tests or scenarios (``given/when/
        then'' style) so outsiders can verify behaviour without guessing.
  \item Clarify module boundaries and ownership of state (what the UI must
        implement versus what depends on other subsystems), and explicitly
        list any dependencies that must exist first.
  \item Include minimal reference implementations or stubs (for example,
        mock endpoints or placeholder services) so the UI can be developed
        and tested before the backend is complete.
\end{itemize}

Overall, implementing another team's module and having another team
implement ours reinforced the same lesson: MIS/MG documentation must be
precise about assumptions, types, validation, and integration. Without
that, outsiders can still implement something, but they will need to make
assumptions, ask questions, and iterate more through PR feedback.



\end{enumerate}

\end{document}