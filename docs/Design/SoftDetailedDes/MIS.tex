\documentclass[12pt, titlepage]{article}

\usepackage{amsmath, mathtools}

\usepackage[round]{natbib}
\usepackage{amsfonts}
\usepackage{amssymb}
\usepackage{graphicx}
\usepackage{colortbl}
\usepackage{xr}
\usepackage{hyperref}
\usepackage{longtable}
\usepackage{xfrac}
\usepackage{tabularx}
\usepackage{float}
\usepackage{siunitx}
\usepackage{booktabs}
\usepackage{multirow}
\usepackage[section]{placeins}
\usepackage{caption}
\usepackage{fullpage}
\usepackage[utf8]{inputenc}


\hypersetup{
bookmarks=true,     % show bookmarks bar?
colorlinks=true,       % false: boxed links; true: colored links
linkcolor=red,          % color of internal links (change box color with linkbordercolor)
citecolor=blue,      % color of links to bibliography
filecolor=magenta,  % color of file links
urlcolor=cyan          % color of external links
}

\usepackage{array}

\externaldocument{../../SRS/SRS}

\input{../../Comments}
%% Common Parts

\newcommand{\progname}{Software Engineering} % PUT YOUR PROGRAM NAME HERE
\newcommand{\authname}{Team \#23, Project Proxi
\\ Savinay Chhabra
\\ Amanbeer Singh Minhas
\\ Gourob Podder
\\ Ajay Grewal} % AUTHOR NAMES                  

\usepackage{hyperref}
    \hypersetup{colorlinks=true, linkcolor=blue, citecolor=blue, filecolor=blue,
                urlcolor=blue, unicode=false}
    \urlstyle{same}
                                


\begin{document}

\title{Module Interface Specification for \progname{}}

\author{\authname}

\date{\today}

\maketitle

\pagenumbering{roman}

\section{Revision History}

\begin{tabularx}{\textwidth}{p{3cm}p{2cm}X}
\toprule {\bf Date} & {\bf Version} & {\bf Notes}\\
\midrule
Date 1 & 1.0 & Notes\\
Date 2 & 1.1 & Notes\\
\bottomrule
\end{tabularx}

~\newpage

\section{Symbols, Abbreviations and Acronyms}

See SRS Documentation at \wss{give url}

\wss{Also add any additional symbols, abbreviations or acronyms}

\newpage

\tableofcontents

\newpage

\pagenumbering{arabic}

\section{Introduction}

The following document details the Module Interface Specifications for
\wss{Fill in your project name and description}

Complementary documents include the System Requirement Specifications
and Module Guide.  The full documentation and implementation can be
found at \url{...}.  \wss{provide the url for your repo}

\section{Notation}

\wss{You should describe your notation.  You can use what is below as
  a starting point.}

The structure of the MIS for modules comes from \citet{HoffmanAndStrooper1995},
with the addition that template modules have been adapted from
\cite{GhezziEtAl2003}.  The mathematical notation comes from Chapter 3 of
\citet{HoffmanAndStrooper1995}.  For instance, the symbol := is used for a
multiple assignment statement and conditional rules follow the form $(c_1
\Rightarrow r_1 | c_2 \Rightarrow r_2 | ... | c_n \Rightarrow r_n )$.

The following table summarizes the primitive data types used by \progname. 

\begin{center}
\renewcommand{\arraystretch}{1.2}
\noindent 
\begin{tabular}{l l p{7.5cm}} 
\toprule 
\textbf{Data Type} & \textbf{Notation} & \textbf{Description}\\ 
\midrule
character & char & a single symbol or digit\\
integer & $\mathbb{Z}$ & a number without a fractional component in (-$\infty$, $\infty$) \\
natural number & $\mathbb{N}$ & a number without a fractional component in [1, $\infty$) \\
real & $\mathbb{R}$ & any number in (-$\infty$, $\infty$)\\
\bottomrule
\end{tabular} 
\end{center}

\noindent
The specification of \progname \ uses some derived data types: sequences, strings, and
tuples. Sequences are lists filled with elements of the same data type. Strings
are sequences of characters. Tuples contain a list of values, potentially of
different types. In addition, \progname \ uses functions, which
are defined by the data types of their inputs and outputs. Local functions are
described by giving their type signature followed by their specification.

\section{Module Decomposition}

The \textbf{Proxi}, is decomposed into a hierarchy of modules following the 
design principles of information hiding and separation of concerns. Each 
module corresponds to a well-defined secret and is independently assignable 
to a developer. The decomposition balances hardware hiding, behaviour 
hiding, and software decision modules.

\begin{table}[H]
\centering
\renewcommand{\arraystretch}{1.2}
\begin{tabular}{p{0.3\textwidth} p{0.65\textwidth}}
\toprule
\textbf{Level 1} & \textbf{Level 2 Modules (Secrets / Responsibilities)}\\
\midrule

\multirow{2}{*}{\textbf{Hardware-Hiding}} &
\textbf{HH-IO (Audio Adapter)} – manages microphone input and audio 
output across different platforms.\\
& \textbf{HH-Auto (System Control)} – performs basic desktop actions 
such as typing, clicking, or launching applications.\\
\midrule

\multirow{7}{*}{\textbf{Behaviour-Hiding}} &
\textbf{BH-Input (Voice \& Text Manager)} – captures user input, 
converts speech to text, and normalizes text commands. Implements 
\textit{FUNC.R.1–R.2}.\\
& \textbf{BH-NLU (Intent Parser)} – interprets text into structured 
intents and parameters based on defined command patterns. Implements 
\textit{FUNC.R.3}.\\
& \textbf{BH-Plan (Task Executor)} – determines which agent or tool 
should handle a command and coordinates its execution. Implements 
\textit{FUNC.R.4}.\\
& \textbf{BH-Safety (Confirmation Gate)} – validates actions that may 
affect files or system settings and requests confirmation. Implements 
\textit{FUNC.R.9}.\\
& \textbf{BH-Session (Context Manager)} – maintains user session data, 
history, and undo information for continuity. Supports 
\textit{FUNC.R.4}.\\
& \textbf{BH-Feedback (Response Manager)} – converts textual responses 
into spoken or visual feedback for the user. Implements 
\textit{FUNC.R.5–R.6}.\\
& \textbf{BH-UI (Proxi Interface)} – presents status updates, 
confirmations, and results; supports full voice-only interaction. 
Implements \textit{FUNC.R.8}.\\
\midrule

\multirow{5}{*}{\textbf{Software-Decision}} &
\textbf{SD-Types (Core Structures)} – defines abstract data types for 
Command, Intent, and Plan.\\
& \textbf{SD-ToolRegistry (Action Map)} – maintains the mapping between 
recognized intents and available system actions.\\
& \textbf{SD-Store (Local Storage)} – handles persistent storage for 
user preferences, session history, and logs.\\
& \textbf{SD-STT/TTS Config} – specifies configuration for speech and 
text synthesis models and supported languages.\\
& \textbf{SD-Log (Event Logger)} – records system actions and feedback 
events for debugging and validation.\\
\bottomrule
\end{tabular}
\caption{Module Hierarchy for Proxi Voice Assistant}
\label{TblMH}
\end{table}


\textbf{Likely Changes:}
\begin{itemize}
  \item The choice of speech recognition or text-to-speech library 
  (for example, switching from Whisper API to a local model).
  \item Adjustments to the user interface layout or how voice commands 
  trigger visible feedback or audio playback.
  \item Fine-tuning thresholds for speech detection and timing between 
  input and response based on user testing.
  \item Updating supported voice commands or adding new MCP tools as 
  features are expanded.
\end{itemize}

\textbf{Unlikely Changes:}
\begin{itemize}
  \item The main processing loop of Input $\rightarrow$ Interpret 
  $\rightarrow$ Plan $\rightarrow$ Execute $\rightarrow$ Feedback.
  \item The core data structures used for storing Commands, Intents, 
  and Action Plans.
  \item The communication pattern between modules through the MCP 
  agent interface.
\end{itemize}

\textbf{Traceability to SRS:}
\begin{itemize}
  \item \textbf{BH-Input} fulfills \textit{FUNC.R.1–R.2}: speech and text 
  input handling with accuracy $\geq 90\%$.
  \item \textbf{BH-NLU} fulfills \textit{FUNC.R.3}: intent recognition 
  accuracy $\geq 90\%$.
  \item \textbf{BH-Plan} fulfills \textit{FUNC.R.4}: agent planning and 
  execution success rate $\geq 85\%$.
  \item \textbf{BH-Feedback} fulfills \textit{FUNC.R.5–R.6}: provides 
  feedback and spoken confirmation within response time $\leq 2\,\text{s}$.
  \item \textbf{BH-UI} fulfills \textit{FUNC.R.8}: supports full hands-free 
  operation for accessibility.
  \item \textbf{BH-Safety} fulfills \textit{FUNC.R.9}: requests confirmation 
  before executing high-risk or destructive actions.
  \item \textbf{Support modules (SD, HH)} enable non-functional goals on 
  latency, reliability, and auditability through structured logging.
\end{itemize}

\newpage

\section{MIS of \wss{Module Name}} \label{Module} \wss{Use labels for
  cross-referencing}

\wss{You can reference SRS labels, such as R\ref{R_Inputs}.}

\wss{It is also possible to use \LaTeX for hypperlinks to external documents.}

\subsection{Module}

\wss{Short name for the module}

\subsection{Uses}


\subsection{Syntax}

\subsubsection{Exported Constants}

\subsubsection{Exported Access Programs}

\begin{center}
\begin{tabular}{p{2cm} p{4cm} p{4cm} p{2cm}}
\hline
\textbf{Name} & \textbf{In} & \textbf{Out} & \textbf{Exceptions} \\
\hline
\wss{accessProg} & - & - & - \\
\hline
\end{tabular}
\end{center}

\subsection{Semantics}

\subsubsection{State Variables}

\wss{Not all modules will have state variables.  State variables give the module
  a memory.}

\subsubsection{Environment Variables}

\wss{This section is not necessary for all modules.  Its purpose is to capture
  when the module has external interaction with the environment, such as for a
  device driver, screen interface, keyboard, file, etc.}

\subsubsection{Assumptions}

\wss{Try to minimize assumptions and anticipate programmer errors via
  exceptions, but for practical purposes assumptions are sometimes appropriate.}

\subsubsection{Access Routine Semantics}

\noindent \wss{accessProg}():
\begin{itemize}
\item transition: \wss{if appropriate} 
\item output: \wss{if appropriate} 
\item exception: \wss{if appropriate} 
\end{itemize}

\wss{A module without environment variables or state variables is unlikely to
  have a state transition.  In this case a state transition can only occur if
  the module is changing the state of another module.}

\wss{Modules rarely have both a transition and an output.  In most cases you
  will have one or the other.}

\subsubsection{Local Functions}

\wss{As appropriate} \wss{These functions are for the purpose of specification.
  They are not necessarily something that is going to be implemented
  explicitly.  Even if they are implemented, they are not exported; they only
  have local scope.}

\newpage

\bibliographystyle {plainnat}
\bibliography {../../../refs/References}

\newpage

\section{Appendix} \label{Appendix}

\wss{Extra information if required}

\newpage{}

\section*{Appendix --- Reflection}

\wss{Not required for CAS 741 projects}

The information in this section will be used to evaluate the team members on the
graduate attribute of Problem Analysis and Design.

\input{../../Reflection.tex}

\begin{enumerate}
  \item What went well while writing this deliverable? 
  \item What pain points did you experience during this deliverable, and how
    did you resolve them?
  \item Which of your design decisions stemmed from speaking to your client(s)
  or a proxy (e.g. your peers, stakeholders, potential users)? For those that
  were not, why, and where did they come from?
  \item While creating the design doc, what parts of your other documents (e.g.
  requirements, hazard analysis, etc), it any, needed to be changed, and why?
  \item What are the limitations of your solution?  Put another way, given
  unlimited resources, what could you do to make the project better? (LO\_ProbSolutions)
  \item Give a brief overview of other design solutions you considered.  What
  are the benefits and tradeoffs of those other designs compared with the chosen
  design?  From all the potential options, why did you select the documented design?
  (LO\_Explores)
\end{enumerate}


\end{document}