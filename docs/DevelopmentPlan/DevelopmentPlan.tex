\documentclass{article}

\usepackage{booktabs}
\usepackage{tabularx}
\usepackage{makecell}

\title{Development Plan\\\progname}

\author{\authname}

\date{}

\input{../Comments}
%% Common Parts

\newcommand{\progname}{Software Engineering} % PUT YOUR PROGRAM NAME HERE
\newcommand{\authname}{Team \#23, Project Proxi
\\ Savinay Chhabra
\\ Amanbeer Singh Minhas
\\ Gourob Podder
\\ Ajay Grewal} % AUTHOR NAMES                  

\usepackage{hyperref}
    \hypersetup{colorlinks=true, linkcolor=blue, citecolor=blue, filecolor=blue,
                urlcolor=blue, unicode=false}
    \urlstyle{same}
                                


\begin{document}

\maketitle

\begin{table}[hp]
\caption{Revision History} \label{TblRevisionHistory}
\begin{tabularx}{\textwidth}{llX}
\toprule
\textbf{Date} & \textbf{Developer(s)} & \textbf{Change}\\
\midrule
September 21, 2025 & Savinay Chhabra  & Created Draft Development Plan; Sections 1, 2, 3, 4, 5, 7\\
September 22, 2025 & Savinay Chhabra  & Sections 6, 8\\ September 22, 2025 & Ajay Grewal & Section 9, 10, 11\\
\bottomrule
\end{tabularx}
\end{table}

\newpage{}

This report outlines the development plan for the Aura Voice AI Assistant. 
The document details the following sections of the Development Plan:
\begin{itemize}
  \item Confidentiality Information
  \item IP Protection
  \item Copyright License
  \item Team Meeting Plans
  \item Team Communication Plans
  \item Team Member Roles
  \item Workflow Plan
  \item Project Decomposition and Scheduling
  \item Proof of Concept Demo Plan
  \item Expected Technology
  \item Coding Standard
\end{itemize}

\section{Confidential Information?}

There is no confidential information to protect.

\section{IP to Protect}

There is no IP to protect.

\section{Copyright License}

This project adopts the MIT License. LICENSE file can be found \href{https://github.com/gkpodder/Capstone/blob/main/LICENSE}
{here}.

\section{Team Meeting Plan}
The Team will meet every Wednesday from 4:00PM - 4:45PM in person. Campus library meeting rooms will be booked weekly, depending on availability.
\begin{itemize}
  \item A rotating chair will lead each meeting and prepare the meeting\'s agenda in advance.
  \item A rotating notetaker will record meeting minutes, action items and decisions made in the meeting.
  \item Feature requests and Decisions will be documented using Github Issues
  \item Meetings with graduate students speacializing in Human-Computer Interfaces will occur biweekly. These will either be virtual or in-person depending on availability.
\end{itemize}

\section{Team Communication Plan}

Urgent issues or clarifications will be communicated through Discord. Discord will also be used for any virtual meetings between the team. 
Meetings with industry advisors and graduate students will take place over MS Teams if held virtually. 
Github issues will the the source of truth for techincal information and progress tracking.

\section{Team Member Roles}

\begin{table}[!h]
\caption{Team Member Roles} \label{TblMemberRoles}
\begin{tabularx}{\textwidth}{lXl}
\toprule
\textbf{Role} & \textbf{Responsibilities} & \textbf{Member(s}\\
\midrule
Meeting Chair & Sets meeting agenda and leads weekly meetings & Rotates weekly\\
Note Taker & Records meeting minutes and decisions & Rotates weekly\\
Team Liaison & Primary point of contact for project supervision and industry advisors & Savinay Chhabra \\
Subject Matter Expert & Develops deep knowledge of the project\'s domain & Gourob Podder \\
Test Lead & Responsible for planning, organizing and overseeing testing process. & Amanbeer Singh Minhas \\
Reviewer & Responsible for evaluating Pull Requests to ensure that they adher to team standards. & Ajay Grewal \\
\bottomrule
\end{tabularx}
\end{table}

\section{Workflow Plan}

\begin{itemize}
  \item Each Feature/Bug/Defect will have a Github Issue created for it. This issue will included detailed information along with current progress, assignee and appropriate labels. Github Projects will be used to track all incoming and in-progress issues.
	\item Each Issue will a branch for any associated development work to be done.
	\item Once the author is satisfied with their solution, they can put up a Pull Request to merge their changes into the main branch. Each Pull request must link to one or more issues and have a detailed description of the changes made and the user impact of those changes.
	\item After the Pull Request has been raised and all required pre-submit checks have succeded, the author can add the "Review Needed" label to their PR.
	\item Each PR must have at least 2 reviewers. After the reviewers have approved, the PR can be merged and the associated issues can be closed.
\end{itemize}

\section{Project Decomposition and Scheduling}

The team will use the Kanban Board on Github Projects to track incoming and in-progress issues.
The Board for the project can be found \href{https://github.com/users/gkpodder/projects/2}{here}.


\begin{table}[!h]
\caption{Project Scheduling} \label{ProjScheduling}
\begin{tabularx}{\textwidth}{lXl}
\toprule
\textbf{Deliverable} & \textbf{Deadline} \\
\midrule
Problem Statement, POC Plan, Development Plan & September 22, 2025\\
Req. Doc. and Hazard Analysis Revision 0 & October 6, 2025 \\
V\&V Plan Revision 0 & October 27, 2025 \\
Design Document Revision -1 & November 10, 2025 \\
Design Document Revision 0 & January 19, 2026 \\
Revision 0 Demonstration &  February 13, 2026\\
V\&V Report and Extras Revision 0 & March 09, 2026 \\
Final Demonstration (Revision 1)  &  March 29, 2026 \\
EXPO Demonstration & TBD \\
Final Documentation (Revision 1) & April 06, 2026 \\
\bottomrule
\end{tabularx}
\end{table}

\section{Proof of Concept Demonstration Plan}

What is the main risk, or risks, for the success of your project?  What will you
demonstrate during your proof of concept demonstration to convince yourself that
you will be able to overcome this risk?

The main risks revolve around security, adaptability, and user trust, since the AI will have access and control to one’s 
computer through model context protocols, functioning as the central master control program. 

Granting the AI full control of the computer introduces major security risks and vulnerabilities. Since the MCP will have 
deep access to a user’s computer, including files, settings, and applications, any vulnerabilities could lead to catastrophic 
consequences. Malicious actors could exploit the system to steal sensitive data, install harmful software, or commit unintended 
actions which could become destructive. This risk must be reduced by having safeguards the AI must adhere to, a thoroughly tested 
environment, user permission before completing risky actions, and the ability to rollback changes. Hence, a robust error handling 
and recovery system is needed so incorrect commands do not disrupt any workflows or overall user experience.

Another big risk is the AI being capable enough to adapt to diverse range of computer environments. Every computer is different, 
with certain operating systems, setups, workflows, and personal preferences. If the AI cannot accurately understand its environment, 
it can lead to incorrect commands, actions, and overall poor experience. The AI must be capable of learning its environment in a 
dynamic manner and customize its actions accordingly. 

Furthermore, since the AI has full control of one’s computer, transparency is essential. If users feel uncertain of how the 
AI is doing things on their computer, it can risk user trust, leading to hesitance in using the tool. Creating a clear and 
intuitive user interface, where the technology is properly explained, will help mitigate this risk. 

To address these risks, the PoC will focus on showing that the MCP and AI can safely and effectively interact with a computer, 
learn its environment, and execute meaningful tasks while keeping user trust. 


\section{Expected Technology}

\begin{itemize}
  \item Programming Language: Python for the AI agent, tool plugins, and overall backend structure. TypeScript and React for desktop UI
  \item Agent Core Libraries: pydantic, asyncio, httpx
  \item System Automation: For Windows we will use pywin32, UIAutomationCore; for MacOS we will use pyobjc, some AppleScript/JXA bridge; For Browser we will use Playwright
  \item Storage/Database: SQL
  \item Pre-trained Models: OpenAI GPT-4o-mini API for natural language understanding and generation
  \item Local LLM For Future (if needed)
  \item Linter Tools: pylint, Prettier, ESLint
  \item CI/CD: Git, Github
\end{itemize}
\section{Coding Standard}

The team will strictly adhere to and follow the following coding standards to ensure clarity, maintainablity and readability:
\begin{itemize}
  \item Python code will follow PEP 8 guidelines, using pylint for linting.
  \item TypeScript code will follow the Airbnb style guide, using Prettier for formatting and ESLint for linting.
  \item Git and GitHub will be used for version control, with feature branching and pull requests for code reviews.
  \item Unit tests, integration tests, and exploratory tests will be performed to ensure code works before merging.
\end{itemize}

\newpage{}

\section*{Appendix --- Reflection}

\wss{Not required for CAS 741}

\input{../Reflection.tex}

\begin{enumerate}
    \item Why is it important to create a development plan prior to starting the
    project?
    \\Having a development plan was crucial for our team because it gave us a shared understanding
    of our goals and responsibilities. I (Aman) realized early on that without a plan, we could 
    easily duplicate work or miss deadlines, like in a previous project where we jumped straight 
    into coding and ended up wasting time fixing miscommunication. Ajay mentioned that in his co-op, 
    projects without clear roles caused constant confusion, so he emphasized setting up structured 
    responsibilities and rotating roles and meeting lead. Savinay shared that documenting meeting agendas 
    and communication methods helped him stay accountable he had seen other projects waste time on 
    unstructured meetings. Gourob, drawing from his experience in AI-focused work, pointed out that 
    a well-documented technical workflow prevents misunderstandings about modules like speech recognition 
    or NLP if we intend to use those to achieve our goals. Overall, the plan helped us coordinate efficiently, 
    understand dependencies, and feel confident moving forward.
    \item In your opinion, what are the advantages and disadvantages of using
    CI/CD?
    \\ Our team had a lot of discussion about CI/CD and its implications. Savinay thought the biggest advantage 
    is catching integration issues early; he recalled a project where manual merges led to significant delays, 
    which could have been avoided with automated pipelines. I (Aman) agreed, but also noted that over-reliance 
    on automated tests can be frustrating when debugging, especially if tests fail unexpectedly as I experienced 
    in a small personal project. Gourob, who has worked with AI pipelines before, emphasized that CI/CD ensures 
    multiple interdependent modules, like speech recognition and text-to-speech, remain stable even as different 
    developers make frequent changes. Ajay raised the practical concern that setting up CI/CD initially is 
    time consuming and misconfigurations can be discouraging, but he concluded that the long-term benefits, 
    such as reliability and reduced errors, outweigh these challenges. Collectively, we agreed that CI/CD will help 
    us maintain a high-quality codebase, streamline collaboration, and catch potential problems early before they escalate.
    \item What disagreements did your group have in this deliverable, if any,
    and how did you resolve them?
    \\Our team had a few meaningful disagreements while preparing this deliverable, mainly around the level of detail
    for our workflow and how to structure GitHub issues. Gourob argued for very detailed issue tracking and branching
    rules because in his previous co-op, unclear workflows had caused confusion and delayed delivery. I (Aman) thought 
    a simpler approach would be faster and less bureaucratic, based on a personal experience where over planning slowed 
    down a small project unnecessarily. Ajay was concerned that if we over complicated the workflow, weekly meetings would 
    become too long and tedious. Savinay suggested a compromise: we would document the main workflow in the development 
    plan but leave flexibility within GitHub issues for the finer technical details.The resolution involved more than 
    just compromise it was about understanding each other’s reasoning. We each shared past experiences, explained why 
    we preferred our approach, and considered potential consequences of both over-planning and under planning. In the 
    end, we created a workflow that balanced structure and flexibility. This disagreement actually strengthened our 
    team dynamic because it forced us to communicate openly, respect different perspectives, and make decisions based 
    on practical experience rather than assumptions.
\end{enumerate}

\newpage{}

\section*{Appendix --- Team Charter}

\wss{borrows from
\href{https://engineering.up.edu/industry_partnerships/files/team-charter.pdf}
{University of Portland Team Charter}}

\subsection*{External Goals}

\wss{What are your team\'s external goals for this project? These are not the
goals related to the functionality or quality fo the project.  These are the
goals on what the team wishes to achieve with the project.  Potential goals are
to win a prize at the Capstone EXPO, or to have something to talk about in
interviews, or to get an A+, etc.}

\subsection*{Attendance}

\subsubsection*{Expectations}

\wss{What are your team\'s expectations regarding meeting attendance (being on
time, leaving early, missing meetings, etc.)?}

\subsubsection*{Acceptable Excuse}

\wss{What constitutes an acceptable excuse for missing a meeting or a deadline?
What types of excuses will not be considered acceptable?}

\subsubsection*{In Case of Emergency}

\wss{What process will team members follow if they have an emergency and cannot
attend a team meeting or complete their individual work promised for a team
deliverable?}

\subsection*{Accountability and Teamwork}

\subsubsection*{Quality} 

\wss{What are your team\'s expectations regarding the quality
of team members\' preparation for team meetings and the quality of the
deliverables that members bring to the team?}

\subsubsection*{Attitude}

\wss{What are your team\'s expectations regarding team members\' ideas,
interactions with the team, cooperation, attitudes, and anything else regarding
team member contributions?  Do you want to introduce a code of conduct?  Do you
want a conflict resolution plan?  Can adopt existing codes of conduct.}

\subsubsection*{Stay on Track}

\wss{What methods will be used to keep the team on track? How will your team
ensure that members contribute as expected to the team and that the team
performs as expected? How will your team reward members who do well and manage
members whose performance is below expectations?  What are the consequences for
someone not contributing their fair share?}

\wss{You may wish to use the project management metrics collected for the TA and
instructor for this.}

\wss{You can set target metrics for attendance, commits, etc.  What are the
consequences if someone doesn\'t hit their targets?  Do they need to bring the
coffee to the next team meeting?  Does the team need to make an appointment with
their TA, or the instructor?  Are there incentives for reaching targets early?}

\subsubsection*{Team Building}

\wss{How will you build team cohesion (fun time, group rituals, etc.)? }

\subsubsection*{Decision Making} 

\wss{How will you make decisions in your group? Consensus?  Vote? How will you
handle disagreements? }

\end{document}