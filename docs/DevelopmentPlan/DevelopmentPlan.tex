\documentclass{article}

\usepackage{booktabs}
\usepackage{tabularx}
\usepackage{makecell}

\title{Development Plan\\\progname}

\author{\authname}

\date{}

\input{../Comments}
%% Common Parts

\newcommand{\progname}{Software Engineering} % PUT YOUR PROGRAM NAME HERE
\newcommand{\authname}{Team \#23, Project Proxi
\\ Savinay Chhabra
\\ Amanbeer Singh Minhas
\\ Gourob Podder
\\ Ajay Grewal} % AUTHOR NAMES                  

\usepackage{hyperref}
    \hypersetup{colorlinks=true, linkcolor=blue, citecolor=blue, filecolor=blue,
                urlcolor=blue, unicode=false}
    \urlstyle{same}
                                


\begin{document}

\maketitle

\begin{table}[hp]
\caption{Revision History} \label{TblRevisionHistory}
\begin{tabularx}{\textwidth}{llX}
\toprule
\textbf{Date} & \textbf{Developer(s)} & \textbf{Change}\\
\midrule
September 21, 2025 & Savinay Chhabra  & Created Draft Development Plan; Sections 1, 2, 3, 4, 5, 7\\
September 22, 2025 & Savinay Chhabra  & Sections 6, 8\\ September 22, 2025 & Ajay Singh Grewal & Section 9, 10, 11\\
September 22, 2025 & Amanbeer Singh Minhas & Team Charter and Reflection\\
\bottomrule
\end{tabularx}
\end{table}

\newpage{}
This report outlines the development plan for the Aura Voice AI Assistant. 
The document details the following sections of the Development Plan:
\begin{itemize}
  \item Confidentiality Information
  \item IP Protection
  \item Copyright License
  \item Team Meeting Plans
  \item Team Communication Plans
  \item Team Member Roles
  \item Workflow Plan
  \item Project Decomposition and Scheduling
  \item Proof of Concept Demo Plan
  \item Expected Technology
  \item Coding Standard
\end{itemize}

\section{Confidential Information?}

There is no confidential information to protect.

\section{IP to Protect}

There is no IP to protect.

\section{Copyright License}

This project adopts the MIT License. LICENSE file can be found \href{https://github.com/gkpodder/Capstone/blob/main/LICENSE}
{here}.

\section{Team Meeting Plan}
The Team will meet every Wednesday from 4:00PM - 4:45PM in person. Campus library meeting rooms will be booked weekly, depending on availability.
\begin{itemize}
  \item A rotating chair will lead each meeting and prepare the meeting\'s agenda in advance.
  \item A rotating notetaker will record meeting minutes, action items and decisions made in the meeting.
  \item Feature requests and Decisions will be documented using Github Issues
  \item Meetings with graduate students speacializing in Human-Computer Interfaces will occur biweekly. These will either be virtual or in-person depending on availability.
\end{itemize}

\section{Team Communication Plan}

Urgent issues or clarifications will be communicated through Discord. Discord will also be used for any virtual meetings between the team. 
Meetings with industry advisors and graduate students will take place over MS Teams if held virtually. 
Github issues will the the source of truth for techincal information and progress tracking.

\section{Team Member Roles}

\begin{table}[!h]
\caption{Team Member Roles} \label{TblMemberRoles}
\begin{tabularx}{\textwidth}{lXl}
\toprule
\textbf{Role} & \textbf{Responsibilities} & \textbf{Member(s}\\
\midrule
Meeting Chair & Sets meeting agenda and leads weekly meetings & Rotates weekly\\
Note Taker & Records meeting minutes and decisions & Rotates weekly\\
Team Liaison & Primary point of contact for project supervision and industry advisors & Savinay Chhabra \\
Subject Matter Expert & Develops deep knowledge of the project\'s domain & Gourob Podder \\
Test Lead & Responsible for planning, organizing and overseeing testing process. & Amanbeer Singh Minhas \\
Reviewer & Responsible for evaluating Pull Requests to ensure that they adher to team standards. & Ajay Singh Grewal \\
\bottomrule
\end{tabularx}
\end{table}

\section{Workflow Plan}

\begin{itemize}
  \item Each Feature/Bug/Defect will have a Github Issue created for it. This issue will included detailed information along with current progress, assignee and appropriate labels. Github Projects will be used to track all incoming and in-progress issues.
	\item Each Issue will a branch for any associated development work to be done.
	\item Once the author is satisfied with their solution, they can put up a Pull Request to merge their changes into the main branch. Each Pull request must link to one or more issues and have a detailed description of the changes made and the user impact of those changes.
	\item After the Pull Request has been raised and all required pre-submit checks have succeded, the author can add the "Review Needed" label to their PR.
	\item Each PR must have at least 2 reviewers. After the reviewers have approved, the PR can be merged and the associated issues can be closed.
\end{itemize}

\section{Project Decomposition and Scheduling}

The team will use the Kanban Board on Github Projects to track incoming and in-progress issues.
The Board for the project can be found \href{https://github.com/users/gkpodder/projects/2}{here}.


\begin{table}[!h]
\caption{Project Scheduling} \label{ProjScheduling}
\begin{tabularx}{\textwidth}{lXl}
\toprule
\textbf{Deliverable} & \textbf{Deadline} \\
\midrule
Problem Statement, POC Plan, Development Plan & September 22, 2025\\
Req. Doc. and Hazard Analysis Revision 0 & October 6, 2025 \\
V\&V Plan Revision 0 & October 27, 2025 \\
Design Document Revision -1 & November 10, 2025 \\
Design Document Revision 0 & January 19, 2026 \\
Revision 0 Demonstration &  February 13, 2026\\
V\&V Report and Extras Revision 0 & March 09, 2026 \\
Final Demonstration (Revision 1)  &  March 29, 2026 \\
EXPO Demonstration & TBD \\
Final Documentation (Revision 1) & April 06, 2026 \\
\bottomrule
\end{tabularx}
\end{table}

\section{Proof of Concept Demonstration Plan}

What is the main risk, or risks, for the success of your project?  What will you
demonstrate during your proof of concept demonstration to convince yourself that
you will be able to overcome this risk?

The main risks revolve around security, adaptability, and user trust, since the AI will have access and control to one’s 
computer through model context protocols, functioning as the central master control program. 

Granting the AI full control of the computer introduces major security risks and vulnerabilities. Since the MCP will have 
deep access to a user’s computer, including files, settings, and applications, any vulnerabilities could lead to catastrophic 
consequences. Malicious actors could exploit the system to steal sensitive data, install harmful software, or commit unintended 
actions which could become destructive. This risk must be reduced by having safeguards the AI must adhere to, a thoroughly tested 
environment, user permission before completing risky actions, and the ability to rollback changes. Hence, a robust error handling 
and recovery system is needed so incorrect commands do not disrupt any workflows or overall user experience.

Another big risk is the AI being capable enough to adapt to diverse range of computer environments. Every computer is different, 
with certain operating systems, setups, workflows, and personal preferences. If the AI cannot accurately understand its environment, 
it can lead to incorrect commands, actions, and overall poor experience. The AI must be capable of learning its environment in a 
dynamic manner and customize its actions accordingly. 

Furthermore, since the AI has full control of one’s computer, transparency is essential. If users feel uncertain of how the 
AI is doing things on their computer, it can risk user trust, leading to hesitance in using the tool. Creating a clear and 
intuitive user interface, where the technology is properly explained, will help mitigate this risk. 

To address these risks, the PoC will focus on showing that the MCP and AI can safely and effectively interact with a computer, 
learn its environment, and execute meaningful tasks while keeping user trust. 


\section{Expected Technology}

\begin{itemize}
  \item Programming Language: Python for the AI agent, tool plugins, and overall backend structure. TypeScript and React for desktop UI
  \item Agent Core Libraries: pydantic, asyncio, httpx
  \item System Automation: For Windows we will use pywin32, UIAutomationCore; for MacOS we will use pyobjc, some AppleScript/JXA bridge; For Browser we will use Playwright
  \item Storage/Database: SQL
  \item Pre-trained Models: OpenAI GPT-4o-mini API for natural language understanding and generation
  \item Local LLM For Future (if needed)
  \item Linter Tools: pylint, Prettier, ESLint
  \item CI/CD: Git, Github
\end{itemize}
\section{Coding Standard}

The team will strictly adhere to and follow the following coding standards to ensure clarity, maintainablity and readability:
\begin{itemize}
  \item Python code will follow PEP 8 guidelines, using pylint for linting.
  \item TypeScript code will follow the Airbnb style guide, using Prettier for formatting and ESLint for linting.
  \item Git and GitHub will be used for version control, with feature branching and pull requests for code reviews.
  \item Unit tests, integration tests, and exploratory tests will be performed to ensure code works before merging.
\end{itemize}

\newpage{}

\section*{Appendix --- Reflection}

\begin{enumerate}
  \item \textbf{Why is it important to create a development plan prior to starting the
  project?}
  \\Having a development plan was crucial for our team because it gave us a shared understanding
  of our goals and responsibilities. I (Aman) realized early on that without a plan, we could 
  easily duplicate work or miss deadlines, like in a previous project where we jumped straight 
  into coding and ended up wasting time fixing miscommunication. Ajay mentioned that in his co-op, 
  projects without clear roles caused constant confusion, so he emphasized setting up structured 
  responsibilities and rotating roles and meeting lead. Savinay shared that documenting meeting agendas 
  and communication methods helped him stay accountable he had seen other projects waste time on 
  unstructured meetings. Gourob, drawing from his experience in AI-focused work, pointed out that 
  a well-documented technical workflow prevents misunderstandings about modules like speech recognition 
  or NLP if we intend to use those to achieve our goals. Overall, the plan helped us coordinate efficiently, 
  understand dependencies, and feel confident moving forward.
  \item \textbf{In your opinion, what are the advantages and disadvantages of using
  CI/CD?}
  \\Our team had a lot of discussion about CI/CD and its implications. Savinay thought the biggest advantage 
  is catching integration issues early; he recalled a project where manual merges led to significant delays, 
  which could have been avoided with automated pipelines. I (Aman) agreed, but also noted that over-reliance 
  on automated tests can be frustrating when debugging, especially if tests fail unexpectedly as I experienced 
  in a small personal project. Gourob, who has worked with AI pipelines before, emphasized that CI/CD ensures 
  multiple interdependent modules, like speech recognition and text-to-speech, remain stable even as different 
  developers make frequent changes. Ajay raised the practical concern that setting up CI/CD initially is 
  time consuming and misconfigurations can be discouraging, but he concluded that the long-term benefits, 
  such as reliability and reduced errors, outweigh these challenges. Collectively, we agreed that CI/CD will help 
  us maintain a high-quality codebase, streamline collaboration, and catch potential problems early before they escalate.
  \item \textbf{What disagreements did your group have in this deliverable, if any,
  and how did you resolve them?}
  \\Our team had a few meaningful disagreements while preparing this deliverable, mainly around the level of detail
  for our workflow and how to structure GitHub issues. Gourob argued for very detailed issue tracking and branching
  rules because in his previous co-op, unclear workflows had caused confusion and delayed delivery. I (Aman) thought 
  a simpler approach would be faster and less bureaucratic, based on a personal experience where over planning slowed 
  down a small project unnecessarily. Ajay was concerned that if we over complicated the workflow, weekly meetings would 
  become too long and tedious. Savinay suggested a compromise: we would document the main workflow in the development 
  plan but leave flexibility within GitHub issues for the finer technical details.The resolution involved more than 
  just compromise it was about understanding each other’s reasoning. We each shared past experiences, explained why 
  we preferred our approach, and considered potential consequences of both over-planning and under planning. In the 
  end, we created a workflow that balanced structure and flexibility. This disagreement actually strengthened our 
  team dynamic because it forced us to communicate openly, respect different perspectives, and make decisions based 
  on practical experience rather than assumptions.
  \end{enumerate}

\section*{Appendix --- Team Charter}

\wss{borrows from
\href{https://engineering.up.edu/industry_partnerships/files/team-charter.pdf}
{University of Portland Team Charter}}

\subsection*{External Goals}

Our team’s external goals for the AuraVoice project go beyond simply building a 
functional AI assistant and eventually make this as our software business. Short term goal is to create 
a product that can be showcased at the Capstone EXPO, demonstrating not only technical functionality but also creativity, 
innovation, and accessibility. Our team wants a polished and professional demonstration 
that can be highlighted in future interviews, showcasing our collaboration and problem-solving 
skills and hopefully land us jobs. Additionally, we are striving to earn an A+ in the course by delivering a high-quality, 
well-documented, and fully integrated project. Coming to long term goals, our team thinks that developing AuraVoice into a viable business product. 
Later we can refine its features, usability, and accessibility, that can lead to potential monetization opportunities such as licensing 
to assistive technology providers, offering subscriptions for educational or workplace settings, or providing customized solutions for 
individuals with disabilities. This dual focus on excellence and potential entrepreneurship motivates the team to deliver the project at the highest standard.

\subsection*{Attendance}

\subsubsection*{Expectations}

We expects all members to attend scheduled meetings on time and stay for the full duration, 
which is typically 45 minutes each week. Arriving more than 10 minutes late or leaving early without 
prior notice is discouraged, as it can disrupt the flow of the meeting and slow down progress. 
In case if a team member thinks they are going to be late or needing to leave early, they should inform the team in advance 
via Discord or email. During meetings, members are expected to actively participate by providing updates, 
sharing ideas, and giving feedback on ongoing work. We as a team believe that consistent attendance and engagement are crucial for 
keeping the team aligned, ensuring everyone understands their responsibilities, and maintaining momentum 
toward both academic and project goals.

\subsubsection*{Acceptable Excuse}

For missing a meeting or a deadline, the team understands that sometimes things come up that you can’t control, 
like being sick, a family emergency, or having an unavoidable class/lab, or exam. If this happens, the member 
should let the team know as soon as possible ideally before the meeting or deadline so we can plan around it. 
It is also understandable that things like transportation issues or technical problems (like laptop not working) 
can happen, and those can be acceptable as long as the member communicates the problem in a timely manner. Excuses that won’t be 
considered acceptable include forgetting about the meeting or deadline,or running personal errands during scheduled team time. 
If the absence or delay could have been prevented with planning or communication, it’s not an acceptable excuse. 
The team expects everyone to be honest,take responsibility for their commitments, and help the group adjust if someone genuinely can’t make it. By being upfront, 
we can redistribute tasks, keep the project on track, and make sure no one is left struggling with last-minute work.

\subsubsection*{In Case of Emergency}

If a team member has an emergency and cannot attend a scheduled meeting or meet a deadline, they should 
notify the team as soon as possible, ideally at least 6 to 8 hours before the meeting or deadline, explaining 
the situation and providing an estimated time to catch up. The team will temporarily adjust responsibilities to 
keep the project on track. A member may miss no more than one meeting or one major deadline due to an emergency without 
extra discussion; missing more than that requires a team plan to catch up and may involve adjusting responsibilities or 
notifying the TA. Once the emergency is resolved, the member must promptly catch up on all missed work, review meeting notes, 
and resume contributions to upcoming tasks and deadlines, ensuring the team maintains momentum and uses scheduled time effectively.

\subsection*{Accountability and Teamwork}

\subsubsection*{Quality} 

Our team expects all members to come to meetings fully prepared, having reviewed their assigned tasks, notes, 
or any relevant materials ahead of time. Deliverables brought to the team whether code, documentation, test cases, 
or research should meet a high standard of clarity, completeness, and correctness. Before any work is submitted or 
integrated into the project, it will be reviewed by at least one other team member to catch errors, improve clarity, 
and ensure consistency. Work should be well-organized and easy for other team members to understand. By maintaining 
high-quality deliverables, thorough preparation, and a peer review process, the team can avoid technical debt, ensure 
smooth collaboration, and create a professional product that not only meets academic goals like earning an A+ but also 
has the potential to be refined and sold as a market-ready solution in the future.

\subsubsection*{Attitude}

We want everyone to be respectful, cooperative, friendly, and able to freely communicate their ideas. 
Everyone is expected to actively participate in discussions, listen to different perspectives, and give 
helpful feedback when reviewing each other’s work. A positive attitude is important because it not only 
boosts individual performance but also helps the whole team collaborate more effectively and stay motivated. 
If conflicts come up, we’ll first talk it through as a team so everyone has a chance to explain their side and
present proofs of why they think one solution is better than another.If we can’t solve it ourselves, we’ll ask the TA for guidance as a last resort.

\subsubsection*{Stay on Track}

To keep the team on track, we’ll rely on GitHub projects and issues to organize tasks, 
track deadlines, and see everyone’s progress in real time. We’ll review completed work, 
upcoming deadlines, and any blockers at every weekly 45-minute meeting so nothing falls through 
the cracks. Everyone is expected to attend most meetings (90 percent benchmark), 
complete their assigned tasks on time,and have their work peer-reviewed before submission. Team members who consistently do a great job 
will be recognized during meetings, and small rewards like leading a discussion or picking a fun 
team activity can be given. If someone falls behind or misses targets, the team will first check 
in with them to see if they need support and may redistribute tasks if needed. Repeated issues might 
involve minor consequences, like bringing snacks or coffee to the next meeting for the highest contributor on the project, or in serious cases, 
asking the TA for guidance. By using clear tracking, regular check-ins, and open communication, we make 
sure everyone contributes fairly, the team stays coordinated, and the project keeps moving forward 
smoothly toward both our A+ goal and the long-term vision of creating a market-ready product.

\subsubsection*{Team Building}

We’ll be working together for a full year, so our team wants to make every meeting and work session something 
we actually look forward to. At the start of each meeting, we might do a quick check in where everyone shares 
a fun story, a small win from the week, or even a random interesting fact just enough to get everyone laughing 
or thinking. When we hit milestones, like finishing a big chunk of work or meeting a tight deadline, we’ll 
celebrate with team dinners or casual outings so we can enjoy the moment and connect outside of coding. We also 
plan to have spontaneous brainstorming sessions, mini challenges, or even 10-minute games to keep our energy up and 
creativity flowing. Spending a whole year together gives us a rare chance to really learn each other’s strengths, 
keep motivation high, and grow AuraVoice beyond a school project, this is something we hope could eventually turn 
into a real business. By keeping things lively, flexible, and supportive, we make sure working on this project is fun, 
productive, and memorable for the whole team.

\subsubsection*{Decision Making} 

The majority of the decisions made by our team are discussed collectively so that everyone can contribute their thoughts.  
We quickly vote to determine the best course of action if we are unable to agree either through discord or 
by show of hands in person.  When differences arise, we make sure that everyone listens to one another and 
shares their point of view.  We seek the TA for assistance or take a brief break if we are still unable to make up our minds.  
Everyone will feel heard in this way, and we can continue to work efficiently and remain inspired on AuraVoice.

\end{document}