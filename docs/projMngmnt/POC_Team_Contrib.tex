\documentclass{article}

\usepackage{float}
\restylefloat{table}

\usepackage{booktabs}

\title{Team Contributions: POC\\\progname}

\author{\authname}

\date{}

\input{../Comments}
%% Common Parts

\newcommand{\progname}{Software Engineering} % PUT YOUR PROGRAM NAME HERE
\newcommand{\authname}{Team \#23, Project Proxi
\\ Savinay Chhabra
\\ Amanbeer Singh Minhas
\\ Gourob Podder
\\ Ajay Grewal} % AUTHOR NAMES                  

\usepackage{hyperref}
    \hypersetup{colorlinks=true, linkcolor=blue, citecolor=blue, filecolor=blue,
                urlcolor=blue, unicode=false}
    \urlstyle{same}
                                


\begin{document}

\maketitle

This document summarizes the contributions of each team member up to the POC
Demo.  The time period of interest is the time between the beginning of the term
and the POC demo.

\section{Demo Plans}

The demo will be the following in order to properly showcase the base features 
of Proxi POC: It will be entirely textual for now, with inputs in the terminal 
to interact with the AI. We will ask the AI to, on my computer, go to my 
instagram, search for a famous instagram influencer (For example Sydney Sweeney)
and they will have to like one of their posts and post a comment. Then in a word
document on my computer it will have to write a small essay about the influencer. 

\section{Team Meeting Attendance}

\wss{For each team member how many team meetings have they attended over the
time period of interest.  This number should be determined from the meeting
issues in the team's repo.  The first entry in the table should be the total
number of team meetings held by the team.}

\begin{table}[H]
\centering
\begin{tabular}{ll}
\toprule
\textbf{Student} & \textbf{Meetings}\\
\midrule
Total & 6\\
Ajay Grewal & 6\\
Savinay Chhabra & 6\\
Amanbeer Singh Minhas & 6\\
Gouroub Podder & 6\\
\bottomrule
\end{tabular}
\end{table}

\wss{If needed, an explanation for the counts can be provided here.}

\section{Supervisor/Stakeholder Meeting Attendance}

\wss{For each team member how many supervisor/stakeholder team meetings have
they attended over the time period of interest.  This number should be determined
from the supervisor meeting issues in the team's repo.  The first entry in the
table should be the total number of supervisor and team meetings held by the
team.  If there is no supervisor, there will usually be meetings with
stakeholders (potential users) that can serve a similar purpose.}

\noindent \textbf{Supervisor's Name: } NO SUPERVISOR \\
\noindent \textbf{Stakeholders: } Team Member's grandparents

\begin{table}[H]
\centering
\begin{tabular}{ll}
\toprule
\textbf{Student} & \textbf{Meetings}\\
\midrule
Total & 1\\
Ajay Grewal & 1\\
Savinay Chhabra & 1\\
Amanbeer Singh Minhas & 1\\
Gouroub Podder & 1\\
\bottomrule
\end{tabular}
\end{table}

Individual Team Members had a meeting with their grandparents as they are one of
the primary stakeholders.

\section{Lecture Attendance}

\wss{For each team member how many lectures have they attended over the time
period of interest.  This number should be determined from the lecture issues in
the team's repo. You can find the number of lectures in the time period of
interest by looking at the
\href{https://calendar.google.com/calendar/u/0/embed?src=rnboqiaki1k2la7rpu3bn0um58@group.calendar.google.com&ctz=America/Toronto}
{Google calendar} for the capstone course.}

\wss{NOTE: There will be approximately 13 lectures between the start of class
and the POC demos}

\begin{table}[H]
\centering
\begin{tabular}{ll}
\toprule
\textbf{Student} & \textbf{Lectures}\\
\midrule
Total & 13\\
Ajay Grewal & 8\\
Savinay Chhabra & 8\\
Amanbeer Singh Minhas & 7\\
Gouroub Podder & 10\\
\bottomrule
\end{tabular}
\end{table}

\wss{If needed, an explanation for the lecture attendance can be provided here.}

\section{TA Document Discussion Attendance}

\wss{For each team member how many of the informal document discussion meetings
with the TA were attended over the time period of interest.}

\noindent \textbf{TA's Name: } Christopher Schankula

\begin{table}[H]
\centering
\begin{tabular}{ll}
\toprule
\textbf{Student} & \textbf{Lectures}\\
\midrule
Total & 1\\
Ajay Grewal & 0\\
Savinay Chhabra & 0\\
Amanbeer Singh Minhas & 0\\
Gouroub Podder & 0\\
\bottomrule
\end{tabular}
\end{table}

Even though they were 3 TA meetings that should have occured, our TA was 
unavailable for 2 of them and the team was not available for one of them.
\section{Commits}

\wss{For each team member how many commits to the main branch have been made
over the time period of interest.  The total is the total number of commits for
the entire team since the beginning of the term.  The percentage is the
percentage of the total commits made by each team member.}

\begin{table}[H]
\centering
\begin{tabular}{lll}
\toprule
\textbf{Student} & \textbf{Commits} & \textbf{Percent}\\
\midrule
Total & 79 & 100\% \\
Ajay Grewal & 22 & 27\% \\
Savinay Chhabra & 26 & 34\% \\
Amanbeer Singh Minhas & 15 & 18\% \\
Gouroub Podder & 16 & 20\% \\
\bottomrule
\end{tabular}
\end{table}

\wss{If needed, an explanation for the counts can be provided here.  For
instance, if a team member has more commits to unmerged branches, these numbers
can be provided here.  If multiple people contribute to a commit, git allows for
multi-author commits.}

\section{Issue Tracker}

\wss{For each team member how many issues have they authored (including open and
closed issues (O+C)) and how many have they been assigned (only counting closed
issues (C only)) over the time period of interest.}

\begin{table}[H]
\centering
\begin{tabular}{lll}
\toprule
\textbf{Student} & \textbf{Authored (O+C)} & \textbf{Assigned (C only)}\\
\midrule
Ajay Grewal & Num & Num \\
Savinay Chhabra & Num & Num \\
Amanbeer Singh Minhas & Num & Num \\
Gouroub Podder & Num & Num \\
\bottomrule
\end{tabular}
\end{table}

\wss{If needed, an explanation for the counts can be provided here.}

\section{CICD}

Our team uses GitHub Actions to manage Continuous Integration and
Continuous Deployment. The following automated steps are executed
throughout the development process:

\begin{itemize}
  \item Each push or pull request triggers a full CI pipeline including
        unit, integration, and system tests executed through
        \texttt{pytest}.
  \item The workflow also performs static analysis using \texttt{flake8}
        and \texttt{pylint} to detect syntax and maintainability issues.
  \item Pull requests cannot be merged unless all tests pass and at least
        one reviewer has approved the changes.
  \item Stable commits are tagged automatically and exported as demo
        builds to support testing and validation milestones.
\end{itemize}



\section{Team Charter Trigger Items}

\textbf{Trigger Summary:} Based on our team charter, the primary triggers
for intervention include:
\begin{itemize}
  \item Missing a scheduled team or supervisor meeting without at least
        24 hours notice.
  \item Repeated missed deadlines or uncommunicated delays on assigned
        deliverables (more than 1--2 occurrences).
  \item Consistently arriving late to meetings without prior notice.
  \item Submitting low-quality or incomplete work without explanation.
  \item Disruptive behavior or conflict without attempting resolution
        through team discussion.
\end{itemize}

\noindent\textbf{Observed Trigger Events:} No formal trigger violations occurred
during this period. All absences or delays were communicated in advance and
approved by the team. Examples include:
\begin{itemize}
  \item One member joining late due to a class scheduling overlap during
        Midterm Week.
  \item Another member briefly missing a work session to attend a lab but
        completing their assigned tasks later the same day.
  \item Short connectivity issues during one online meeting resolved within
        minutes.
\end{itemize}

All instances were communicated promptly and did not affect project progress
or deliverable quality.

\vspace{0.5em}
\noindent\textbf{Plan and Reflection:} No corrective actions were necessary as all
team members demonstrated responsibility and respect for deadlines and
communication. Moving forward, the team will:
\begin{itemize}
  \item Continue to provide early notice of scheduling conflicts.
  \item Balance academic workload and project responsibilities fairly.
  \item Maintain consistent participation and accountability.
\end{itemize}

Our current triggers remain appropriate and effective, requiring no revisions
at this stage.


\section{Additional Productivity Metrics}

To measure our overall progress and teamwork, the following simple
metrics were tracked during the period leading up to the POC demo:

\begin{itemize}
  \item \textbf{Meeting participation:} Each member attended over 90\%
        of team and supervisor meetings.
  \item \textbf{On-time submissions:} All major deliverables (SRS, VnV
        Plan, and Hazard Analysis) were submitted before the deadline.
  \item \textbf{Peer review response time:} Feedback on documents was
        usually provided within 2 days.
\end{itemize}

\end{document}