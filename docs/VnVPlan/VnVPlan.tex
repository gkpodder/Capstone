\documentclass[12pt, titlepage]{article}

\usepackage{booktabs}
\usepackage{tabularx}
\usepackage{hyperref}
\usepackage{float}
\usepackage{longtable}
\usepackage{caption}

\hypersetup{
    colorlinks,
    citecolor=blue,
    filecolor=black,
    linkcolor=red,
    urlcolor=blue
}
\usepackage[round]{natbib}

\input{../Comments}
%% Common Parts

\newcommand{\progname}{Software Engineering} % PUT YOUR PROGRAM NAME HERE
\newcommand{\authname}{Team \#23, Project Proxi
\\ Savinay Chhabra
\\ Amanbeer Singh Minhas
\\ Gourob Podder
\\ Ajay Grewal} % AUTHOR NAMES                  

\usepackage{hyperref}
    \hypersetup{colorlinks=true, linkcolor=blue, citecolor=blue, filecolor=blue,
                urlcolor=blue, unicode=false}
    \urlstyle{same}
                                


\begin{document}

\title{System Verification and Validation Plan for \progname{}} 
\author{\authname}
\date{\today}
	
\maketitle

\pagenumbering{roman}

\section*{Revision History}

\begin{tabularx}{\textwidth}{p{3cm}p{2cm}X}
\toprule {\bf Date} & {\bf Version} & {\bf Notes}\\
\midrule
Date 1 & 1.0 & Notes\\
Date 2 & 1.1 & Notes\\
\bottomrule
\end{tabularx}

~\\
\wss{The intention of the VnV plan is to increase confidence in the software.
However, this does not mean listing every verification and validation technique
that has ever been devised.  The VnV plan should also be a \textbf{feasible}
plan. Execution of the plan should be possible with the time and team available.
If the full plan cannot be completed during the time available, it can either be
modified to ``fake it'', or a better solution is to add a section describing
what work has been completed and what work is still planned for the future.}

\wss{The VnV plan is typically started after the requirements stage, but before
the design stage.  This means that the sections related to unit testing cannot
initially be completed.  The sections will be filled in after the design stage
is complete.  the final version of the VnV plan should have all sections filled
in.}

\newpage

\tableofcontents

\newpage

\pagenumbering{arabic}

This document outlines the Verification and Validation plan for Project Proxi. 
It explains how our team plans to check the system we're building against our 
requirements. The plan goes over what tests will be run, how documents and designs
will be validated and how everything connects back to the original SRS document.
The main goal is to make sure Project Proxi is reliable, easy to use, and truly 
helps people who will use it for accessibility.

\section{General Information}

\subsection{Summary}
The aim of this document is to layout the verification and validation plan for
Project Proxi. Project Proxi is an AI powered voice assistant which allows it's 
users to use their voice to interface with their computer. The general target 
demographic for Project Proxi is seniors and individuals with disabilities who 
wouldn't otherwise be able to use their computers for varying tasks. 

\subsection{Objectives}
The main objective of the Verification and Validation (V\&V) plan is to confirm
that Project Proxi satisfies it's stated functional and non functional 
requirements; and fulfuills its primary purpose of enhancing computer 
accessibility for seniors and individuals with disabilities.

The principal objectives that this VnV plan aims to meet are:
\begin{itemize}
  \item \textbf{Verify Functional Correctness:} Ensure that the features 
  described in the SRS are implemented and behave as expected through Unit, 
  Integration and System Level testing.
  \item \textbf{Validate Usability and Accessibility:} demonstrate that users 
  within the target demographic can effectively operate the system using natural
  language commands while meeting the relevant accessibility standards like 
  WCAG 2.
  \item \textbf{Assess Performance and reliability:} Confirm the application 
  meets performance, stability and error goals as defined in the SRS document.
  \item \textbf{Hardware Compatibility Testing:} Meet verficiation goals on a 
  variety of consumer hardware devices. Speacialized or adaptive devices will
  not be included; only commercially available Windows/MacOS Computers.
\end{itemize}

Out of scope objectives include Extensive long-term field usability studies, 
comprehensive privacy compliance, and Speacialized Hardware Testing. The project
will assume that any external libraries have already been verfied by their 
respective development team. These exclusions are justified given the project's 
limited time, budget and academic context. Verification efforts are therefore 
directed towards objectives that help meet core software requirements.


\subsection{Challenge Level and Extras}

This project does not include a designated challenge level, as its primary focus
 is on the development and validation of a functional, accessible, and 
 user-friendly AI-powered desktop assistant. The project scope aligns with the 
 general expectations of the capstone course and emphasizes reliability, 
 usability, and compliance with accessibility standards rather than advanced 
 AI research or complex algorithmic innovation.


This project includes the following approved extras:
\begin{itemize}
  \item \textbf{User Manual:} A user manual that explains how to install, setup,
  and use the software on a day-to-day basis.
  \item \textbf{Usability Report:} A usability report will be developed that 
  will assess the systems core objectives and requirements. It will summarize 
  the results of testing as specfied in the SRS doc.
\end{itemize}

\subsection{Relevant Documentation}

The following documentation is directly relevant to the Verification and 
Validation activities for Project Proxi:
\begin{itemize}
  \item \textbf{Development Plan [\citet{DevPlan}]:} This document defines the overall project 
  workflow, timelines and milestones for the different phases of the project.
  \item \textbf{Problem Statement and Goals [\citet{ProbStatement}]:}: This document defines the problems,
   goals and stakeholders of the project.
  \item \textbf{Software Requirements Specification [\citet{SRS}]:}  Describes the 
  functional and non functional requirements of the project.
  \item \textbf{Hazard Analysis [\citet{HazardAnalysis}]:} This document identifies any potenital sources
  of harm or risk that may be associated with the project. Potential risks like
  unauthorized access, misuse or data breaches and their mitigation strategies 
  are documented here.
  \item \textbf{Usability Report[\citet{UsabilityReport}]:} Includes results of user evaluation and beta 
  testing during validation stage. It summarizes key findings related to the 
  system meeting it's design and usability goals.
  \item \textbf{User Manual [\citet{UserManual}]:} Provides documentation that explains how to 
  install, setup and use the software. Includes helpful instruction along with 
  walkthroughs and examples to help users with varying techincal proficiency.
\end{itemize}

\section{Plan}
This section outlines the strategy for Verification and Validation (V\&V)
across all phases of the Project Proxi lifecycle. The plan defines the roles
and responsibilities of the V\&V team and specifies structured approaches for
verifying the Software Requirements Specification (SRS), Design, and
Implementation, as well as the approach for Software Validation. This
systematic process increases confidence in the correctness, accessibility,
and reliability of the final system.

Verification ensures that the system is built correctly and meets its stated
requirements, while validation confirms that the right system has been built
for its intended users. Both static techniques (reviews, inspections, and
walkthroughs) and dynamic techniques (testing and user evaluations) are used.
All artifacts, results, and issues are tracked through GitHub for full
traceability between requirements, test cases, and evidence.

\subsection{Verification and Validation Team}

The V\&V activities are shared among all members of the Proxi team. Each
member’s role leverages their technical strengths while maintaining overlap
to ensure redundancy and consistent quality assurance. All verification tasks
and results are peer-reviewed, documented, and tracked in GitHub.

\begin{table}[H]
\centering
\begin{tabular}{|p{3cm}|p{3.5cm}|p{7.5cm}|}
\hline
\textbf{Name} & \textbf{Role} & \textbf{V\&V Responsibilities} \\ \hline
Savinay Chhabra & Team-Lead/ Project-Manager &
Coordinates all V\&V activities, schedules document and design reviews,
ensures traceability matrices are complete, and validates plan feasibility. \\
\hline
Amanbeer Singh Minhas & Verification Engineer / Documentation Lead &
Authors test specifications, manages accessibility validation (WCAG 2.1 AA),
executes FR and NFR tests, and maintains V\&V evidence and metrics. \\
\hline
Gourob Podder & AI/Backend Developer &
Verifies accuracy and latency of voice-recognition and NLP modules,
conducts integration and performance tests, and performs code inspections. \\
\hline
Ajay Singh Grewal & Front-End / Integration Engineer &
Implements automated UI tests, validates responsiveness and error handling,
and ensures cross-platform consistency and workflow reliability. \\
\hline
Team Proxi & Quality-Assurance and Review Team &
Participate in peer reviews of all documents (SRS, MIS, V\&V, and Hazard
Analysis). Execute assigned system and unit tests, record results, file issues
on GitHub, and contribute to final validation reports \\
\hline
\end{tabular}
\caption{Verification and Validation Team Roles}
\label{tab:1}
\end{table}

\subsection{SRS Verification}

The Software Requirements Specification (SRS) will be verified using a
structured review process involving internal checks, peer feedback, and
automated validation. The goal is to confirm that all requirements are clear,
consistent, and ready for design and testing.

\vspace{0.5em}
\noindent\textbf{Planned Verification Stages:}
\begin{itemize}
\item \textbf{Internal Team Review:}  
  Each team member reviews assigned SRS sections to ensure requirements are
  clearly written, uniquely numbered, and testable. Feedback and questions are
  added as GitHub issues under the ``SRS-review'' tag for tracking.
\item \textbf{Peer Team Feedback:}  
  Another project team will review our SRS to identify unclear or missing
  requirements. Their comments will be discussed in a short review meeting,
  and agreed changes will be included in the next SRS revision.
\item \textbf{Automated Consistency and Traceability Check:}  
  Scripts and manual scans confirm that every requirement label (FR, NFR, etc.)
  appears in the traceability table and links to a planned test in the V\&V
  Plan. Any missing or duplicate IDs will be flagged for correction.
\item \textbf{Team Validation Meeting:}  
  After all updates, the team meets to confirm that project goals such as
  accessibility, reliability, and usability are fully captured in the SRS.
  The final version is approved as the baseline for design and testing.
\end{itemize}

\subsection{Design Verification}

The design of Project Proxi will be verified through focused peer reviews and
checklists to confirm that the architecture, interfaces, and data flows align
with the approved SRS. Each review will help identify inconsistencies or
missing details before implementation begins.

\vspace{0.5em}
\noindent\textbf{Planned Verification Activities:}
\begin{itemize}
\item \textbf{Internal Design Review:}  
  Team members will examine their assigned modules to ensure the design is
  modular, consistent with SRS requirements, and feasible for implementation.
  Each reviewer will complete a checklist covering naming conventions,
  interface clarity, and data flow accuracy.

\item \textbf{Cross-Team Design Review:}  
  Another project team will review our design document and provide feedback on
  readability, logic, and completeness. Their comments will be logged as
  GitHub issues tagged ``Design-review'' and resolved in the next revision.

\item \textbf{Walkthrough Meeting:}  
  The team will conduct a collaborative walkthrough where each member explains
  their subsystem’s role and connections. This helps verify that all modules
  integrate correctly and that there are no overlapping or missing elements.

\item \textbf{Traceability Check:}  
  A simple matrix will confirm that every major design component maps to at
  least one SRS requirement. Any unmatched modules or requirements will be
  flagged for revision before coding starts.
\end{itemize}

\vspace{0.5em}
\noindent\textbf{Design Review Checklist Items:}
\begin{itemize}
\item Module responsibilities are clear and non-overlapping.
\item Inputs, outputs, and interfaces are well defined and consistent.
\item Data flow diagrams reflect correct information movement.
\item Error handling and edge cases are addressed.
\item Design supports accessibility and responsiveness goals.
\item Traceability to the SRS is complete for all components.
\end{itemize}

\subsection{Verification and Validation Plan Verification}

The Verification and Validation (V\&V) Plan itself will also be verified to
ensure it is complete, consistent, and realistic. This review confirms that
the plan defines clear responsibilities, test coverage, and measurable
evidence of software quality.

\vspace{0.5em}
\noindent\textbf{Planned Verification Activities:}
\begin{itemize}
\item \textbf{Internal Review:}  
  Team members will inspect the V\&V Plan for completeness, logical flow, and
  consistency with other project documents such as the SRS, MIS, and Hazard
  Analysis. Issues will be logged on GitHub under the tag ``VnV-review''.

\item \textbf{Peer Team Review:}  
  A partner project team will review the plan to provide external feedback on
  clarity, organization, and feasibility. Comments will be discussed in a team
  meeting, and accepted suggestions will be incorporated into the next version.

\item \textbf{Cross-Document Consistency Check:}  
  The team will confirm that all V\&V activities listed here appear in other
  relevant documents and that requirement identifiers match across artifacts.

\item \textbf{Review Process Validation:}  
  Small, intentional edits (such as removing a trace link or changing a test
  description) will be introduced to confirm that the team’s review process
  can detect inconsistencies. This validates the effectiveness of our review
  workflow.
\end{itemize}

\vspace{0.5em}
\noindent\textbf{V\&V Plan Review Checklist:}
\begin{itemize}
\item All documents are correctly referenced and consistent.
\item Roles and responsibilities match Table~\ref{tab:1}.
\item Each section clearly states its purpose and expected outcomes.
\item Traceability between requirements and tests is complete.
\item Internal and peer review processes are clearly defined.
\item Plan is feasible and aligned with the project timeline and tools.
\end{itemize}

\subsection{Implementation Verification}

The implementation of Project Proxi will be verified through a combination of
dynamic and static techniques to ensure the system performs correctly,
reliably, and according to its specified design. This section summarizes how
unit, integration, and system testing activities will be supported by
automated tools and structured reviews.

\vspace{0.5em}
\noindent\textbf{Dynamic Verification:}
\begin{itemize}
\item \textbf{Unit Testing:}  
  Each software module will have dedicated unit tests as outlined in the Unit
  Testing Plan. Tests will confirm correct outputs, data flow, and error
  handling for voice-processing, NLP, and UI modules. Coverage results will be
  monitored through automated reports to ensure that all major functions are
  exercised.

\item \textbf{Integration and System Testing:}  
  Combined module testing will validate interactions between the speech engine,
  user interface, and backend components. These tests correspond directly to
  the functional and non-functional test cases listed in this document.

\item \textbf{Mutation Testing:}  
  To evaluate the quality of our test suite, small artificial faults (mutants)
  will be introduced into the codebase. Existing unit and system tests will be
  rerun to confirm that they detect the injected faults. A high mutation score
  will increase confidence in the strength and completeness of our test suite.
\end{itemize}

\vspace{0.5em}
\noindent\textbf{Static Verification:}
\begin{itemize}
\item \textbf{Code Walkthroughs and Inspections:}  
  Scheduled walkthrough meetings will allow each developer to explain their
  code and receive peer feedback on structure, readability, and consistency.
  The final class presentation will also serve as a partial code walkthrough
  and an opportunity to discuss implementation decisions.

\item \textbf{Static Analysis Tools:}  
  Linters and analyzers (such as ESLint, Pylint, or similar) will be used to
  detect common issues, unused variables, and potential logic errors. Results
  will be documented as part of the verification evidence.

\item \textbf{Code Review via Pull Requests:}  
  All implementation changes will go through GitHub pull requests with at
  least one reviewer. Reviewers will verify coding standards, style, and
  alignment with design diagrams.
\end{itemize}

\subsection{Automated Testing and Verification Tools}

Automated tools will be used throughout the implementation phase to support
testing, continuous integration, and quality assurance. These tools help
reduce manual effort, detect regressions early, and provide measurable
evidence of code reliability and maintainability.

\vspace{0.5em}
\noindent\textbf{Unit Testing Framework:}
\begin{itemize}
\item The project will use the built-in \texttt{pytest} framework for Python to
  create and execute unit tests. Each module will have its own test file and
  fixtures to verify inputs, outputs, and edge cases. Test results will be
  automatically generated after every commit.
\end{itemize}

\vspace{0.5em}
\noindent\textbf{Continuous Integration (CI):}
\begin{itemize}
\item GitHub Actions will be configured to automatically run unit, integration,
  and system tests whenever new code is pushed. The CI workflow will include
  static analysis, linting, and coverage checks. Test outcomes and code
  coverage metrics will be displayed in each pull request.
\end{itemize}

\vspace{0.5em}
\noindent\textbf{Code Coverage and Reporting:}
\begin{itemize}
\item Tools like \texttt{coverage.py} will be used to measure how much of the source code is
  executed during tests. Reports will include statement, branch, and function
  coverage. The goal is to maintain at least 85\% overall coverage. Coverage
  summaries will be included in milestone reports.
\end{itemize}

\vspace{0.5em}
\noindent\textbf{Static Analysis and Linters:}
\begin{itemize}
\item Static analyzers such as \texttt{pylint} and \texttt{flake8} will check for
  syntax errors, unused variables, and potential logic flaws. These tools also
  verify that code follows project style guidelines and maintainability rules.
  Results will be logged and tracked as part of the verification evidence.
\end{itemize}

\vspace{0.5em}
\noindent\textbf{Build and Automation Utilities:}
\begin{itemize}
\item The project will use \texttt{make} commands and scripts to automate
  environment setup, dependency installation, and test execution. This ensures
  consistent testing results across all development machines.
\end{itemize}

\subsection{Software Validation}

Software validation confirms that Project Proxi satisfies user needs and
performs as intended in realistic scenarios. The focus is on demonstrating
that the system achieves its goals of accessibility, responsiveness, and ease
of use for all users, including those with limited mobility or vision.

\vspace{0.5em}
\noindent\textbf{Planned Validation Activities:}
\begin{itemize}
\item \textbf{User Testing Sessions:}  
  A small group of peers and volunteers will perform task-based evaluations,
  such as launching applications, dictating text, and managing files through
  voice commands. Their feedback will measure ease of learning, efficiency,
  and satisfaction.

\item \textbf{Accessibility Validation:}  
  Validation will focus on confirming compliance with WCAG 2.1 AA
  accessibility principles. Voice feedback clarity, contrast, and response
  times will be recorded and compared with expected usability metrics defined
  in the SRS.

\item \textbf{Stakeholder Review Session:}  
  The team will present the system in a demonstration that acts as a Rev 0
  validation milestone. Feedback from course staff and classmates will be
  collected through short surveys and used to improve the user experience.

\item \textbf{External Data and Comparison:}  
  Where available, open-source voice datasets will be used to validate recognition
  accuracy against baseline results. This ensures that the system behaves consistently
  with established standards.
\end{itemize}


\section{System Tests}

Each test is specific and measurable, providing sufficient detail for replication 
once the system is built. The tests are grouped into logical areas covering input 
handling, natural language understanding, task execution, feedback mechanisms, 
memory management, logging, and system safety.

\subsection{Tests for Functional Requirements}

The following tests verify that each functional requirement defined in the SRS
has been properly implemented and behaves as intended. Each test specifies the
type of testing, initial system state, inputs and expected results, and how the
test will be performed.

\subsubsection{Input and Recognition Requirements}

\begin{enumerate}

\item{Test FUNC.IR.1 - Speech and Text Input Acceptance\\}
\textbf{Type:} Functional / Manual Testing

\textbf{Initial State:} System initialized and idle at the main interface.

\textbf{Input/Condition:} Users provide 20 valid commands, 10 by text input (keyboard)
and 10 by speech through the microphone interface. Commands include examples like
“Open the calendar,” “What’s the weather today?”, and “Send an email to John.”

\textbf{Output/Result:} System correctly recognizes and processes at least 95\% of both
speech and text inputs, generating the corresponding interpreted command in the console
or display window.

\textbf{How test will be performed:} Each input is logged automatically. The test supervisor
will compare recognized commands to expected ones and calculate the success rate. Failures
and misinterpretations will be noted for analysis.

\item{Test FUNC.IR.2 - Speech-to-Text Accuracy (ASR Module)\\}
\textbf{Type:} Functional / Automated Evaluation

\textbf{Initial State:} System with ASR module enabled and configured in a controlled
acoustic environment.

\textbf{Input/Condition:} 50 pre-recorded audio samples containing common computer commands
spoken by 3 different speakers with varied accents.

\textbf{Output/Result:} Transcribed text matches the ground-truth transcript with at least
90\% word accuracy.

\textbf{How test will be performed:} Audio samples will be batch-fed into the ASR module.
Output transcripts will be compared to reference text using a word-error-rate (WER)
metric to compute accuracy.
\end{enumerate}
\textbf{Functional Requirements Covered:} FUNC.R.1 - FUNC.R.2

\subsubsection{Intent Understanding and Task Execution}

\begin{enumerate}

\item{Test FUNC.IT.1 - Intent Recognition Accuracy\\}
\textbf{Type:} Functional / Automated

\textbf{Initial State:} Natural language understanding (NLU) component initialized
with trained intent classification model.

\textbf{Input/Condition:} 100 labeled user commands in natural language, each mapped
to a known ground-truth intent (e.g., send email, open browser, play music).

\textbf{Output/Result:} In at least 90\% of cases, the intent predicted by the system
matches the labeled ground truth.

\textbf{How test will be performed:} A test script feeds the commands into the NLU module.
Predicted intents are logged and compared to human-labeled references for accuracy scoring.

\item{Test FUNC.IT.2 - Task Planning and Execution Success Rate\\}
\textbf{Type:} Functional / Integration

\textbf{Initial State:} System active with available software agents and network
connectivity.

\textbf{Input/Condition:} 20 representative high-level commands (e.g., “Schedule a
meeting,” “Search the web for news,” “Summarize this document”).

\textbf{Output/Result:} Agents execute actions successfully and complete tasks in at
least 85\% of test cases.

\textbf{How test will be performed:} Commands are executed sequentially. Testers verify
that each resulting task outcome matches the intended user goal. Failures are logged
with error cause (planning vs. execution failure).
\end{enumerate}
\textbf{Functional Requirements Covered:} FUNC.R.3 - FUNC.R.4

\subsubsection{Feedback and Responsiveness}

\begin{enumerate}

\item{Test FUNC.FR.1 - Response Feedback Timing\\}
\textbf{Type:} Performance / Manual Timing

\textbf{Initial State:} System ready to execute user commands.

\textbf{Input/Condition:} 10 varied user tasks executed (e.g., “Create a note,” “Check
calendar,” “Send a message”).

\textbf{Output/Result:} Each command generates a feedback or confirmation message within
2 seconds of task completion.

\textbf{How test will be performed:} A timer will start at the moment of task completion.
Feedback appearance time is recorded for each trial. Average and max response times will
be compared to the 2-second threshold.
\end{enumerate}
\textbf{Functional Requirements Covered:} FUNC.R.5

\subsubsection{Contextual Memory and Logging }

\begin{enumerate}

\item{Test FUNC.ML.1 - Contextual Follow-Up Command Interpretation\\}
\textbf{Type:} Functional / Manual Sequence Test

\textbf{Initial State:} System with active short-term conversational memory.

\textbf{Input/Condition:} 20 conversational command sequences (e.g., “Open my calendar” then
“Add meeting at 2 PM” then “Move it to tomorrow”).

\textbf{Output/Result:} Follow-up commands correctly interpreted in at least 90\% of
cases based on prior context.

\textbf{How test will be performed:} Each conversation sequence is entered by testers.
The interpreted action and memory context will be reviewed to ensure proper resolution
of pronouns and implicit references.

\item{Test FUNC.ML.2 - Interaction and Task Logging Verification\\}
\textbf{Type:} Functional / Inspection

\textbf{Initial State:} Logging module enabled with write access to local or cloud storage.

\textbf{Input/Condition:} 10 user sessions including a mix of successful and failed
tasks performed via speech and text.

\textbf{Output/Result:} Each session generates a timestamped log containing input,
interpreted intent, actions performed, and results.

\textbf{How test will be performed:} After each session, reviewers inspect the log file
for completeness and proper formatting (timestamp, session ID, action outcome).
\end{enumerate}
\textbf{Functional Requirements Covered:} FUNC.R.6 - FUNC.R.7

\subsubsection{Usability and Safety}

\begin{enumerate}

\item{Test FUNC.US.1 - High-Level Command Usability\\}
\textbf{Type:} Usability / Observational

\textbf{Initial State:} Deployed system interface with speech and text enabled.

\textbf{Input/Condition:} 10 participants attempt to complete 10 standard tasks (e.g.,
opening documents, browsing the web, checking system info) using only natural language.

\textbf{Output/Result:} 100\% of users complete tasks without needing to use low-level
system interfaces (e.g., file paths, terminal commands).

\textbf{How test will be performed:} Observers record each participant’s interactions.
Any instance where a user needs to access a system-level interface is noted as a failure.

\item{Test FUNC.US.2 - Privileged Action Confirmation\\}
\textbf{Type:} Functional / Security Validation

\textbf{Initial State:} System running with permission-sensitive agents enabled.

\textbf{Input/Condition:} Privileged commands such as “Delete file,” “Shut down
computer,” or “Access settings.”

\textbf{Output/Result:} Each privileged command triggers a confirmation prompt and
executes only after explicit user approval.

\textbf{How test will be performed:} Testers issue privileged commands via both text
and speech. Observers confirm that the system requests confirmation before any
irreversible or high-privilege action.
\end{enumerate}
\textbf{Functional Requirements Covered:} FUNC.R.8 - FUNC.R.9



\subsection{Tests for Nonfunctional Requirements}

In here specific tests are defined to help determine if Proxi's nonfunctional
requirements have been met.  Every test in this will look at the requirements
from the SRS and determine how every result will be verified.

\subsubsection{Look and Feels Requirements}

\begin{enumerate}
					
\item{Test NF.LF.1 - UI Conformance and Readability\\}
\textbf{Type:} Static inspection and Manual
          
\textbf{Initial State:} The latest UI build
          
\textbf{Input/Condition:} Walkthrough the man screens and all of the status
indicators (listening, thinking, and ready) all with a checklist.
          
\textbf{Output/Result:} Presence and visibility of main controls such as listen,
stop, and undo; minimal clutter and advanced options hidden; large targets and
a theme toggle being present. 
          
\textbf{How test will be performed:} In this two reviewers will go through the
UI and check off the items on the checklist.

\textbf{Nonfunctional Requirements Covered:} APP.1 - APP.5

\item{Test NF.LF.2 - Test Style and Iconography Check\\}
\textbf{Type:} Static inspection and Manual
          
\textbf{Initial State:} Build with speech output enabled and the UI icon set 
implemented.
          
\textbf{Input/Condition:} It will be a trigger system speech, with review 
labels, captions, and icons.
          
\textbf{Output/Result:} Plain, short labels; synchronized on screen captions for 
spoken output; consistent iconography with textual labelling.
          
\textbf{How test will be performed:} Inspecting the strings and visual elements
and flows; record any inconsistencies found with videos and screenshots for
evidence.

\textbf{Nonfunctional Requirements Covered:} STY.1 - STY.3

\end{enumerate}

\subsubsection{Usability and Humanity Requirements}

\begin{enumerate}
					
\item{NF.US.1 - Task Based Usability and Learnability Study\\}
\textbf{Type:} Dynamic with manual usability test that will have timed tasks and
a survey.
          
\textbf{Initial State:} New users will get a 5 minute orientation to Proxi.
          
\textbf{Input/Condition:} Doing the core tasks such as opening the app, reading
files, browsing, and saving files/actions.
          
\textbf{Output/Result:} A simple command button works for simple tasks; main
actions are obvious; the participants do every task without assistance; time to
do a repeat task decreases; examples on what to say are easily found; messages
are simple, risky actions are confirmed, and error messages suggest next steps.

\textbf{How test will be performed:} Observing timed tasks, collecting
completion rates and any errors.

\textbf{Nonfunctional Requirements Covered:} EOU.R.1 - EOU.R.3, LEA.R.1 - 
LEA.R.2, UAP.R.1 - UAP.R.3

\item{NF.US.2 - Accessibility, personalization, and AODA/WCAG conformance\\}
\textbf{Type:} Dynamic test and accessibility inspection
          
\textbf{Initial State:} Build with voice only pathway enabled; captions, text
scaling, and color contrast tools implemented.
          
\textbf{Input/Condition:} There is voice only interactive execution of setup
and exit; repeated dictation sessions with captions enabled and locale set to
English.
          
\textbf{Output/Result:} All actions are possible with voice only; captions,
scaling, and contrast verified; speech model adaption over user session are 
noticeable; Canadian english formats are used.

\textbf{How test will be performed:} Run some WCAG checks, verify locale outputs,
and have users test the voice only pathway.

\textbf{Nonfunctional Requirements Covered:} ACC.R.1 - ACC.R.2, PER.R.1 - 
PER.R.2

\end{enumerate}

\subsubsection{Performance Requirements}

\begin{enumerate}

\item{NF.PF.1 - Performance Study\\}
\textbf{Type:} Dynamic, mostly automatic performance testing
          
\textbf{Initial State:} Standard hardware environment with typical system load
          
\textbf{Input/Condition:} Speak a small scripted set of normal commands and a few 
malformed ones; keep one run in a quiet room and one with moderate indoor noise. 
Let the app run for a few hours
          
\textbf{Output/Result:} Respond in less than 2 seconds; actions should be done
with 80 to 90 percent accuracy; features load fast; settings are kept; rollbacks
fast.

\textbf{How test will be performed:} Time commands, tally accuracy, log resources,
and perform updates and rollbacks.

\textbf{Nonfunctional Requirements Covered:} SAL.R.1-R.2, SAF.R.1-R.2, POA.R.1-R.2,
CAP.R.1-R.2, SOE.R.1-R.2, LON.R.1-R.2.

\end{enumerate}

\subsubsection{Operational and Enviromental Requirements}

\begin{enumerate}

\item{NF.OP.1 - Environment, ecosystem fit, interfaces, packaging, and release\\}
\textbf{Type:} Dynamic and checklist based testing
          
\textbf{Initial State:} Fresh install on Windows/macOS with integrations turned
on.

\textbf{Input/Condition:} Use the app indoors, with normal environment, 
including noise, temperature, and distance; clean install; check release info

\textbf{Output/Result:} Works within env bounds; does not disrupt other apps; 
uses secure interfaces with logs; installs easily; versioned and documented
releases.

\textbf{How test will be performed:} Spot checking tasks and env, glancing at 
traffic + logs, running clean installs, and reviewing the releasing page. 

\textbf{Nonfunctional Requirements Covered:} EPE.R.1-R.4; WER.R.1-R.2; 
IAS.R.1-R.4; PRD.R.1-R.3; REL.R.1-R.2

\end{enumerate}

\subsubsection{Maintability and Support Requirements}

\begin{enumerate}

\item{NF.MS.1 - Onboarding, support, and adaptability\\}
\textbf{Type:} Light task trial and checklist based testing
          
\textbf{Initial State:} Fresh clone; Report Problem presentable and the config 
editable.

\textbf{Input/Condition:} A new person builds and runs using the README; they add
one small tool through the registry; generate support bundle; and change default 
apps via the config and retry.

\textbf{Output/Result:} The build run succeeds; the tool added without touching
others; the support bundle has logs and info, it still works after the default 
app change.

\textbf{How test will be performed:} Time the setup, add and register the tool,
click report to inspect the ZIP, and switch default apps and retest.

\textbf{Nonfunctional Requirements Covered:} 14.1-1, 14.1-2; 14.2-1, 14.2-2; 
14.3-1, 14.3-2

\end{enumerate}

\subsubsection{Security Requirements}

\begin{enumerate}

\item{NF.SEC.1 - Access, integrity, privacy, audit stance, immunity\\}
\textbf{Type:} Permission checks and quick scans

\textbf{Initial State:} Fresh install; privacy policy; dependency scanner.

\textbf{Input/Condition:} Trigger sensitive actions; look for stored and sent data;
run dependency scans.

\textbf{Output/Result:} No account needed; sensitive actions require explicit grants;
data is protected; minimal third party data accessed with consent; no audit trails
are left; no known vulnerabilities in dependencies.

\textbf{How test will be performed:} Start flows that require permissions, look 
for traffic and logs, and run dependency scans. Run SCA scans.

\textbf{Nonfunctional Requirements Covered:} ACS-01, ACS-02; INT-01; PRIV-01; 
IMM-01, IMM-02

\end{enumerate}

\subsubsection{Cultural Requirements}

\begin{enumerate}

\item{NF.CUL.1 - Language and Tone\\}
\textbf{Type:} Quick review of content

\textbf{Initial State:} Final strings and language selector.

\textbf{Input/Condition:} Scan common screens and messages. Switch languages and 
check key screens.

\textbf{Output/Result:} Polite neutral wording; multiple languages will be usable 
for many basic flows.

\textbf{How test will be performed:} Two people will go through the common screens
and messages; one person will switch languages, complete tasks, and check key 
screens.

\textbf{Nonfunctional Requirements Covered:} CULR-01, CULR-02

\end{enumerate}

\subsubsection{Compliance Requirements}

\begin{enumerate}

\item{NF.CUL.1 - Legal and standards\\}
\textbf{Type:} Checklist

\textbf{Initial State:} Licenses, privacy policy, accessibility notes in repo
and release.

\textbf{Input/Condition:} Map behaviors to legal and standards checklist. This 
includes PIPEDA, MIT.

\textbf{Output/Result:} PIPEDA IS followed; all licenses are compatible and
included. The given standards are followed correctly.

\textbf{How test will be performed:} Check policies vs behaviors, open the
LICENSE and release notes, do quick scan and spot check, and confirm protocols
are secure.

\textbf{Nonfunctional Requirements Covered:} CULR-01, CULR-02

\end{enumerate}

\subsection{Traceability Between Test Cases and Requirements}
\begin{center}
\captionsetup{type=table}
\caption{Requirements and Test Traceability Matrix}
\begin{longtable}{|p{2.5cm}|p{4cm}|p{7.5cm}|}
\hline
\textbf{Test ID} & \textbf{Requirement ID} & \textbf{Comment} \\ \hline
\endfirsthead

\multicolumn{3}{c}%
{\tablename\ \thetable\ -- \textit{Continued from previous page}} \\ \hline
\textbf{Test ID} & \textbf{Requirement ID} & \textbf{Comment} \\ \hline
\endhead

\hline \multicolumn{3}{r}{\textit{Continued on next page}} \\ \hline
\endfoot

\hline
\endlastfoot

% --- Function Requirements ---
FUNC.IR.1  & FUNC.R.1 & Manual Test that checks if system successfully accepted user input\\
\hline
FUNC.IR.2  & FUNC.R.2 & Automated Test that checks Speech to Text Accuracy \\ \hline
FUNC.IT.1  & FUNC.R.3 & Automated Test that checks if system corrected interpreted user intent\\
\hline
FUNC.IT.2  & FUNC.R.4 & Integration Test that checks the system end-to-end\\
\hline
FUNC.FR.1  & FUNC.R.5 & Manual Test that checks if system is ready to execute user commands\\
\hline
FUNC.ML.1  & FUNC.R.6 & Manual Test that checks the short term memory of the model\\
\hline
FUNC.ML.2  & FUNC.R.7 & Automated Functional that checks if system logs are working as expected\\
\hline
FUNC.US.1  & FUNC.R.8 & Manual Observational test to check if users can complete task without needing low-level system interface\\
\hline
FUNC.US.2  & FUNC.R.9 & Automated test to make sure system doesn't use any unauthorized commands without prompting\\
\hline

% --- Non-Functional Requirements ---
NF.LF.1 & APP.1 - APP.5 & Manual Test that covers the Appearance Requirements\\
\hline
NF.LF.2 & STY.1 - STY.3 & Manual Test that covers the Style Requirements\\
\hline
NF.US.1 & EOU.R.1 - EOU.R.3, LEA.R.1, LEA.R.2, UAP.R.1 - UAP.R.3 & Manual Test that covers the Usability, Learning, Understanding, and Politeness Requirements\\
\hline
NF.PF.1  & SAL.R.1-R.2, SAF.R.1-R.2, POA.R.1-R.2, CAP.R.1-R.2, SOE.R.1-R.2, LON.R.1-R.2 & Automatic Test that covers the Performance Requirements\\
\hline
NF.OP.1 & EPE.R.1-R.4; WER.R.1-R.2; IAS.R.1-R.4; PRD.R.1-R.3; REL.R.1-R.2 & Automatic Test that covers the Operation and Environmental Requirements\\
\hline
NF.MS.1 & 14.1-1, 14.1-2; 14.2-1, 14.2-2; 14.3-1, 14.3-2 & Manual Test that covers the Maintainability and Support Requirements\\
\hline
NF.SEC.1 & CS-01, ACS-02, INT-01, PRIV-01, IMM-01, IMM-02 & Automatic Test that covers the Security Requirements\\
\hline
NF.CUL.1 & CULR-01, CULR-02 & Manual Test that covers the Cultural Requirements\\
\hline
NF.COM.1 & LGL-01, LGL-02 & Automated Test that covers the Compliance Requirements\\
\hline

\end{longtable}
\end{center}

\section{Unit Test Description}

\wss{This section should not be filled in until after the MIS (detailed design
  document) has been completed.}

\wss{Reference your MIS (detailed design document) and explain your overall
philosophy for test case selection.}  

\wss{To save space and time, it may be an option to provide less detail in this section.  
For the unit tests you can potentially layout your testing strategy here.  That is, you 
can explain how tests will be selected for each module.  For instance, your test building 
approach could be test cases for each access program, including one test for normal behaviour 
and as many tests as needed for edge cases.  Rather than create the details of the input 
and output here, you could point to the unit testing code.  For this to work, you code 
needs to be well-documented, with meaningful names for all of the tests.}

\subsection{Unit Testing Scope}

\wss{What modules are outside of the scope.  If there are modules that are
  developed by someone else, then you would say here if you aren't planning on
  verifying them.  There may also be modules that are part of your software, but
  have a lower priority for verification than others.  If this is the case,
  explain your rationale for the ranking of module importance.}

\subsection{Tests for Functional Requirements}

\wss{Most of the verification will be through automated unit testing.  If
  appropriate specific modules can be verified by a non-testing based
  technique.  That can also be documented in this section.}

\subsubsection{Module 1}

\wss{Include a blurb here to explain why the subsections below cover the module.
  References to the MIS would be good.  You will want tests from a black box
  perspective and from a white box perspective.  Explain to the reader how the
  tests were selected.}

\begin{enumerate}

\item{test-id1\\}

Type: \wss{Functional, Dynamic, Manual, Automatic, Static etc. Most will
  be automatic}
					
Initial State: 
					
Input: 
					
Output: \wss{The expected result for the given inputs}

Test Case Derivation: \wss{Justify the expected value given in the Output field}

How test will be performed: 
					
\item{test-id2\\}

Type: \wss{Functional, Dynamic, Manual, Automatic, Static etc. Most will
  be automatic}
					
Initial State: 
					
Input: 
					
Output: \wss{The expected result for the given inputs}

Test Case Derivation: \wss{Justify the expected value given in the Output field}

How test will be performed: 

\item{...\\}
    
\end{enumerate}

\subsubsection{Module 2}

...

\subsection{Tests for Nonfunctional Requirements}

\wss{If there is a module that needs to be independently assessed for
  performance, those test cases can go here.  In some projects, planning for
  nonfunctional tests of units will not be that relevant.}

\wss{These tests may involve collecting performance data from previously
  mentioned functional tests.}

\subsubsection{Module ?}
		
\begin{enumerate}

\item{test-id1\\}

Type: \wss{Functional, Dynamic, Manual, Automatic, Static etc. Most will
  be automatic}
					
Initial State: 
					
Input/Condition: 
					
Output/Result: 
					
How test will be performed: 
					
\item{test-id2\\}

Type: Functional, Dynamic, Manual, Static etc.
					
Initial State: 
					
Input: 
					
Output: 
					
How test will be performed: 

\end{enumerate}

\subsubsection{Module ?}

...

\subsection{Traceability Between Test Cases and Modules}

\wss{Provide evidence that all of the modules have been considered.}

\bibliographystyle{plainnat}

\bibliography{../../refs/References}

\newpage

\section{Appendix}

This is where you can place additional information.

\subsection{Symbolic Parameters}

The definition of the test cases will call for SYMBOLIC\_CONSTANTS.
Their values are defined in this section for easy maintenance.

\subsection{Usability Survey Questions?}

The purpose of this survey is to validate that Project Proxi meets its
usability, accessibility, and satisfaction goals from the SRS. Responses
will be collected after user testing sessions to evaluate clarity,
efficiency, and inclusiveness. Each question below lists how the response
will be recorded and the intent behind it.

\vspace{0.5em}
\begin{enumerate}

\item \textbf{How easy was it to perform tasks using voice commands?}\\
Response Type: 1–5 scale (Very Hard → Very Easy).\\
Intent: Measure how intuitive and learnable the system is for new users.

\item \textbf{Were system responses clear and understandable?}\\
Response Type: 1–5 scale (Very Unclear → Very Clear).\\
Intent: Assess clarity of feedback and correctness of system output.

\item \textbf{Did the system react quickly to your commands?}\\
Response Type: 1–5 scale (Very Slow → Very Fast).\\
Intent: Validate perceived responsiveness and latency satisfaction.

\item \textbf{How intuitive did the interface feel while completing tasks?}\\
Response Type: 1–5 scale (Confusing → Very Intuitive).\\
Intent: Evaluate the overall user interface flow and predictability.

\item \textbf{Was visual or audio feedback easy to notice and understand?}\\
Response Type: 1–5 scale (Not Accessible → Highly Accessible).\\
Intent: Verify accessibility and multimodal feedback effectiveness.

\item \textbf{Did you feel confident that Proxi understood your intent?}\\
Response Type: 1–5 scale (Never → Always).\\
Intent: Measure trust in speech recognition and NLP accuracy.

\item \textbf{How satisfied are you with the overall experience?}\\
Response Type: 1–5 scale (Very Unsatisfied → Very Satisfied).\\
Intent: Determine overall satisfaction and user acceptance level.

\item \textbf{Did you encounter confusion or frustration while using Proxi?}\\
Response Type: Open text.\\
Intent: Identify specific usability or interaction issues.

\item \textbf{Would you consider using Proxi for daily tasks?}\\
Response Type: Yes / No.\\
Intent: Gauge adoption potential and long-term usability perception.

\item \textbf{Do you have any comments or suggestions for improvement?}\\
Response Type: Open text.\\
Intent: Gather qualitative feedback for future iterations.

\end{enumerate}

\newpage{}
\section*{Appendix --- Reflection}

\wss{This section is not required for CAS 741}

The information in this section will be used to evaluate the team members on the
graduate attribute of Lifelong Learning.

\input{../Reflection.tex}

\begin{enumerate}
  \item What went well while writing this deliverable? 
  \item What pain points did you experience during this deliverable, and how
    did you resolve them?
  \item What knowledge and skills will the team collectively need to acquire to
  successfully complete the verification and validation of your project?
  Examples of possible knowledge and skills include dynamic testing knowledge,
  static testing knowledge, specific tool usage, Valgrind etc.  You should look to
  identify at least one item for each team member.
  \item For each of the knowledge areas and skills identified in the previous
  question, what are at least two approaches to acquiring the knowledge or
  mastering the skill?  Of the identified approaches, which will each team
  member pursue, and why did they make this choice?
\end{enumerate}

\newpage
\section*{Reflection --- Amanbeer Singh Minhas}

\textbf{1. What went well while writing this deliverable?}  
For this deliverable, I worked mainly on Section 3 (Plan) and Section 6.2
(Usability Survey Questions). These sections went well because they helped
connect the technical and human sides of our project. In Section 3, I focused
on creating a clear roadmap for verification that was realistic for our team.
It felt good to see how each part linked to the SRS and our earlier work.
Section 6.2 was also rewarding because it allowed me to design usability
questions that measure user experience in a meaningful way.

\noindent\textbf{2. What pain points did you experience during this deliverable, and how
did you resolve them?}  
The main challenge was figuring out how much detail to include in each
section. At first, some parts of our plan were too general, while others were
too detailed. Finding the right balance took a few rounds of review and
feedback. Another difficulty was maintaining consistency in tone and wording
across sections written by different team members. I helped fix this by
revising and merging ideas so the final version felt smooth and unified.

\noindent\textbf{3. What knowledge and skills will the team collectively need to acquire
to successfully complete the verification and validation of your project?}  
As a team, we realized we need stronger testing and validation skills. I plan
to improve my understanding of automated documentation, coverage metrics, and
survey analysis. Savinay aims to deepen his knowledge of integration testing.
Gourob wants to explore performance and stress testing for our AI modules, and
Ajay is focusing on automated UI and accessibility verification.

\noindent\textbf{4. For each of the knowledge areas and skills identified in the
previous question, what are at least two approaches to acquiring the knowledge
or mastering the skill? Of the identified approaches, which will each team
member pursue, and why did they make this choice?}  
To build the skills we identified, our team plans to rely on both formal and
self-directed learning. We will start by reviewing course material from
SFWRENG 3S03, which covered testing concepts such as unit testing, coverage,
and static analysis. This will help refresh the theoretical foundation before
moving into practical work. We also plan to read online documentation and
tutorials on dynamic testing and usability validation to better understand
industry practices. Each member will then apply this knowledge directly to
our project tasks for example, by creating small prototype tests or running
sample validation sessions. This approach keeps learning active and helps us
improve through real application instead of only theory.

\section*{Reflection --- Ajay Singh Grewal}

\textbf{1. What went well while writing this deliverable?}  
For this deliverable, I worked mainly on section 4.2 to figure out the tests
for non-functional requirements. This section went pretty good for me because
it allowed me to think critically about how to measure different aspects
for our software and how to make sure it essentially meets the needs of our users.

\noindent\textbf{2. What pain points did you experience during this deliverable, and how
did you resolve them?}  
The main challenge I had faced during this deliverable was making sure that the 
test cases for the nonfunctional requirements made sense and were realistic to 
do. To properly do this, I had to really think about what our software was going
to be used for and how we could truly measure the success of these 
nonfunctional requirements.

\noindent\textbf{3. What knowledge and skills will the team collectively need to acquire
to successfully complete the verification and validation of your project?}  
The knowledge and skills that our team will need in order to ensure success
in the verification and validation of our project include dynamic testing
knowledge, static testing knowledge, and certain tool usage. I hope to improve 
my knowledge in various testing methods that can helpy verify and validate our 
projecet.  

\noindent\textbf{4. For each of the knowledge areas and skills identified in the
previous question, what are at least two approaches to acquiring the knowledge
or mastering the skill? Of the identified approaches, which will each team
member pursue, and why did they make this choice?}  
For each of the knowledge areas and skills we indentified, our team plans to 
use a combo of formal and self directed learning. We can start by reviewing 
course material and by reading online articles and tutorials on relevant topics.
Each member will then apply this learnt knowledge directly into the successful
creation of our project. We can also do small prototype tests or run sample 
validation sessions to help us learn better.

\section*{Reflection --- Gourob Podder}

\textbf{1. What went well while writing this deliverable?}  

Writing this deliverable went smoothly once I established a clear mapping
 between each functional requirement and its corresponding tests. The structure 
 of the SRS helped guide our testing logic, and I was able to design specific, 
 measurable, and realistic test cases for all major system components from input 
 handling to agent-based task execution. Collaboratively, we divided responsibilities 
 efficiently, which allowed us to maintain a consistent level of technical depth and 
 clarity throughout the VnV plan.

\noindent\textbf{2. What pain points did you experience during this deliverable, and how did you resolve them?}  

One major challenge we faced was time management. Our group had to apply for an MSAF 
(McMaster Student Absence Form) to obtain academic relief for a late submission due 
to unexpected health issues amongst our group members. Getting approval 
for the MSAF was a bit of a hassle — it required communication with prof and the 
relevant university department. This delay temporarily disrupted our workflow and 
added stress to the deliverable timeline. We resolved it by setting up a stricter internal 
schedule after the approval to get the work completed efficiently.  

\noindent\textbf{3. What knowledge and skills will the team collectively need to acquire to successfully complete the verification and validation of your project?}  
We had to learn a lot on how testing and verification worked for black boxes
like LLMs:  
\begin{itemize}
    \item Testing and evaluation of AI/NLP models (e.g., measuring intent classification 
    accuracy and ASR performance).  
    \item Usability testing with diverse user groups, particularly those with 
    accessibility needs.  
\end{itemize}

\noindent\textbf{4. For each of the knowledge areas and skills identified in the previous question, what are at least two approaches to acquiring the knowledge or mastering the skill? Of the identified approaches, which will each team member pursue, and why did they make this choice?}  

\begin{itemize}
    \item \textbf{AI/NLP Testing and Evaluation:}  
    Approaches include (1) completing online courses or tutorials on NLP evaluation 
    (e.g., Coursera, Hugging Face documentation) and (2) implementing small-scale 
    evaluation scripts for our ASR and intent recognition modules. I will focus 
    on the implementation approach since I already have a strong technical 
    background and prefer hands-on experimentation with model outputs.  

    \item \textbf{Usability and Accessibility Testing:}  
    Approaches include (1) conducting supervised user testing sessions with 
    accessibility-focused participants One of our teammates with a human–computer 
    interaction background will lead user testing, while I will focus on analyzing 
    the qualitative data collected from these sessions to guide improvements.  
\end{itemize}




\section*{Reflection --- Savinay Chhabra}

\textbf{1. What went well while writing this deliverable?}  
The team did a much better job communicating what each team member was doing 
well in advance. This made working on dependent sections a lot easier than 
previous deliverables. This prevented unnecessary back and forth between team 
members which saved significant time.

\noindent\textbf{2. What pain points did you experience during this deliverable, and how
did you resolve them?}  
The biggest challenge with this deliverable was that the entire team was absent 
for the deadline and several days before that as well. We were unclear on how to 
request academic relief which made it even more challenging. In the end, we were
able to work with the Office of the Associate Dean and the course professor to 
get an extension. All this would have been a lot simpler and easier for everyone 
if we requested academic relief well before the deadline.

\noindent\textbf{3. What knowledge and skills will the team collectively need to acquire
to successfully complete the verification and validation of your project?}  
Some basic verficiation and validation skills including coming up with test plans,
scenarios, user groups and correctness are required. This is important as the 
objectives we wish to validate would require comprehensive test cases which 
cover the vast majority of scenarios we would expect to see in the real world. 
For that, knowledge about Unit Testing, Integration Testing, Stability Testing 
and End to End Integration Testing would be very useful to have.

\noindent\textbf{4. For each of the knowledge areas and skills identified in the
previous question, what are at least two approaches to acquiring the knowledge
or mastering the skill? Of the identified approaches, which will each team
member pursue, and why did they make this choice?}  
A good start would be to review prior course material that taught us all of 
these concepts. SFWRENG 3S03 was a very useful course that taught about testing. 
This would help build a strong theoretical foundation. For practical learning, 
going through open source software projects and their tests (both automated and 
manual) would be very helpful. Reading their test plans and contributing to their
testing when feasible would be a great way to get some practical experience.

\end{document}